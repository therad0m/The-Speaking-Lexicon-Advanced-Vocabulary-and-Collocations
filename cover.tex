\begin{titlepage}
   
\begin{center}
\textbf{\LARGE TÀI LIỆU TỔNG HỢP}\\[0.5cm] \vspace{20pt}

\vfill

\textbf{\LARGE  TỪ VỰNG CHUYÊN ANH DÀNH CHO\\ KỸ NĂNG NÓI}\\

\vfill

\textbf {\large Biên soạn:}\\[0.2cm]
\Large {Nguyễn Thái Bảo}\\[0.1cm]
\end{center}

\end{titlepage}

\begin{center}
    
    \large \textbf{TỪ VỰNG CHUYÊN ANH DÀNH CHO KỸ NĂNG NÓI}\\
    \vspace{2mm}
    Mai Thành Sơn$^{1}$, Ngô Thanh Tâm$^{2}$, Phạm Bá Đạt$^{3}$\\
    \vspace{7.5mm}
    Nguyễn Thái Bảo \LaTeX.\\ Đại Học Công Nghệ Thông Tin, ĐHQG TPHCM  

        \begin{center}
		    \textcolor{azul}{\rule{170mm}{0.5mm}}
	    \end{center}	
\end{center}

\section*{LỜI GIỚI THIỆU}
\thispagestyle{empty}
\noindent
Toàn bộ từ vựng trong sách này đều được trích từ sách \emph{TỪ VỰNG CHUYÊN ANH DÀNH CHO KỸ NĂNG NÓI} do thầy Mai Thành Sơn (Chủ biên), thầy Phạm Bá Đạt, cô Ngô Thanh Tâm đồng tác giả.

\vspace{3mm}
\noindent
Tài liệu này được biên soạn nhằm mục đích số hoá và hệ thống lại toàn bộ kho từ vựng,
giúp người học có một nguồn tham khảo thống nhất và tiện lợi. 
Bên cạnh việc giữ nguyên tinh thần của sách gốc, bản biên soạn này còn được 
tổ chức lại theo định dạng rõ ràng, dễ tra cứu và thuận tiện cho việc học tập tự chủ.


\vspace{3mm}
\noindent
Đối tượng hướng tới là học sinh, sinh viên và những ai quan tâm tới việc 
nâng cao kỹ năng nói tiếng Anh một cách học thuật, đặc biệt trong bối cảnh 
luyện thi IELTS và các chứng chỉ quốc tế. 


\vspace{3mm}
\noindent
Người học có thể sử dụng tài liệu này theo nhiều cách: 

\begin{itemize}
    \item Làm nguồn tra cứu khi gặp những chủ đề cần vốn từ vựng nâng cao.
    \item Tự học theo từng phần nhỏ để tích luỹ dần.
    \item Hoặc dùng làm tài liệu ôn luyện hệ thống trước kỳ thi.
\end{itemize}


 

\vspace{3mm}
\noindent
Hi vọng rằng tài liệu này không chỉ hữu ích trong quá trình 
học tập mà còn truyền cảm hứng để người đọc tiếp tục khám phá vẻ đẹp 
và chiều sâu của tiếng Anh.

\begin{flushright}
    {\emph{NTB.}}
\end{flushright}

\newpage
\clearpage
{
  \makeatletter
  \let\ps@plain\ps@empty
  \makeatother
  \pagestyle{empty}
  \tableofcontents
}
\clearpage