\begin{glossarymc}[Cambridge 5]
    \begin{test}{TEST 1}
    \noindent
    \textbf{Part 1. Your Country}
        \begin{qa}{Which part of your country do most people live in?}
        \textbf{To the best of my knowledge}, due to massive urbanization, people \textbf{are predisposed to} flocking to \textbf{metropolis}. As a result, when deciding which part of the country to live in, I believe the majority of city and rural dwellers will not hesitate to opt for the \textbf{bright lights of metropolitan} cities such as Hanoi and Ho Chi Minh City. Residents in these two cities stand a better chance of enjoying healthcare services and winning \textbf{well-paid} jobs.
        \end{qa}

        \begin{qa}{Tell me about the main industries there.}
        \textbf{In my humble opinion}, there are two main industries in each said mega city. The first one is a “\textbf{smoke-free industry}”, namely tourism. Millions of visitors have \textbf{highlighted the importance} of visiting either of the two typical cities of Vietnam. The second is construction. Lots of high-rise buildings such as skyscrapers and apartment blocks have been \textbf{booming} in these two cities to meet the demand of city expansion and accommodation.
        \end{qa}

        \begin{qa}{How easy is it to travel around your country?}
        People generally enjoy several means of transportation to go \textbf{up north} and \textbf{down south}. It is not hard to determine which way suits anyone most. They can choose to \textbf{set off} either by air or by land in the form of airplanes and cars, bicycles, motorbikes or trains, respectively. They can also enjoy \textbf{going afloat} by travelling on ships as well.
        \end{qa}

        \begin{qa}{Has your country changed much since you were a child?}
        My country has experienced \textbf{a paradigm shift} since I was little. Smartphones and computers, either in the form of laptops or PCs, were considered luxurious items in the past. \textbf{In this day and age}, they are products of necessity instead of luxury. Furthermore, Facebook, which did not come into existence in my childhood, has become the largest social networking site in this world, and I suppose more than one fifth of the Vietnamese population are \textbf{hooked}. Now, Facebook \textbf{transcends} Yahoo, one of the most popular sites back then, and the reign of Yahoo has \textbf{come to an end}.
        \end{qa}

        \begin{VocabExplain}[Part 1]
            \begin{ExplainCard}{to the best of my knowledge}[phrase][B2]
            \EN{As far as I know based on what I have learned or verified.}
            \SY{as far as I know; to my knowledge}
            \VI{Theo như tôi biết.}
            \EX{\textit{To the best of my knowledge}, the train leaves at 7.}
            \EX{\textit{To my knowledge}, no official guidelines have been issued.}
            \CO{to the best of my knowledge/ability}
            \end{ExplainCard}

            \begin{ExplainCard}{be predisposed to}[v~phr][C1]
            \EN{To be likely or inclined to do or experience something, often due to prior conditions or tendencies.}
            \SY{be inclined to; tend to; be prone to}
            \VI{Có xu hướng/nghiêng về; dễ bị.}
            \EX{Young graduates \textit{are predisposed to} move to big cities.}
            \EX{Low income groups \textit{are prone to} informal employment.}
            \CO{predisposed to migration/illness/risk-taking}
            \end{ExplainCard}

            \begin{ExplainCard}{metropolis}[n][C1]
            \EN{A very large and important city, often the main city of a region or country.}
            \SY{major city; megacity; urban hub}
            \VI{Đại đô thị; thành phố lớn.}
            \EX{Ho Chi Minh City is a bustling \textit{metropolis}.}
            \EX{Global \textit{urban hubs} attract high-skilled talent.}
            \CO{global/teeming/thriving metropolis}
            \end{ExplainCard}

            \begin{ExplainCard}{the bright lights (of the city)}[idiom][B2]
            \EN{The excitement and attractions of city life that draw people from elsewhere.}
            \SY{urban allure; city attractions}
            \VI{Sức hấp dẫn, ánh hào quang của đô thị.}
            \EX{Many teens chase \textit{the bright lights} after school.}
            \EX{Rural-urban migration is fuelled by the \textit{urban allure}.}
            \CO{seek/chase the bright lights; bright-lights lifestyle}
            \end{ExplainCard}

            \begin{ExplainCard}{well-paid}[adj][B2]
            \EN{Earning a high salary for the work done.}
            \SY{high-paying; lucrative}
            \VI{Lương cao; đãi ngộ tốt.}
            \EX{Tech firms offer \textit{well-paid} roles.}
            \EX{\textit{Lucrative} positions cluster in metropolitan areas.}
            \CO{well-paid job/sector/profession}
            \end{ExplainCard}

            \begin{ExplainCard}{in my humble opinion}[discourse][B2]
            \EN{A polite way to introduce your personal view.}
            \SY{personally; in my view}
            \VI{Theo ý kiến cá nhân tôi.}
            \EX{\textit{In my humble opinion}, tourism drives growth.}
            \EX{\textit{Personally}, transport should come first.}
            \CO{In my humble opinion, + clause}
            \end{ExplainCard}

            \begin{ExplainCard}{smoke-free industry}[n~phr][C1]
            \EN{A non-polluting service sector such as tourism, finance, or education.}
            \SY{clean industry; service industry}
            \VI{Ngành “không khói” (dịch vụ, ít phát thải).}
            \EX{Tourism is a key \textit{smoke-free industry}.}
            \EX{Cities shift from heavy industry to \textit{clean sectors}.}
            \CO{develop/promote smoke-free industries}
            \end{ExplainCard}

            \begin{ExplainCard}{highlight the importance (of)}[v~collocation][B2]
            \EN{To emphasize that something matters a great deal.}
            \SY{underscore; stress; emphasize}
            \VI{Nhấn mạnh tầm quan trọng (của).}
            \EX{The report \textit{highlights the importance of} public transport.}
            \EX{Findings \textit{underscore} early childhood education.}
            \CO{highlight/underscore the importance/significance of}
            \end{ExplainCard}

            \begin{ExplainCard}{booming}[adj][C1]
            \EN{Growing very fast and successfully.}
            \SY{thriving; surging; flourishing}
            \VI{Bùng nổ; tăng trưởng mạnh.}
            \EX{The skyline reflects a \textit{booming} construction sector.}
            \EX{A \textit{surging} startup scene attracts capital.}
            \CO{booming demand/market/industry}
            \end{ExplainCard}

            \begin{ExplainCard}{up north / down south}[adv~phr][B2]
            \EN{Informal ways to refer to travelling toward the northern/southern part of a country.}
            \SY{to the north/south}
            \VI{Ra miền Bắc / vào miền Nam.}
            \EX{We’re heading \textit{up north} this weekend.}
            \EX{They moved \textit{down south} for work.}
            \CO{go/drive/fly up north; move down south}
            \end{ExplainCard}

            \begin{ExplainCard}{set off}[phr.v][B1]
            \EN{Begin a journey.}
            \SY{depart; head out}
            \VI{Khởi hành, lên đường.}
            \EX{We \textit{set off} at dawn to avoid traffic.}
            \EX{Researchers \textit{departed} for the field site at 6 a.m.}
            \CO{set off early/at dawn/for Hanoi}
            \end{ExplainCard}

            \begin{ExplainCard}{go afloat}[v~phr][B2]
            \EN{Travel by water; go by boat or ship.}
            \SY{sail; travel by sea}
            \VI{Đi đường thủy; ra khơi.}
            \EX{Tourists \textit{go afloat} for island-hopping.}
            \EX{Freight \textit{sails} along coastal routes.}
            \CO{go afloat on/along; stay afloat (different meaning)}
            \end{ExplainCard}

            \begin{ExplainCard}{paradigm shift}[n][C1]
            \EN{A fundamental change in approach or underlying assumptions.}
            \SY{sea change; tectonic shift}
            \VI{Sự thay đổi mô hình/tư duy căn bản.}
            \EX{Smartphones triggered a \textit{paradigm shift} in media use.}
            \EX{Remote work marks a \textit{sea change} in management.}
            \CO{experience/undergo a paradigm shift in}
            \end{ExplainCard}

            \begin{ExplainCard}{in this day and age}[idiom][B2]
            \EN{Nowadays; in the present time.}
            \SY{nowadays; today}
            \VI{Thời buổi này; ngày nay.}
            \EX{\textit{In this day and age}, internet access is essential.}
            \EX{Data literacy is vital \textit{today}.}
            \CO{in this day and age, + clause}
            \end{ExplainCard}

            \begin{ExplainCard}{hooked (on)}[adj][B2]
            \EN{Very interested in or addicted to something.}
            \SY{addicted; absorbed; into}
            \VI{Nghiện/thích mê; bị cuốn hút.}
            \EX{Half my friends are \textit{hooked on} short videos.}
            \EX{Users became \textit{addicted} to daily streaks.}
            \CO{hooked on/with; get/become hooked}
            \end{ExplainCard}

            \begin{ExplainCard}{transcend}[v][C1]
            \EN{To be or go beyond the usual limits of something; to surpass.}
            \SY{surpass; outstrip; go beyond}
            \VI{Vượt lên; vượt quá giới hạn.}
            \EX{Facebook \textit{transcends} earlier platforms in reach.}
            \EX{Great design \textit{outstrips} short-term trends.}
            \CO{transcend boundaries/limits/culture}
            \end{ExplainCard}

            \begin{ExplainCard}{come to an end}[idiom][B2]
            \EN{Finish; stop happening.}
            \SY{end; conclude; wind up}
            \VI{Kết thúc; chấm dứt.}
            \EX{The Yahoo era has \textit{come to an end}.}
            \EX{The pilot will \textit{conclude} next quarter.}
            \CO{finally/gradually come to an end}
            \end{ExplainCard}
        \end{VocabExplain}

    \noindent
    \textbf{Part 2.}
    \begin{qa}{Describe a well-known person you like or admire}
        You should say:\\
        \textbullet\ Who this person is\\
        \textbullet\ What this person has done\\
        \textbullet\ Why this person is well known\\
        \textbullet\ and explain why you admire this person.

        \textbf{In all honesty}, I used to work for a world-class event-planning company called BICKY, so I had the opportunity to meet and collaborate with \textbf{countless} famous people. However, the \textbf{celebrity} that has a long-lasting impression on me is Sia, and I would like to talk about her today. I guess the name Sia is familiar to you because she is a \textbf{household name} in the music industry. She is versatile because she is capable of singing, producing music and even directing music videos. She is 45 years old now, but she is young at heart. I guess if you have the chance to meet her \textbf{in person}, you cannot \textbf{take your eyes off} her.

        I first met her when her company \textbf{launched a fund-raising event} called “Cheap Thrills For Kids”. This event lasted for 3 hours, and its purpose was to financially support \textbf{impoverished} children suffering from critical diseases such as cancer or HIV. Because I \textbf{had a flair for} English, I was \textbf{in charge of} interpreting her messages into Vietnamese and vice versa. What struck me is that she had a \textbf{heart of gold}. By chance, she read an article about the high rate of children in Vietnam struggling against cancer or HIV, and she knew that she could do something to \textbf{go to bat for} them.

        The reasons why I admire Sia are that she is a brilliant artist. I \textbf{have a liking for} her music, and I love the messages in her songs. They cheer me up whenever I am \textbf{down}, and they have \textbf{ignited my passion} for becoming a teacher like I am today. I should not forget to mention that she is \textbf{a people person}. Unlike other \textbf{rising artists}, she is very sociable and close to everyone, so there is no gap between her and ordinary people.
        \end{qa}

        \begin{VocabExplain}[Part 2]
            \begin{ExplainCard}{in all honesty}[discourse][B2]
            \EN{Used to introduce a frank, sincere remark.}
            \SY{to be honest; frankly}
            \VI{Thật lòng mà nói; thành thật.}
            \EX{\textit{In all honesty}, I was star-struck when I met her.}
            \EX{\textit{Frankly}, the plan needed a clearer budget.}
            \CO{In all honesty, + clause}
            \end{ExplainCard}

            \begin{ExplainCard}{countless}[adj][B2]
            \EN{So many that it is impossible to count; very numerous.}
            \SY{innumerable; innumerous; myriad}
            \VI{Vô số; không đếm xuể.}
            \EX{He has collaborated with \textit{countless} artists.}
            \EX{There are \textit{myriad} ways to support a cause.}
            \CO{countless times/occasions/examples}
            \end{ExplainCard}

            \begin{ExplainCard}{celebrity}[n][B2]
            \EN{A famous person, especially in entertainment.}
            \SY{star; public figure}
            \VI{Người nổi tiếng.}
            \EX{The \textit{celebrity} hosted a charity livestream.}
            \EX{Brands often hire \textit{public figures} for campaigns.}
            \CO{A-list/local celebrity; celebrity endorsement}
            \end{ExplainCard}

            \begin{ExplainCard}{household name}[n~phrase][C1]
            \EN{A person or brand that is widely known by the public.}
            \SY{icon; widely known name}
            \VI{Cái tên ai cũng biết; tên tuổi quen thuộc.}
            \EX{Sia is a \textit{household name} in pop music.}
            \EX{The vaccine maker became a \textit{household name}.}
            \CO{become/remain a household name; make sb a household name}
            \end{ExplainCard}

            \begin{ExplainCard}{in person}[adv~phrase][B1]
            \EN{Physically present rather than online or via media.}
            \SY{face to face; on site}
            \VI{Trực tiếp; gặp mặt.}
            \EX{I finally met her \textit{in person} backstage.}
            \EX{Applicants must submit forms \textit{in person}.}
            \CO{meet/see/verify in person}
            \end{ExplainCard}

            \begin{ExplainCard}{take your eyes off (sb/sth)}[idiom][B2]
            \EN{Stop looking at someone or something because it is very attractive or striking.}
            \SY{look away from; stop staring at}
            \VI{Không thể rời mắt khỏi.}
            \EX{It was hard to \textit{take my eyes off} her performance.}
            \EX{You can’t \textit{look away from} that stage design.}
            \CO{can/can’t take one’s eyes off}
            \end{ExplainCard}

            \begin{ExplainCard}{launch a fund-raising event}[v~phrase][B2]
            \EN{To start an organised activity to collect money for a cause.}
            \SY{organise a charity drive; kick off a fundraiser}
            \VI{Khởi động sự kiện gây quỹ.}
            \EX{The label \textit{launched a fund-raising event} for kids’ cancer.}
            \EX{They \textit{kicked off a fundraiser} after the floods.}
            \CO{launch/run/host a fund-raising (campaign/event)}
            \end{ExplainCard}

            \begin{ExplainCard}{impoverished}[adj][C1]
            \EN{Extremely poor; lacking financial resources.}
            \SY{destitute; underprivileged}
            \VI{Nghèo khó; túng thiếu.}
            \EX{The charity supports \textit{impoverished} children.}
            \EX{\textit{Underprivileged} families received scholarships.}
            \CO{impoverished communities/areas/families}
            \end{ExplainCard}

            \begin{ExplainCard}{have a flair for}[v~phrase][B2]
            \EN{To have a natural ability or talent for something.}
            \SY{have a knack for; have an aptitude for}
            \VI{Có năng khiếu về.}
            \EX{She \textit{has a flair for} languages.}
            \EX{Students with an \textit{aptitude for} design flourish here.}
            \CO{flair for writing/teaching/negotiation}
            \end{ExplainCard}

            \begin{ExplainCard}{in charge of}[prep~phrase][B2]
            \EN{Responsible for something or someone.}
            \SY{responsible for; in control of}
            \VI{Phụ trách; chịu trách nhiệm.}
            \EX{I was \textit{in charge of} interpreting.}
            \EX{She is \textit{responsible for} event logistics.}
            \CO{be/put sb in charge of; take charge of}
            \end{ExplainCard}

            \begin{ExplainCard}{a heart of gold}[idiom][B2]
            \EN{A very kind and generous nature.}
            \SY{kind-heartedness; generosity}
            \VI{Tấm lòng vàng; rất tốt bụng.}
            \EX{Despite the fame, she has \textit{a heart of gold}.}
            \EX{Donors with \textit{big hearts} kept the programme alive.}
            \CO{have/show a heart of gold}
            \end{ExplainCard}

            \begin{ExplainCard}{go to bat for (sb)}[idiom][C1]
            \EN{Actively support or defend someone.}
            \SY{stand up for; advocate for}
            \VI{Đứng ra bảo vệ/đấu tranh vì ai.}
            \EX{She \textit{went to bat for} sick children in Vietnam.}
            \EX{Colleagues \textit{advocated for} fair pay.}
            \CO{go to bat for a colleague/cause}
            \end{ExplainCard}

            \begin{ExplainCard}{have a liking for}[v~phrase][B2]
            \EN{To enjoy or be fond of something.}
            \SY{be fond of; have a taste for}
            \VI{Thích; có cảm tình với.}
            \EX{I \textit{have a liking for} her lyrics.}
            \EX{Many teens \textit{have a taste for} indie pop.}
            \CO{have a liking for music/spicy food/sci-fi}
            \end{ExplainCard}

            \begin{ExplainCard}{down}[adj][B1]
            \EN{Feeling sad or less happy than usual.}
            \SY{blue; low; depressed (mild)} 
            \VI{Buồn; xuống tinh thần.}
            \EX{Her songs lift me up when I’m \textit{down}.}
            \EX{He felt \textit{low} after the setback.}
            \CO{feel/get/be down; down mood/period}
            \end{ExplainCard}

            \begin{ExplainCard}{ignite (one’s) passion}[v~phrase][C1]
            \EN{To make someone feel a strong enthusiasm for something.}
            \SY{spark; kindle; fuel}
            \VI{Châm ngòi/khơi dậy đam mê.}
            \EX{Her story \textit{ignited my passion} for teaching.}
            \EX{A mentor can \textit{spark} students’ curiosity.}
            \CO{ignite/spark/rekindle a passion for}
            \end{ExplainCard}

            \begin{ExplainCard}{a people person}[n~phrase][B2]
            \EN{Someone who enjoys and is good at dealing with other people.}
            \SY{sociable person; extrovert}
            \VI{Người giỏi giao tiếp; thân thiện.}
            \EX{She’s \textit{a people person}—fans feel at ease with her.}
            \EX{Customer roles suit \textit{sociable} graduates.}
            \CO{be/need a people person for (role)}
            \end{ExplainCard}

            \begin{ExplainCard}{rising artist}[n~phrase][B2]
            \EN{A performer whose popularity and career are growing quickly.}
            \SY{up-and-coming artist; emerging talent}
            \VI{Nghệ sĩ đang lên; tài năng mới nổi.}
            \EX{Unlike many \textit{rising artists}, she stays approachable.}
            \EX{The festival showcases \textit{emerging talent}.}
            \CO{rising/young/emerging artist; up-and-coming singer}
            \end{ExplainCard}
        \end{VocabExplain}

    \noindent
    \textbf{Part 3.}
    \begin{qa}{What kind of people become famous people these days?}
    For the most part, those with \textbf{exceptional} talents or \textbf{remarkable} achievements will be \textbf{in the lime light}. For example, a singer with hit songs can easily build a \textbf{sizeable fan base}. Other examples of those who make the headlines might be scientists. Who deserves to be awarded Nobel prize must come up with \textbf{life-changing} medical breakthroughs. Last but not least, once a “Youtuber” can create something bizzare but interesting, his or her videos might \textbf{go viral} and their creator can \textbf{make a name for himself or herself}.
    \end{qa}

    \begin{qa}{Is this different from the kind of achievement that made people famous in the past? In what way?}
    Long time ago, becoming \textbf{celebrities} seemed easier than it is nowadays. In the early 20th century, for example, \textbf{pioneers} with \textbf{world-changing} products such as computers or telephones, \textbf{laid a foundation} for human life. However, in recent years, the market has become \textbf{saturated}, leaving \textbf{slim chances} for \textbf{successors} to achieve \textbf{fame and fortune} despite the enhancement of products.
    \end{qa}

    \begin{qa}{How do you think people will become famous in the future?}
    No matter what era are we in now, I believe that fame should be achieved from \textbf{dedication} and \textbf{conscientiousness}. Even if it might take time for some to be \textbf{blooming}, their \textbf{working their tail off} will \textbf{pay off} one day. Besides, we are living in an \textbf{egalitarian society} which people are given equality to fulfil their \textbf{desires}.
    \end{qa}

    \begin{qa}{What are the good things about being famous?}
    Without a doubt, a \textbf{secured income} and a \textbf{privileged status} are two first and foremost advantages. By this, I mean a celebrity can satisfy their own material needs \textbf{with ease} like \textbf{squandering} their money on \textbf{pricey} items or travelling around the world without much thinking. In general, unlike ordinary people with a \textbf{meager} income to get by every day, a \textbf{household name} can freely enjoy life in a \textbf{luxurious} way.
    \end{qa}

    \begin{qa}{Are there any disadvantages?}
    I would say yes because everything has a \textbf{trade-off}. Becoming a \textbf{prominent} figure is likely to come along with pressure and anxiety. I mean, that person must \textbf{live up to} the expectations of the public, and cannot \textbf{reveal} any \textbf{nasty} habits as they may destroy his career. \textbf{Coming under the spotlight} is another downside that the celebrities would suffer due to their massive popularity. The very thought of having \textbf{legions} of paparazzis \textbf{stick their nose in} the celebrity’s affairs has broken my heart already.
    \end{qa}

    \begin{qa}{How does the media in your country treat famous people?}
    To be honest, I am not really into entertainment news, but it is \textbf{noticeable} that information about famous figures is usually on the top list of \textbf{press coverage}. This is because their stories or even scandals are \textbf{striking} enough for publishers to attract readers and increase daily \textbf{circulation}. Moreover, online media might take advantage of these \textbf{crowd pullers’} stories to increase the \textbf{viewership} ratings.
    \end{qa}

    \begin{qa}{Why do you think ordinary people are interested in the lives of famous people?}
    Understandably, that average people pay much attention into the lives of their idols or super stars can be seen as a \textbf{token} of sincere admiration and \textbf{esteem} to their icons. The private lives of famous people never fail to \textbf{fascinate} the public. However, this practice should be limited to some extent in order not to \textbf{invade} the \textbf{privacy} of the \textbf{famed} ones.
    \end{qa}

    \begin{VocabExplain}[Part 3]
        \begin{ExplainCard}{exceptional}{adj}
        \EN{(1) unusually good; outstanding. (2) very unusual; rare.}
        \SY{outstanding; extraordinary; singular}
        \VI{(1) \textit{xuất chúng}, vượt trội. (2) \textit{hiếm có}, khác thường.}
        \EX{Her performance was truly exceptional.}
        \EX{Such exceptionally cold weather is rare for this region.}
        \CO{exceptional talent; exceptional circumstances; truly exceptional}
        \end{ExplainCard}

        \begin{ExplainCard}{remarkable}{adj}
        \EN{unusual or surprising in a way that attracts attention or admiration.}
        \SY{notable; extraordinary; striking}
        \VI{\textit{đáng chú ý}, khác thường theo cách gây ấn tượng.}
        \EX{He made a remarkable recovery after the operation.}
        \EX{It’s a remarkable achievement for a small team.}
        \CO{remarkable achievement; quite remarkable; remarkable progress}
        \end{ExplainCard}

        \begin{ExplainCard}{sizeable}{adj}
        \EN{fairly large in amount, number, or extent.}
        \SY{substantial; considerable; hefty}
        \VI{\textit{khá lớn}, đáng kể.}
        \EX{They raised a sizeable sum for charity.}
        \EX{A sizeable proportion of users prefer the new design.}
        \CO{sizeable amount; sizeable audience; sizeable proportion}
        \end{ExplainCard}

        \begin{ExplainCard}{life-changing}{adj}
        \EN{having such a strong effect that it significantly alters someone’s life.}
        \SY{transformative; momentous; pivotal}
        \VI{\textit{thay đổi cuộc đời}, có tác động lớn.}
        \EX{Studying abroad was a life-changing experience.}
        \EX{A single life-changing decision can set a new course.}
        \CO{life-changing experience; life-changing decision; truly life-changing}
        \end{ExplainCard}

        \begin{ExplainCard}{go viral}{phrase}
        \EN{to spread extremely quickly online through social platforms and messaging.}
        \SY{blow up; explode online; trend widely}
        \VI{\textit{lan truyền chóng mặt} trên Internet.}
        \EX{The clip went viral overnight.}
        \EX{Brands hope their campaigns go viral before launch day.}
        \CO{go viral overnight; viral sensation; viral trend}
        \end{ExplainCard}

        \begin{ExplainCard}{make a name for oneself}{idiom}
        \EN{to become well known for one’s achievements.}
        \SY{build a reputation; rise to prominence; gain recognition}
        \VI{\textit{tạo dựng tên tuổi}, trở nên nổi tiếng nhờ thành tựu.}
        \EX{She made a name for herself in fintech.}
        \EX{Young researchers are making a name for themselves with bold ideas.}
        \CO{make a name for oneself in {tech/film/science}}
        \end{ExplainCard}

        \begin{ExplainCard}{celebrity}{n}
        \EN{a famous person, especially in entertainment or media.}
        \SY{star; public figure; A-lister}
        \VI{\textit{người nổi tiếng}, đặc biệt trong giới giải trí.}
        \EX{The celebrity arrived to enthusiastic fans.}
        \EX{Celebrity culture influences consumer trends.}
        \CO{celebrity status; celebrity endorsement; A-list celebrity}
        \end{ExplainCard}

        \begin{ExplainCard}{lay foundation}{phrase}
        \EN{to create the basic groundwork for something future to be built upon.}
        \SY{lay the groundwork; pave the way; establish the basis}
        \VI{\textit{đặt nền tảng} cho điều gì về sau.}
        \EX{Early savings lay foundation for financial security.}
        \EX{This research laid foundation for later breakthroughs.}
        \CO{lay the foundation for; solid foundation; foundational work}
        \end{ExplainCard}

        \begin{ExplainCard}{saturated}{adj}
        \EN{(1) filled completely so nothing more can be added. (2) (market) crowded with competitors or products.}
        \SY{(1) replete; swamped \quad (2) oversupplied; crowded}
        \VI{(1) \textit{bão hoà}, đầy kín. (2) (thị trường) \textit{quá tải đối thủ/sản phẩm}.}
        \EX{The city is saturated with billboards.}
        \EX{It’s hard to grow in a saturated market.}
        \CO{saturated market; saturated with ads; market saturation}
        \end{ExplainCard}

        \begin{ExplainCard}{a slim chance}{phrase}
        \EN{a very small possibility of something happening.}
        \SY{remote chance; long shot; faint possibility}
        \VI{\textit{cơ hội rất thấp}, mong manh.}
        \EX{There’s a slim chance of rain today.}
        \EX{He knew his appeal had only a slim chance of success.}
        \CO{a slim chance of; slim possibility; a long shot}
        \end{ExplainCard}

        \begin{ExplainCard}{successor}{n}
        \EN{a person or thing that takes another’s place or position.}
        \SY{heir; replacement; follow-up}
        \VI{\textit{người/kẻ kế nhiệm}, vật kế tiếp.}
        \EX{She was named the CEO’s successor.}
        \EX{This model is the direct successor to last year’s phone.}
        \CO{successor to {the throne/the role}; immediate successor}
        \end{ExplainCard}

        \begin{ExplainCard}{fame and fortune}{phrase}
        \EN{the state of being widely known and having great wealth.}
        \SY{stardom and wealth; renown and riches; celebrity and money}
        \VI{\textit{danh vọng và tiền bạc}.}
        \EX{Many move to big cities seeking fame and fortune.}
        \EX{Fame and fortune don’t guarantee happiness.}
        \CO{seek fame and fortune; path to fame and fortune}
        \end{ExplainCard}

        \begin{ExplainCard}{dedication}{n}
        \EN{persistent effort and commitment to a purpose viewed as important.}
        \SY{commitment; devotion; diligence}
        \VI{\textit{sự tận tuỵ}, miệt mài với mục tiêu.}
        \EX{Her dedication to research inspired the team.}
        \EX{Success often reflects years of quiet dedication.}
        \CO{dedication to {work/study}; show dedication; tireless dedication}
        \end{ExplainCard}

        \begin{ExplainCard}{conscientiousness}{n}
        \EN{the quality of being careful, thorough, and responsible in doing tasks.}
        \SY{meticulousness; diligence; scrupulousness}
        \VI{\textit{sự chu đáo/cẩn trọng}, làm việc có trách nhiệm.}
        \EX{Conscientiousness predicts reliability at work.}
        \EX{His conscientiousness shows in every detail.}
        \CO{high conscientiousness; conscientious worker; trait of conscientiousness}
        \end{ExplainCard}

        \begin{ExplainCard}{blooming}{adj}
        \EN{(of a person) looking healthy, energetic, and attractive.}
        \SY{radiant; flourishing; glowing}
        \VI{\textit{hồng hào, tràn đầy sức sống}.}
        \EX{She looks blooming after the holiday.}
        \EX{With rest and good food, he’s blooming again.}
        \CO{look blooming; blooming with health; positively blooming}
        \end{ExplainCard}

        \begin{ExplainCard}{work one’s tail off}{idiom}
        \EN{to work extremely hard.}
        \SY{graft; toil; grind}
        \VI{\textit{làm việc cật lực}, hết sức mình.}
        \EX{We worked our tails off to hit the deadline.}
        \EX{She’s been working her tail off for the exams.}
        \CO{work {my/your} tail off; work like crazy; put in long hours}
        \end{ExplainCard}

        \begin{ExplainCard}{pay off}{phrase}
        \EN{(1) to bring good results; succeed. (2) to finish paying a debt.}
        \SY{(1) bear fruit; pan out \quad (2) settle; clear}
        \VI{(1) \textit{được đền đáp}, thành công. (2) \textit{trả hết} nợ.}
        \EX{Years of practice finally paid off.}
        \EX{We aim to pay off the loan within five years.}
        \CO{hard work pays off; pay off in the long run; pay off a loan}
        \end{ExplainCard}

        \begin{ExplainCard}{egalitarian society}{phrase}
        \EN{a society organized to promote and protect equality of status and opportunity.}
        \SY{equal society; classless ideal; merit-based order}
        \VI{\textit{xã hội bình đẳng}, đề cao sự công bằng.}
        \EX{Policies aim to build a more egalitarian society.}
        \EX{Education access is central to an egalitarian society.}
        \CO{build an egalitarian society; egalitarian values; egalitarian ideals}
        \end{ExplainCard}

        \begin{ExplainCard}{desire}{n}
        \EN{a strong wish to have or to do something.}
        \SY{longing; yearning; aspiration}
        \VI{\textit{sự khao khát}, mong muốn mạnh mẽ.}
        \EX{Her desire to help others shaped her career.}
        \EX{He expressed a desire for constructive feedback.}
        \CO{desire to {learn/help}; desire for {success/freedom}; strong desire}
        \end{ExplainCard}

        \begin{ExplainCard}{privileged status}{phrase}
        \EN{a position that carries special rights or advantages not available to most people.}
        \SY{special standing; favored position; advantage}
        \VI{\textit{địa vị đặc quyền}, có lợi thế so với số đông.}
        \EX{Not everyone enjoys a privileged status in society.}
        \EX{Such privileges can undermine fairness.}
        \CO{enjoy a privileged status; privileged position; social privilege}
        \end{ExplainCard}

        \begin{ExplainCard}{with ease}{phrase}
        \EN{easily; without difficulty or much effort.}
        \SY{effortlessly; readily; smoothly}
        \VI{\textit{một cách dễ dàng}, không tốn nhiều sức.}
        \EX{She solved the puzzle with ease.}
        \EX{They passed the test with ease.}
        \CO{do {sth} with ease; handle with ease; glide with ease}
        \end{ExplainCard}

        \begin{ExplainCard}{squander}{v}
        \EN{to waste money, time, or opportunities in a careless or foolish way.}
        \SY{waste; fritter away; blow}
        \VI{\textit{phung phí/hoang phí} tiền bạc, thời gian, cơ hội.}
        \EX{Don’t squander this scholarship opportunity.}
        \EX{He squandered his savings on pricey gadgets.}
        \CO{squander money; squander time; squander an opportunity}
        \end{ExplainCard}

        \begin{ExplainCard}{pricey}{adj}
        \EN{costing a lot of money; expensive.}
        \SY{costly; steep; high-priced}
        \VI{\textit{đắt đỏ}.}
        \EX{That restaurant is a bit pricey for students.}
        \EX{The jacket’s stylish but pricey.}
        \CO{a bit pricey; pricey item; pricey purchase}
        \end{ExplainCard}

        \begin{ExplainCard}{trade-off}{n}
        \EN{a balance between two desirable but incompatible things; a compromise.}
        \SY{compromise; balancing act; quid pro quo}
        \VI{\textit{sự đánh đổi}, cân bằng giữa hai điều khó có đồng thời.}
        \EX{There’s a trade-off between speed and accuracy.}
        \EX{Design often involves trade-offs.}
        \CO{make a trade-off; trade-off between {A} and {B}}
        \end{ExplainCard}

        \begin{ExplainCard}{prominent}{adj}
        \EN{(1) important and well known. (2) easily noticeable or sticking out.}
        \SY{(1) notable; leading \quad (2) conspicuous; pronounced}
        \VI{(1) \textit{nổi bật, có tiếng}. (2) \textit{dễ thấy}, lồi rõ.}
        \EX{She is a prominent figure in AI ethics.}
        \EX{A prominent scar runs across his eyebrow.}
        \CO{prominent figure; prominent role; prominent feature}
        \end{ExplainCard}

        \begin{ExplainCard}{reveal}{v}
        \EN{to make something known or visible that was hidden or unknown.}
        \SY{disclose; unveil; expose}
        \VI{\textit{để lộ/tiết lộ} điều trước đó chưa rõ.}
        \EX{The report revealed serious flaws.}
        \EX{He refused to reveal his sources.}
        \CO{reveal the truth; reveal details; reveal findings}
        \end{ExplainCard}

        \begin{ExplainCard}{nasty}{adj}
        \EN{very unpleasant, offensive, or unkind.}
        \SY{unpleasant; mean; disagreeable}
        \VI{\textit{xấu tính/khó chịu}, tệ hại.}
        \EX{Online trolls left nasty comments.}
        \EX{We ran into a nasty storm on the way.}
        \CO{nasty comment; nasty surprise; nasty habit}
        \end{ExplainCard}

        \begin{ExplainCard}{coming under the spotlight}{phrase}
        \EN{being intensely examined by the public or journalists.}
        \SY{under scrutiny; in the limelight; in the glare}
        \VI{\textit{trở thành tâm điểm chú ý} của dư luận và báo chí.}
        \EX{The charity’s spending is coming under the spotlight.}
        \EX{After the award, her research came under the spotlight.}
        \CO{come under the spotlight; thrust into the spotlight; bring {sth} into the spotlight}
        \end{ExplainCard}

        \begin{ExplainCard}{legions of}{phrase}
        \EN{very many; great numbers of (people or things).}
        \SY{multitudes of; hordes of; countless}
        \VI{\textit{rất nhiều}, vô số.}
        \EX{Legions of fans queued outside the venue.}
        \EX{There are legions of reasons to act now.}
        \CO{legions of fans; legions of followers; legions of problems}
        \end{ExplainCard}

        \begin{ExplainCard}{stick one’s nose in}{idiom}
        \EN{to involve oneself intrusively in matters that are not one’s business.}
        \SY{meddle; pry; interfere}
        \VI{\textit{chõ mũi vào chuyện người khác}.}
        \EX{Don’t stick your nose in my private affairs.}
        \EX{He’s always sticking his nose in office politics.}
        \CO{stick your nose into {sb’s business}; pry into; meddle in}
        \end{ExplainCard}

        \begin{ExplainCard}{noticeable}{adj}
        \EN{easy to see or detect; clear enough to be observed.}
        \SY{evident; conspicuous; discernible}
        \VI{\textit{dễ nhận thấy}, rõ rệt.}
        \EX{There’s a noticeable improvement in her writing.}
        \EX{The difference is barely noticeable to beginners.}
        \CO{noticeable difference; readily noticeable; become noticeable}
        \end{ExplainCard}

        \begin{ExplainCard}{press coverage}{n}
        \EN{reporting about something in newspapers, magazines, and other media.}
        \SY{media coverage; news reporting; publicity}
        \VI{\textit{báo chí đưa tin}, sự đưa tin của truyền thông.}
        \EX{The event received extensive press coverage.}
        \EX{Negative press coverage hurt the campaign.}
        \CO{extensive press coverage; front-page coverage; positive coverage}
        \end{ExplainCard}

        \begin{ExplainCard}{striking}{adj}
        \EN{very attractive or impressive, often because it is unusual or vivid.}
        \SY{eye-catching; arresting; dramatic}
        \VI{\textit{gây ấn tượng}, nổi bật.}
        \EX{The building’s striking design draws visitors.}
        \EX{She has striking features and a confident presence.}
        \CO{striking contrast; striking features; visually striking}
        \end{ExplainCard}

        \begin{ExplainCard}{circulation}{n}
        \EN{the usual number of copies a newspaper or magazine sells in a period.}
        \SY{distribution; readership; sales}
        \VI{\textit{lượng phát hành} báo/tạp chí.}
        \EX{The paper’s circulation doubled this year.}
        \EX{They redesigned to boost circulation.}
        \CO{newspaper circulation; weekly circulation; boost circulation}
        \end{ExplainCard}

        \begin{ExplainCard}{token}{n}
        \EN{something that represents a feeling, fact, or event; a symbol or small sign.}
        \SY{symbol; emblem; keepsake}
        \VI{\textit{vật tượng trưng/dấu hiệu}, kỷ vật.}
        \EX{Please accept this gift as a token of our appreciation.}
        \EX{The apology felt like a token gesture.}
        \CO{a token of appreciation; token gesture; token gift}
        \end{ExplainCard}

        \begin{ExplainCard}{esteem}{n}
        \EN{great respect and admiration; a favorable opinion of someone.}
        \SY{respect; regard; admiration}
        \VI{\textit{sự tôn kính/đánh giá cao}.}
        \EX{She is held in high esteem by colleagues.}
        \EX{Volunteering raised his esteem in the community.}
        \CO{hold in high esteem; earn esteem; self-esteem}
        \end{ExplainCard}

        \begin{ExplainCard}{invade}{v}
        \EN{(1) to enter a place in large numbers or by force, causing harm or disorder. (2) to intrude upon someone’s space or privacy.}
        \SY{(1) overrun; occupy \quad (2) intrude on; encroach on}
        \VI{(1) \textit{xâm lược/xâm chiếm}. (2) \textit{xâm phạm} không gian/quyền riêng tư.}
        \EX{Tourists can invade the beaches in peak season.}
        \EX{He felt the camera invaded his privacy.}
        \CO{invade a country; invade privacy; invading forces}
        \end{ExplainCard}
    \end{VocabExplain}


    \begin{VocabHighlights}
        \VH{to the best of my knowledge}{(phrase) based on what you know/ believe, but you are not completely sure}{(cụm từ) theo những gì tôi biết}
        \VH{to be predisposed to V-ing}{(adj) likely to think, feel, or behave in a particular way}{(tính từ) có xu hướng}
        \VH{the bright lights}{(idiom) a big city where you can have an exciting life and be successful}{(thành ngữ) chốn phồn hoa}
        \VH{metropolis}{(n) a big city, especially considered as somewhere that is very busy and exciting}{(danh từ) thành phố lớn}
        \VH{metropolitan}{(adj) belonging to a big city, or typical of big cities}{(tính từ) thuộc về thành phố lớn}
        \VH{well-paid}{(adj) pays a lot of money}{(tính từ) trả lương cao}
        \VH{in my humble opinion}{(phrase) used for expressing your opinion about something, especially when you are sure}{(cụm từ) theo ý tôi thì}
        \VH{smoke-free}{(adj) in which no people are smoking, or in which smoking is not allowed}{(tính từ) không khói}
        \VH{boom}{(v) experience a period of economic success, with a lot of trade and business activity}{(động từ) bùng nổ}
        \VH{up north}{(idiom) to/in, or at the northern part of a country or the world.}{(thành ngữ) về phía bắc}
        \VH{down south}{(idiom) to or in the south of a country}{(thành ngữ) về phía nam}
        \VH{to set off}{(v) to start a journey}{(động từ) bắt đầu đi}
        \VH{to go afloat}{(phrase) floating in water; not sinking}{(cụm từ) trôi lênh đênh}
        \VH{a paradigm shift}{(phrase) a dramatic change}{(cụm từ) sự thay đổi lớn lao}
        \VH{in this day and age}{(phrase) at the present time; in the modern era}{(cụm từ) trong thời đại mới này}
        \VH{to be hooked}{(adj) enjoying something so much that you are unable to stop having it, watching it, doing it, etc}{(tính từ) thích mê}
        \VH{to transcend}{(v) to be or go beyond the usual limits of something}{(động từ) vượt xa khỏi}
        \VH{to come to an end}{(phrase) to stop or end}{(cụm từ) chấm dứt}
        \VH{in all honesty}{(phrase) expressing your opinion truthfully}{(cụm từ) thú thật là}
        \VH{countless}{(adj) very many, or too many to be counted}{(tính từ) không thể đếm được}
        \VH{a household name}{(idiom) a famous person}{(thành ngữ) một người nổi tiếng}
        \VH{in person}{(idiom) directly}{(thành ngữ) trực tiếp}
        \VH{not take your eyes off}{(idom) to stop looking at someone or something}{(thành ngữ) không thể rời mắt}
        \VH{to launch a fund-raising event}{(phrase) an event is organized to receive more money from people}{(cụm từ) thực hiện 1 sự kiện khuyến góp, gây quỹ}
        \VH{impoverished}{(adj) very poor}{(tính từ) nghèo đói}
        \VH{to have a flair for}{(phrase) natural ability to do something well}{(cụm từ) có khả năng thiên phú}
        \VH{to be in charge of}{(phrase) in control or with overall responsibility}{(cụm từ) chịu trách nhiệm}
        \VH{a heart of gold}{(phrase) to be very kind}{(cụm từ) tốt bụng}
        \VH{to go to bat for somebody}{(idiom) to support somebody}{(thành ngữ) ủng hộ ai}
        \VH{to have a liking for}{(phrase) like something}{(cụm từ) yêu thích}
        \VH{to ignite my passion}{(phrase) to explode powerful feeling}{(cụm từ) đốt cháy đam mê}
        \VH{to be down}{(phr.v) be in a bad mood}{(thành ngữ) xuống tinh thần}
        \VH{a people person}{(idiom) someone who is good at dealing with other people}{(thành ngữ) người giỏi giao tiếp, hoà đồng với mọi người}
        \VH{rising artists}{(phrase) some famous people are attracting more attention from the public}{(cụm từ) nghệ sĩ đang nổi}
        \VH{exceptional}{(adj) very unusual}{(tính từ) phi thường, xuất chúng}
        \VH{remarkable}{(adj) unusual or surprising in a way that causes people to take notice}{(tính từ) đáng chú ý}
        \VH{sizeable}{(adj) fairly large}{(tính từ) khá lớn}
        \VH{life-changing}{(adj) having an effect that is strong enough to change someone’s life}{(tính từ) có ảnh hưởng làm thay đổi cuộc sống}
        \VH{to go viral}{(phrase) spread quickly and widely on the Internet through social media and e-mail}{(cụm từ) lan truyền nhanh}
        \VH{to make a name for oneself}{(idiom) to become popular}{(thành ngữ) trở nên nổi tiếng}
        \VH{a celebrity}{(n) a famous person}{(danh từ) người nổi tiếng}
        \VH{to lay foundation}{(phrase) to create a base for something}{(cụm từ) đặt nền tảng cho cái gì đó}
        \VH{saturated}{(adj) to fill a thing or place completely so that no more can be added}{(tính từ) bão hoà}
        \VH{a slim chance}{(phrase) a very small possibility}{(cụm từ) cơ hội rất thấp}
        \VH{successor}{(n) a person or thing that comes after somebody/something else and takes their/its place}{(danh từ) người kế tục; người kế tiếp}
        \VH{fame and fortune}{(phrase) the state of being well-known and having a large amount of money}{(cụm từ) danh vọng và tiền bạc}
        \VH{dedication}{(n) the hard work and effort that somebody puts into an activity or a purpose because they think it is important}{(danh từ) sự tận tuỵ, chăm chỉ}
        \VH{conscientiousness}{(n) the quality of doing things carefully and correctly}{(danh từ) sự chu đáo}
        \VH{blooming}{(adj) a person who is blooming has a healthy, energetic, and attractive appearance}{(tính từ) thành công}
        \VH{to work one’s tail off}{(idiom) to work really hard}{(thành ngữ) làm việc rất chăm chỉ}
        \VH{to pay off}{(phrase) a course of action you have done yields good results or to be successful}{(cụm từ) đền đáp xứng đáng}
        \VH{an egalitarian society}{(phrase) a society that promotes equality}{(cụm từ) một xã hội bình đẳng}
        \VH{desire}{(n) a strong wish to have or do something}{(danh từ) sự khao khát}
        \VH{a privileged status}{(phrase) a status that have special rights or advantages that most people do not have}{(cụm từ) địa vị đặc quyền}
        \VH{with ease}{(phrase) if you do something with ease, you do it easily, without difficulty or effort}{(cụm từ) dễ dàng}
        \VH{to squander}{(v) to waste money, time, etc. in a stupid or careless way}{(động từ) phung phí; hoang phí}
        \VH{pricey}{(adj) very expensive}{(tính từ) đắt đỏ}
        \VH{a trade-off}{(noun) a balance achieved between two desirable but incompatible features; a compromise}{(danh từ) sự đánh đổi}
        \VH{prominent}{(adj) important and well-known}{(tính từ) nổi bật, xuất chúng}
        \VH{to reveal}{(v) to show something that previously could not be seen}{(động từ) để lộ ra; tiết lộ}
        \VH{nasty}{(adj) very bad or unpleasant}{(tính từ) không tốt; xấu tính}
        \VH{coming under the spotlight}{(phrase) if someone or something comes under the spotlight, they are thoroughly examined, especially by journalists and the public}{(cụm từ) trở thành tâm điểm chú ý của cộng đồng và báo chí}
        \VH{legions of}{(phrase) lots of (people)}{(cụm từ) nhiều người}
        \VH{to stick one’s nose in}{(idiom) involve oneself in an intrusive or nosy manner into something that is not one’s business or responsibility}{(thành ngữ) chõ mũi vào việc người khác}
        \VH{noticeable}{(adj) easy to see or notice, clear or definite}{(tính từ) dễ nhận thấy}
        \VH{press coverage}{(n) reports about something in newspapers, and magazines and other media}{(danh từ) báo chí đưa tin}
        \VH{striking}{(adj) very attractive, often in an unusual way}{(tính từ) gây ấn tượng}
        \VH{circulation}{(n) the usual number of copies of a newspaper or magazine that are sold each day, week, etc.}{(danh từ) lượng phát hành}
        \VH{a token}{(n) something that is a symbol of feeling, a fact, an event}{(danh từ) biểu tượng}
        \VH{esteem}{(n) great respect and admiration; a good opinion of somebody}{(danh từ) sự tôn kính, ngưỡng mộ}
        \VH{to invade}{(v) to enter a place in large numbers, especially in a way that causes damage or confusion}{(động từ) xâm phạm}
    \end{VocabHighlights}

    \end{test}

    \begin{test}{TEST 2}
    \noindent
    \textbf{Part 1. Colour}
    \begin{qa}{What's your favourite colour? (Why?)}{The color of my choice is red, especially the shade of scarlet. It symbolizes the national Vietnamese flag. In particular, the red color of the flag \textbf{embodies} the blood \textbf{shed} by numerous generations of \textbf{forefathers} to \textbf{secure} the freedom and sovereignty for the whole nation after hundreds of years in wars. Besides, red is also the traditional color of my favorite football team, Manchester United.}
    \end{qa}

    \begin{qa}{Do you like the same colours now as you did when you were younger? (Why/Why not?)}{Yes, I do. Red, as I said earlier, represents the color of Vietnam's national flag and Manchester United's jersey so it is my \textbf{all-time} favorite color. However, recently, I have a \textbf{predilection} for wearing dark colors such as black and navy blue. The reason is because I am \textbf{humongous}, I wish to look thinner and more \textbf{presentable} by putting on dark outfits. Otherwise, I would look quite chubby in white or any other bright colors.}
    \end{qa}

    \begin{qa}{What can you learn about a person from the colours they like?}{Well, there is a saying that “Don't judge a book by its cover” and I believe I cannot assess a person by their favorite color. In Vietnam, those \textbf{having a keen interest in} purple are said to be loyal and faithful but this sounds ridiculous to me. Particularly, I have witnessed lots of people \textbf{idolizing} purple but \textbf{cheating} on their partners instead.}
    \end{qa}

    \begin{qa}{Do any colours have special meaning in your culture?}{Well, the answer is yes. White is generally regarded the symbol of purity and innocence. Yellow, thanks to its being the color of gold, is considered the \textbf{embodiment} of luxury. Red, besides its meaning interpreted earlier, is deemed lucky. In English-speaking countries, when we say someone is \textbf{in the red}, it means that he or she is in debt but in Vietnam, red is often \textbf{associated} with luck instead.}
    \end{qa}

    \begin{VocabExplain}[Part 1]
        \begin{ExplainCard}{embody}{v}
        \EN{(1) to express or represent an idea, quality, or feeling in a clear, tangible way; (2) to be a perfect example of something.}
        \SY{exemplify; personify; epitomize}
        \VI{(1) \textit{biểu hiện/hiện thân} của một ý tưởng hay phẩm chất; (2) \textit{là ví dụ điển hình} cho điều gì.}
        \EX{The constitution embodies core democratic values.}
        \EX{She embodies resilience and grace under pressure.}
        \CO{embody values; embody principles; fully embody the spirit}
        \end{ExplainCard}

        \begin{ExplainCard}{shed}{v}
        \EN{(1) to let something fall or drop naturally (e.g., leaves, hair, skin); (2) to get rid of something unnecessary; (3) to cause or lose (blood, tears).}
        \SY{cast; discard; slough}
        \VI{(1) \textit{rụng/rơi ra}; (2) \textit{loại bỏ}; (3) \textit{đổ} (máu, nước mắt).}
        \EX{Autumn trees shed their leaves.}
        \EX{History records the blood shed for independence.}
        \CO{shed light on; shed blood; shed pounds}
        \end{ExplainCard}

        \begin{ExplainCard}{forefather}{n}
        \EN{an ancestor, especially one from many generations back.}
        \SY{ancestor; forebear; predecessor}
        \VI{\textit{tổ tiên}, thế hệ đi trước.}
        \EX{We honor the sacrifices of our forefathers.}
        \EX{These customs were handed down by our forefathers.}
        \CO{our forefathers; founding forefathers; forefathers’ legacy}
        \end{ExplainCard}

        \begin{ExplainCard}{secure}{v}
        \EN{(1) to obtain or achieve something, especially with effort; (2) to make safe or protect; (3) to fasten firmly.}
        \SY{(1) obtain; clinch \quad (2) safeguard; protect \quad (3) fasten}
        \VI{(1) \textit{giành được/đạt được}; (2) \textit{bảo vệ/đảm bảo an toàn}; (3) \textit{cố định}.}
        \EX{She secured funding for the project.}
        \EX{Please secure the door before you leave.}
        \CO{secure a job; secure victory; secure funding; secure the perimeter}
        \end{ExplainCard}

        \begin{ExplainCard}{all-time}{adj}
        \EN{used to describe the best, worst, etc., at any time in history.}
        \SY{record; unsurpassed; historic}
        \VI{\textit{mọi thời đại}, đạt mức kỷ lục (tốt/xấu) nhất từ trước tới nay.}
        \EX{That’s my all-time favorite song.}
        \EX{Inflation hit an all-time high last year.}
        \CO{all-time high; all-time low; all-time favorite/best}
        \end{ExplainCard}

        \begin{ExplainCard}{predilection}{n}
        \EN{a strong liking or preference for something.}
        \SY{preference; fondness; penchant}
        \VI{\textit{thiên hướng/ưa thích} điều gì.}
        \EX{She has a predilection for dark colors.}
        \EX{His predilection for data-driven decisions shapes the team’s culture.}
        \CO{predilection for {music/science/dark colors}}
        \end{ExplainCard}

        \begin{ExplainCard}{humongous}{adj}
        \EN{informal: extremely large.}
        \SY{enormous; gigantic; massive}
        \VI{\textit{khổng lồ}, rất to.}
        \EX{They built a humongous stadium outside the city.}
        \EX{The dataset is humongous—over a billion rows.}
        \CO{humongous amount; humongous difference; humongous file}
        \end{ExplainCard}

        \begin{ExplainCard}{presentable}{adj}
        \EN{neat, clean, and suitable to be seen by others.}
        \SY{tidy; well-groomed; decent}
        \VI{\textit{chỉnh tề, ưa nhìn}, đủ lịch sự để gặp người khác.}
        \EX{Let me get changed into something more presentable.}
        \EX{Candidates should look presentable at the interview.}
        \CO{look presentable; smart and presentable; presentable outfit}
        \end{ExplainCard}

        \begin{ExplainCard}{Don't judge a book by its cover}{proverb}
        \EN{do not form an opinion of someone or something by appearance alone.}
        \SY{appearances can be deceiving; looks can mislead}
        \VI{\textit{đừng trông mặt mà bắt hình dong}: đừng đánh giá chỉ qua vẻ bề ngoài.}
        \EX{He seems strict, but don’t judge a book by its cover—he’s very kind.}
        \EX{The cafe looks plain, yet the food is excellent—don’t judge a book by its cover.}
        \CO{proverb; caution against snap judgments}
        \end{ExplainCard}

        \begin{ExplainCard}{have a keen interest in}{idiom}
        \EN{to be very interested in something.}
        \SY{be fascinated by; be passionate about; have a strong interest in}
        \VI{\textit{rất hứng thú/thích mê} điều gì.}
        \EX{She has a keen interest in astrophysics.}
        \EX{Investors have a keen interest in green technologies.}
        \CO{keen interest in {AI/history}; show keen interest}
        \end{ExplainCard}

        \begin{ExplainCard}{idolize}{v}
        \EN{to admire or love someone excessively; to treat as an idol.}
        \SY{adore; worship; lionize}
        \VI{\textit{thần tượng hoá}, ngưỡng mộ quá mức.}
        \EX{Teenagers often idolize pop stars.}
        \EX{He was idolized as a national hero.}
        \CO{idolize celebrities; be idolized by fans}
        \end{ExplainCard}

        \begin{ExplainCard}{cheat on}{phr.\ v}
        \EN{to be sexually unfaithful to a spouse or regular partner.}
        \SY{be unfaithful to; two-time; betray}
        \VI{\textit{ngoại tình/phản bội} người yêu hoặc vợ/chồng.}
        \EX{She broke up with him after he cheated on her.}
        \EX{He regrets cheating on his partner.}
        \CO{cheat on {sb}; caught cheating on}
        \end{ExplainCard}

        \begin{ExplainCard}{embodiment}{n}
        \EN{someone or something that is a perfect example or representation of a quality or idea.}
        \SY{epitome; personification; incarnation}
        \VI{\textit{sự hiện thân/hiện hữu} của một phẩm chất hoặc ý niệm.}
        \EX{She is the embodiment of patience.}
        \EX{The building stands as the embodiment of modern minimalism.}
        \CO{the embodiment of {courage/beauty}; pure embodiment}
        \end{ExplainCard}
    \end{VocabExplain}


    \noindent
    \textbf{Part 2.}
    \begin{qa}{Describe a song or a piece of music you like. You should say:}
        \begin{itemize}
        \item What the song or music is
        \item What kind of song or music it is
        \item Where you first heard it
        \item and explain why you like it.
        \end{itemize}

        {I \textbf{feel affection for} music, so my life would be terrible without it. If I have the chance to talk about my favorite song, the one that springs to my mind is "Never give up", which is composed and sung by Sia. Sia has made it to the top as a prominent figure in music industry, and I have been a huge fan of her for 4 years. Speaking of the song, it is the \textbf{soundtrack} of the film titled "Lion" and it was nominated for a Grammy award for the best song written for Visual Media. It is one of her \textbf{massive hits}, and many moviegoers believed that it contributed greatly to the success of the film. The song is about a journey of a five-year-old son trying to find his way back home after he escaped from a trafficking troop. I first heard this song when I attended her concert 3 years ago. At the time, she was going on an international tour, and luckily, I got a ticket from my colleague. She is also \textbf{fanatical about} Sia's music, and she booked 2 tickets beforehand. This song was added to the list as a tribute to her father who was struggling to overcome cancer. I was \textbf{immediately hooked on} its catchy melody and meaningful lyrics. The reason why I love this song is that it gives more energy to cope with daily stresses at the workplace. Whenever I am under pressure or I cannot achieve a balance between work and rest, I listen to this song to \textbf{blow off steam}. More importantly, it is her voice that strikes me. Her vocal style is addictive with screams at perfect pitches, so it soon \textbf{topped the music chart} when it was released.}
    \end{qa}

        \begin{VocabExplain}[Part 2]
            \begin{ExplainCard}{feel affection for}{phrase}
            \EN{to have warm feelings of liking and care toward someone or something.}
            \SY{be fond of; care for; hold dear}
            \VI{\textit{yêu thích}, có tình cảm trân quý với ai/điều gì.}
            \EX{I’ve always felt affection for acoustic music.}
            \EX{Many city dwellers still feel affection for their rural hometowns.}
            \CO{feel deep affection for; show affection for; strong affection for}
            \end{ExplainCard}

            \begin{ExplainCard}{soundtrack}{n}
            \EN{the recorded music or sounds used in a film, series, or game; also, the album containing that music.}
            \SY{film score; background score; original soundtrack}
            \VI{\textit{nhạc phim}/âm thanh của phim; cũng chỉ album nhạc phim.}
            \EX{The movie’s soundtrack blends traditional instruments with modern beats.}
            \EX{She bought the soundtrack after hearing the theme song in the credits.}
            \CO{original soundtrack (OST); film soundtrack; soundtrack album}
            \end{ExplainCard}

            \begin{ExplainCard}{massive hits}{phrase}
            \EN{songs that achieve very great popularity and commercial success.}
            \SY{chart-toppers; smash hits; runaway hits}
            \VI{\textit{những bản hit lớn}, cực kỳ phổ biến và thành công.}
            \EX{The band scored three massive hits last summer.}
            \EX{Streaming platforms helped turn the single into a massive hit.}
            \CO{score a massive hit; a string of massive hits; chart-topping hits}
            \end{ExplainCard}

            \begin{ExplainCard}{be fanatical about}{phrase}
            \EN{to be extremely enthusiastic about something, often to an excessive degree.}
            \SY{be obsessed with; be crazy about; be passionate about}
            \VI{\textit{phát cuồng}/rất mê mẩn về điều gì.}
            \EX{He’s fanatical about vinyl records and rare pressings.}
            \EX{Some fans are fanatical about following every stop on the tour.}
            \CO{fanatical about {fitness/football/music}; almost fanatical dedication}
            \end{ExplainCard}

            \begin{ExplainCard}{be hooked on}{phr.v}
            \EN{to be very interested in or unable to stop doing/using something.}
            \SY{be addicted to; be captivated by; can’t get enough of}
            \VI{\textit{nghiện}/mê mẩn một thứ gì đó.}
            \EX{I got hooked on the song after one listen.}
            \EX{She’s hooked on podcasts during her commute.}
            \CO{hooked on {a show/games/social media}; become hooked on}
            \end{ExplainCard}

            \begin{ExplainCard}{be under pressure}{phrase}
            \EN{to experience stress or strain, especially because of demands or deadlines.}
            \SY{feel stressed; be under strain; be pressed}
            \VI{\textit{bị áp lực}, căng thẳng do yêu cầu/công việc.}
            \EX{Students are under pressure during exam season.}
            \EX{He performs well even when he is under intense pressure.}
            \CO{under intense pressure; work under pressure; be under time pressure}
            \end{ExplainCard}

            \begin{ExplainCard}{blow off steam}{idiom}
            \EN{to release strong feelings or stress through activity or expression.}
            \SY{let off steam; unwind; decompress}
            \VI{\textit{xả hơi}/giải toả căng thẳng qua hoạt động hay tâm sự.}
            \EX{After work I jog to blow off steam.}
            \EX{She called a friend to blow off steam after the meeting.}
            \CO{blow off steam by {running/gaming/talking}; a way to blow off steam}
            \end{ExplainCard}

            \begin{ExplainCard}{top the music chart}{phrase}
            \EN{to reach number one on a music ranking for a given period.}
            \SY{top the charts; hit number one; claim the top spot}
            \VI{\textit{dẫn đầu bảng xếp hạng âm nhạc}.}
            \EX{The single topped the music chart within a week of release.}
            \EX{Only a few local artists have topped the charts internationally.}
            \CO{top the music charts; chart-topping single; climb to the top spot}
            \end{ExplainCard}
        \end{VocabExplain}

    \noindent
    \textbf{Part 3.}
    \begin{qa}{What kinds of music are popular with young people in your culture?}{It is true that there are numerous music \textbf{genres} in the music industry, and preference diversity has made people into different groups. But, basically, I guess Pop is the \textbf{mainstream} genre among teenagers in my country thanks to its \textbf{catchy rhythm} and \textbf{insightful} lyrics. Hip-hop and Rap are also on top of trends as well.}
    \end{qa}

    \begin{qa}{What do you think influences a young person's taste in music?}{There are, of course, \textbf{critical} factors that shape musical tastes of every single person. The first thing that \textbf{springs to my mind} is gender. Boys might go for \textbf{heavy metal} as an indicator of their \textbf{masculinity} while girls express strong preferences for pop music as it is \textbf{upbeat} and lively. A person's educational background, especially language competence, might have a profound impact on a youngster's taste as well. For instance, if one is good at English, then US-UK songs may be played often. On the contrary, if one's English level is low, US-UK songs might sound Greek to him and a song written in his mother tongue is likely to be preferred.}
    \end{qa}

    \begin{qa}{How has technology affected the kinds of music popular with young people?}{Technological advancements like the Internet or smart phones have made songs \textbf{accessible} to young people in recent years. It is easier to \textbf{tune in} and \textbf{recite} songs whenever people are \textbf{on the move} as a way of recreation. Gradually, young people, \textbf{are wont to} listen to music made by computer software instead of live music played by \textbf{musical instruments}. That accounts for the EDM's meteoric rise in worldwide popularity in recent years.}
    \end{qa}

    \begin{qa}{Tell me about any traditional music in your culture.}{Well, I did not live in the time when traditional music was \textbf{prevailing}. But from what I have heard from my grandparents, the \textbf{elderly} show little interest in \textbf{contemporary} music like rock or jazz. However, they are \textbf{avid listeners} of traditional or \textbf{folk music} which are two of popular music genres. A typical feature of these music genres is its slow tempo. Moreover, traditional genres reflect a sense of belonging among their listeners as they reflect humans' lives in the past.}
    \end{qa}

    \begin{qa}{How important is it for a culture to have musical traditions?}{Definitely, musical heritage is vital to \textbf{cultural identity}. It will \textbf{set one country apart from others} and enrich \textbf{intangible} values of that country. Traditional tunes are a \textbf{reflection} of emotional life in the old days which young people are \textbf{incapable} of experiencing.}
    \end{qa}

    \begin{qa}{Why do you think countries have national anthems or songs?}{I believe all nations must have their own national anthems. In general, the anthem is \textbf{in line with} a formal announcement of independence and \textbf{integration}. The song is a \textbf{patriotic} \textbf{musical composition} which expresses appreciation towards ancestors who devote their life to national peace. Besides, it \textbf{gives me a big thrill} to sing the solemn melody of the anthem and chant with a sea of people with pride.}
    \end{qa}

        \begin{VocabExplain}[Part 3]
            \begin{ExplainCard}{genres}{n}
            \EN{categories of art—here, music—characterized by a particular style, form, or content.}
            \SY{categories; styles; types}
            \VI{\textit{thể loại} (âm nhạc, nghệ thuật) có phong cách/nội dung riêng.}
            \EX{Streaming apps recommend new genres based on your history.}
            \EX{She studies how music genres evolve over time.}
            \CO{music genres; genre-blending; cross-genre}
            \end{ExplainCard}

            \begin{ExplainCard}{mainstream}{adj/n}
            \EN{(1) \textbf{adj}: accepted by or typical of the majority. (2) \textbf{n}: the dominant trend or current of opinion/activity.}
            \SY{(1) popular; conventional \quad (2) the majority; the norm}
            \VI{(1) \textit{chính thống/phổ biến}. (2) \textit{dòng chính}, xu hướng chủ đạo.}
            \EX{(1) K-pop is now mainstream in many countries.}
            \EX{(2) Indie artists sometimes cross into the mainstream after a viral hit.}
            \CO{mainstream culture; enter the mainstream; mainstream audience}
            \end{ExplainCard}

            \begin{ExplainCard}{catchy rhythm}{phrase}
            \EN{a beat that is memorable and easy to hum or move to.}
            \SY{infectious beat; hooky groove; memorable tempo}
            \VI{\textit{nhịp điệu bắt tai}, dễ nhớ và dễ lắc lư theo.}
            \EX{The song’s catchy rhythm made it an instant favorite at parties.}
            \EX{Ads often use a catchy rhythm to stick in your head.}
            \CO{catchy rhythm; catchy hook; irresistibly catchy}
            \end{ExplainCard}

            \begin{ExplainCard}{insightful}{adj}
            \EN{showing a deep understanding of people or situations; revealing useful ideas.}
            \SY{perceptive; illuminating; thought-provoking}
            \VI{\textit{sâu sắc}, thể hiện sự thấu hiểu.}
            \EX{Her review is insightful without being harsh.}
            \EX{The documentary offers an insightful look at youth culture.}
            \CO{insightful analysis; insightful commentary; remarkably insightful}
            \end{ExplainCard}

            \begin{ExplainCard}{critical}{adj}
            \EN{(1) extremely important for the outcome. (2) expressing disapproval or judgment. (3) at a dangerous or decisive stage.}
            \SY{(1) crucial; vital \quad (2) censorious; fault-finding \quad (3) precarious; acute}
            \VI{(1) \textit{then chốt, sống còn}. (2) \textit{phê phán}. (3) \textit{nguy kịch/quan trọng}.}
            \EX{(1) Parental support is critical to children’s learning.}
            \EX{(2) The article was critical of the festival’s organization.}
            \EX{(3) The patient is in a critical condition after surgery.}
            \CO{critical role; be critical of; critical stage}
            \end{ExplainCard}

            \begin{ExplainCard}{spring to my mind}{idiom}
            \EN{to come into one’s thoughts suddenly.}
            \SY{come to mind; occur to me; pop into my head}
            \VI{\textit{chợt nảy ra trong đầu}.}
            \EX{When you say “innovation,” startups spring to my mind.}
            \EX{The first example that springs to my mind is BTS.}
            \CO{what springs to mind; immediately springs to mind}
            \end{ExplainCard}

            \begin{ExplainCard}{heavy metal}{n}
            \EN{a loud, aggressive style of rock music featuring amplified guitars and powerful rhythms.}
            \SY{metal; hard rock (related)}
            \VI{\textit{dòng nhạc metal} với tiếng guitar khuếch đại và nhịp mạnh.}
            \EX{He learned double-kick drumming from heavy metal tracks.}
            \EX{Heavy metal festivals attract dedicated international fans.}
            \CO{heavy metal band; metal subgenres; thrash metal}
            \end{ExplainCard}

            \begin{ExplainCard}{masculinity}{n}
            \EN{qualities or attributes traditionally associated with men.}
            \SY{manliness; virility; machismo (contextual)}
            \VI{\textit{tính nam tính}.}
            \EX{Some genres are stereotypically linked to masculinity.}
            \EX{Modern ads try to redefine healthy masculinity.}
            \CO{toxic masculinity; expressions of masculinity; traditional masculinity}
            \end{ExplainCard}

            \begin{ExplainCard}{upbeat}{adj}
            \EN{(1) cheerful and positive. (2) (music) having a lively, fast tempo.}
            \SY{(1) optimistic; buoyant \quad (2) lively; energetic}
            \VI{(1) \textit{lạc quan, vui vẻ}. (2) \textit{sôi động, nhịp nhanh}.}
            \EX{(1) The class ends on an upbeat note.}
            \EX{(2) The DJ kept the crowd moving with upbeat tracks.}
            \CO{upbeat mood; upbeat tempo; stay upbeat}
            \end{ExplainCard}

            \begin{ExplainCard}{accessible}{adj}
            \EN{(1) easy to reach, obtain, or use. (2) easy to understand or appreciate.}
            \SY{(1) available; reachable \quad (2) comprehensible; user-friendly}
            \VI{(1) \textit{dễ tiếp cận}. (2) \textit{dễ hiểu}.}
            \EX{(1) Online platforms make music accessible to teens everywhere.}
            \EX{(2) The lecture was accessible even to beginners.}
            \CO{accessible content; make accessible; accessibility}
            \end{ExplainCard}

            \begin{ExplainCard}{tune in}{phr.v}
            \EN{(1) to watch or listen to a broadcast. (2) to become attentive or sensitized to something.}
            \SY{(1) switch on; listen in \quad (2) attune; pay attention}
            \VI{(1) \textit{theo dõi/phát kênh}. (2) \textit{chú tâm/đồng điệu}.}
            \EX{(1) Millions tune in to the live show every Friday.}
            \EX{(2) Good teachers tune in to students’ needs.}
            \CO{tune in to {a station/podcast}; tune in live}
            \end{ExplainCard}

            \begin{ExplainCard}{recite}{v}
            \EN{(1) to repeat aloud from memory. (2) to list or enumerate details.}
            \SY{(1) declaim; quote \quad (2) enumerate; reel off}
            \VI{(1) \textit{đọc/thuật lại} thuộc lòng. (2) \textit{kể/đọc vanh vách} chi tiết.}
            \EX{(1) She recited the poem flawlessly.}
            \EX{(2) He recited the safety rules before the trip.}
            \CO{recite a poem; recite lines; recite from memory}
            \end{ExplainCard}

            \begin{ExplainCard}{on the move}{phrase}
            \EN{(1) traveling from place to place. (2) very busy or active.}
            \SY{(1) on the go; in transit \quad (2) hectic; bustling}
            \VI{(1) \textit{đang di chuyển}. (2) \textit{bận rộn, hoạt động liên tục}.}
            \EX{(1) I listen to podcasts on the move.}
            \EX{(2) Startups are constantly on the move to meet deadlines.}
            \CO{always on the move; work on the move}
            \end{ExplainCard}

            \begin{ExplainCard}{wont to (be)}{adj/phrase}
            \EN{having a habit of doing something; accustomed.}
            \SY{accustomed to; prone to; inclined to}
            \VI{\textit{có thói quen/thường hay} làm gì.}
            \EX{Young people are wont to stream music rather than buy CDs.}
            \EX{As he was wont to do, he arrived early.}
            \CO{be wont to {do sth}; as is his wont}
            \end{ExplainCard}

            \begin{ExplainCard}{musical instruments}{n}
            \EN{devices created or adapted to produce musical sounds.}
            \SY{instruments; sound-making devices}
            \VI{\textit{nhạc cụ}, dụng cụ tạo âm nhạc.}
            \EX{Learning musical instruments improves aural skills.}
            \EX{She collects traditional musical instruments from across Asia.}
            \CO{play a musical instrument; traditional instruments; instrument mastery}
            \end{ExplainCard}

            \begin{ExplainCard}{prevailing}{adj}
            \EN{(1) existing or widespread at a particular time. (2) having superior power or influence.}
            \SY{(1) prevalent; widespread \quad (2) dominant; prevailing over}
            \VI{(1) \textit{thịnh hành, phổ biến}. (2) \textit{chiếm ưu thế, thắng thế}.}
            \EX{(1) The prevailing style then was folk ballads.}
            \EX{(2) The prevailing view eventually shaped policy.}
            \CO{prevailing trend; prevailing wind; prevailing view}
            \end{ExplainCard}

            \begin{ExplainCard}{elderly}{adj/n}
            \EN{(1) \textbf{adj}: older, typically past middle age. (2) \textbf{n}: the elderly—older people as a group.}
            \SY{senior; aged; older adults}
            \VI{(1) \textit{cao tuổi}. (2) \textit{người cao tuổi}.}
            \EX{(1) Many elderly listeners prefer live bands.}
            \EX{(2) The elderly often value nostalgic tunes.}
            \CO{elderly people; care for the elderly; elderly population}
            \end{ExplainCard}

            \begin{ExplainCard}{contemporary}{adj}
            \EN{(1) modern or of the present time. (2) belonging to the same period as something else.}
            \SY{(1) modern; current \quad (2) coeval; coexisting}
            \VI{(1) \textit{đương đại, hiện đại}. (2) \textit{cùng thời}.}
            \EX{(1) She performs contemporary jazz.}
            \EX{(2) Mozart was not a contemporary of Chopin.}
            \CO{contemporary music/art; contemporary with}
            \end{ExplainCard}

            \begin{ExplainCard}{avid listeners}{phrase}
            \EN{people who are very enthusiastic and keen about listening, often regularly.}
            \SY{devoted listeners; ardent fans; enthusiastic audience}
            \VI{\textit{người nghe cuồng nhiệt/đam mê}.}
            \EX{They’re avid listeners of late-night radio.}
            \EX{Podcasts have turned commuters into avid listeners.}
            \CO{avid listeners of {folk/jazz}; avid fanbase}
            \end{ExplainCard}

            \begin{ExplainCard}{folk music}{n}
            \EN{traditional music passed down within a community, often reflecting local life and history.}
            \SY{traditional music; roots music}
            \VI{\textit{nhạc dân gian}, phản ánh đời sống và lịch sử cộng đồng.}
            \EX{Folk music often features simple melodies and storytelling.}
            \EX{The festival celebrates regional folk music traditions.}
            \CO{traditional folk music; folk instruments; folk ballads}
            \end{ExplainCard}

            \begin{ExplainCard}{cultural identity}{phrase}
            \EN{the sense of belonging to a cultural group, shaped by shared values, traditions, and symbols.}
            \SY{heritage; collective identity; cultural belonging}
            \VI{\textit{bản sắc văn hoá}, cảm thức thuộc về một cộng đồng văn hoá.}
            \EX{Music is central to the cultural identity of many minorities.}
            \EX{Preserving festivals helps maintain cultural identity.}
            \CO{strong cultural identity; preserve cultural identity}
            \end{ExplainCard}

            \begin{ExplainCard}{set one country apart from others}{phrase}
            \EN{to make a nation distinctive compared with others.}
            \SY{distinguish; differentiate; make stand out}
            \VI{\textit{làm một quốc gia khác biệt} so với các nước khác.}
            \EX{Unique instruments set this country apart from others.}
            \EX{Language policies can set one country apart in education rankings.}
            \CO{set {sb/sth} apart; set a country apart}
            \end{ExplainCard}

            \begin{ExplainCard}{intangible}{adj}
            \EN{(1) not physical; unable to be touched. (2) hard to quantify but real in effect.}
            \SY{(1) immaterial \quad (2) non-quantifiable; subtle}
            \VI{(1) \textit{phi vật thể/không sờ nắm được}. (2) \textit{khó đo lường nhưng có giá trị}.}
            \EX{(1) Songs carry intangible heritage across generations.}
            \EX{(2) Brand trust is an intangible asset for musicians.}
            \CO{intangible cultural heritage; intangible value; intangible asset}
            \end{ExplainCard}

            \begin{ExplainCard}{reflection}{n}
            \EN{(1) a sign or expression that shows the nature of something. (2) serious thought or consideration.}
            \SY{(1) expression; manifestation \quad (2) contemplation; deliberation}
            \VI{(1) \textit{sự phản ánh/biểu hiện}. (2) \textit{sự suy ngẫm}.}
            \EX{(1) Traditional songs are a reflection of rural life.}
            \EX{(2) After some reflection, she changed the arrangement.}
            \CO{a reflection of; on reflection; deep reflection}
            \end{ExplainCard}

            \begin{ExplainCard}{incapable (of)}{adj}
            \EN{not able to do something or to achieve a particular result.}
            \SY{unable; powerless; unfit}
            \VI{\textit{không có khả năng} (làm điều gì).}
            \EX{Some apps are incapable of high-quality recording.}
            \EX{Without training, many are incapable of reading scores.}
            \CO{incapable of {doing/handling}; prove incapable}
            \end{ExplainCard}

            \begin{ExplainCard}{in line with}{phrase}
            \EN{consistent with; in accordance with.}
            \SY{consistent with; aligned with; according to}
            \VI{\textit{phù hợp/nhất quán với}.}
            \EX{The anthem’s lyrics are in line with national values.}
            \EX{Budgets were revised in line with new forecasts.}
            \CO{in line with policy; in line with expectations}
            \end{ExplainCard}

            \begin{ExplainCard}{integration}{n}
            \EN{(1) the process of combining parts into a whole. (2) the inclusion of individuals into a social or cultural group.}
            \SY{(1) consolidation; incorporation \quad (2) inclusion; assimilation}
            \VI{(1) \textit{sự tích hợp/kết hợp}. (2) \textit{hoà nhập}.}
            \EX{(1) Tech integration transformed music distribution.}
            \EX{(2) Community choirs aid the integration of migrants.}
            \CO{system integration; social integration; integrate into}
            \end{ExplainCard}

            \begin{ExplainCard}{patriotic}{adj}
            \EN{showing love, support, or pride for one’s country.}
            \SY{nationalistic (neutral); devoted; loyal}
            \VI{\textit{yêu nước}, thể hiện niềm tự hào quốc gia.}
            \EX{The patriotic anthem is sung at every ceremony.}
            \EX{Patriotic lyrics often feature historical imagery.}
            \CO{patriotic song; patriotic spirit; patriotic duty}
            \end{ExplainCard}

            \begin{ExplainCard}{musical composition}{n}
            \EN{an original piece of music created by a composer.}
            \SY{piece; work; score}
            \VI{\textit{tác phẩm âm nhạc}, bản nhạc do nhạc sĩ sáng tác.}
            \EX{Her musical composition blends folk and electronic textures.}
            \EX{The competition awards the best musical composition each year.}
            \CO{compose a piece; original composition; orchestral composition}
            \end{ExplainCard}

            \begin{ExplainCard}{give (me) a big thrill}{phrase}
            \EN{to cause a surge of excitement or delight.}
            \SY{excite greatly; exhilarate; electrify}
            \VI{\textit{khiến (tôi) phấn khích/tự hào tột độ}.}
            \EX{It gives me a big thrill to sing the anthem in a stadium.}
            \EX{Seeing the crowd sing along gave the band a big thrill.}
            \CO{a big thrill; thrilling moment; give {sb} a thrill}
            \end{ExplainCard}
        \end{VocabExplain}

    \begin{VocabHighlights}
        \VH{to embody}{(v) to express or represent an idea or a quality}{(động từ) tượng trưng cho}
        \VH{to shed}{(v) to let something fall; to drop something}{(động từ) đổ ra}
        \VH{forefather}{(n) ancestor}{(danh từ) tổ tiên}
        \VH{to secure}{(v) to obtain or achieve something, especially when this means using a lot of effort}{(động từ) giành được cái gì}
        \VH{all-time}{(adj) of any time}{(tính từ) mọi lúc}
        \VH{predilection}{(n) liking, preference}{(danh từ) thiên hướng thích gì}
        \VH{humongous}{(adj) very big}{(tính từ) rất to béo}
        \VH{presentable}{(adj) looking good enough for people to see}{(tính từ) trông ưa nhìn}
        \VH{don't judge a book by its cover}{(proverb) one shouldn’t prejudge the worth or value of something by its outward appearance alone}{(tục ngữ) đừng trông mặt mà bắt hình dong}
        \VH{to have a keen interest in}{(idiom) to be very interested in something}{(thành ngữ) thích mê mệt}
        \VH{to idolize}{(v) to admire or love somebody very much}{(động từ) thần tượng hóa}
        \VH{to cheat on}{(phr. v) (of somebody who is married or who has a regular sexual partner) to have a secret sexual relationship with somebody else}{(cụm động từ) lừa tình ai}
        \VH{embodiment}{(n) a person or thing that represents or is a typical example of an idea or a quality}{(danh từ) là hiện thân của}
        \VH{to feel affection for}{(phrase) to like something}{(cụm từ) yêu thích}
        \VH{soundtrack}{(n) the sounds, especially the music of a film, or a separate recording of this}{(danh từ) nhạc phim}
        \VH{massive hits}{(phrase) big hits}{(cụm từ) các bài hát làm nên tên tuổi nghệ sĩ}
        \VH{to be fanatical about}{(phrase) to be extremely interested in}{(cụm từ) thích thú, phát cuồng về}
        \VH{to be hooked on}{(phr.v) to be addicted to}{(cụm động từ) nghiện}
        \VH{to be under pressure}{(phrase) difficult to deal with stress}{(cụm từ) bị áp lực}
        \VH{to blow off steam}{(idiom) to say something that helps you to get rid of strong feelings or energy}{(thành ngữ) xả hơi}
        \VH{to top the music chart}{(phrase) to hold the first position according to popularity during a given period of time}{(cụm từ) dẫn đầu bảng xếp hạng âm nhạc}
        \VH{genre}{(n) a particular type or style of art, film or music}{(danh từ) thể loại}
        \VH{mainstream}{(adj) considered normal because it reflects what is done or accepted by most people}{(tính từ) xu hướng chủ đạo, thịnh hành}
        \VH{catchy rhythm}{(phrase) pleasing and easily remembered with a strong regular repeated pattern of sounds}{(cụm từ) nhịp điệu dễ nghe, lôi cuốn}
        \VH{insightful}{(adj) showing a clear understanding of a person or situation}{(tính từ) sâu sắc, nhiều ý nghĩa}
        \VH{critical}{(adj) extremely important because a future situation will be affected by it}{(tính từ) quan trọng}
        \VH{to spring to one's mind}{(phrase) to come quickly into one's mind}{(cụm từ) điều gì nhảy ra trong đầu}
        \VH{heavy metal}{(n) a genre of rock music}{(danh từ) dòng nhạc heavy metal thuộc thể loại nhạc rock}
        \VH{masculinity}{(n) the quality of being masculine}{(danh từ) sự nam tính}
        \VH{upbeat}{(adj) positive and enthusiastic; making you feel that the future will be good}{(tính từ) phấn khởi, lạc quan}
        \VH{accessible}{(adj) that can be reached, entered, used, seen, etc}{(tính từ) dễ tiếp cận}
        \VH{tune in}{(phrase) to listen to a radio programme or watch a television programme}{(cụm từ) nghe chương trình radio; xem chương trình truyền hình}
        \VH{recite}{(v) to say a list or series of things}{(động từ) lẩm bẩm hát theo}
        \VH{on the move}{(phrase) in the process of moving from one place or job to another}{(cụm từ) trên đường đi, di chuyển}
        \VH{be wont to}{(adj) to be likely to do something}{(tính từ) dễ làm gì}
        \VH{prevailing}{(adj) existing or most common at a particular time}{(tính từ) thịnh hành}
        \VH{the elderly}{(n) used as a polite word for ‘old’}{(danh từ) người cao tuổi}
        \VH{contemporary}{(adj) belonging to the same time}{(tính từ) đương thời, đương đại}
        \VH{an avid listener of}{(phrase) a person who likes listening to something very much}{(cụm từ) người đam mê; thích nghe gì}
        \VH{cultural identity}{(n) the identity or feeling of belonging to a group}{(danh từ) bản sắc văn hóa}
        \VH{set one country apart from others}{(phrase) to distinguish a country from other countries}{(cụm từ) làm quốc gia này khác biệt với quốc gia khác}
        \VH{intangible}{(adj) that exists but it is difficult to describe, understand or measure}{(tính từ) vô hình}
        \VH{reflection}{(n) a sign that shows the state or nature of something}{(danh từ) sự phản ánh}
        \VH{to be incapable of}{(adj) not able to do something}{(tính từ) không có khả năng làm được}
        \VH{integration}{(n) the act or process of combining two or more things so that they work together}{(danh từ) sự hợp nhất; thống nhất}
        \VH{patriotic}{(adj) having or expressing a great love of your country}{(tính từ) tính yêu nước}
        \VH{composition}{(n) a piece of music or art, or a poem}{(danh từ) sự sáng tác}
        \VH{to give somebody a big thrill}{(phrase) to thrill somebody}{(cụm từ) khiến ai rùng mình}
    \end{VocabHighlights}

    \end{test}

    \begin{test}{TEST 3}
    \noindent
    \textbf{Part 1. Entertainment}
    \begin{qa}{Do you prefer relaxing at home or going out in the evening? (Why?)}{It depends on each day of the week. On weekdays, I have a \textbf{preference} for staying at home to \textbf{clear my mind} after a hard-working day. At weekends, I tend to \textbf{gravitate towards} city attractions such as department stores, cinemas and museums \textbf{in the company of} my family. Going to such places of interests requires a huge amount of time so it is better to spend on weekends as these are my days off work.}
    \end{qa}

    \begin{qa}{When you go out for an evening, what do you like to do?}{Well, as I said earlier, when I enjoy an evening out, I have \textbf{a penchant for} arriving at department stores to do the shopping or going to the cinemas to enjoy films, especially blockbuster ones, there.}
    \end{qa}

    \begin{qa}{How popular is this with other people in your country?}{I am not sure about this but come to think of it, I think my preference is quite similar to others'. \textbf{From time to time}, I enter department stores or cinemas on weekdays instead of weekends. It is apparent that the weekend goers \textbf{outnumber} the weekday visitors.}
    \end{qa}

    \begin{qa}{Is there any kind of entertainment you do not like? (Why/Why not?)}{Yes, there is. Many people in Vietnam engage in \textbf{gambling}, which they believe is a great way of entertainment to \textbf{kill time} and test their luck. However, \textbf{Lady Luck} is hardly on their sides so \textbf{in all likelihood} they will \textbf{go bankrupt} soon and be a burden on their family instead.}
    \end{qa}

        \begin{VocabExplain}[Part 1]
            \begin{ExplainCard}{preference}{n}
            \EN{a greater liking for one alternative over another; what you like more.}
            \SY{liking; inclination; predisposition}
            \VI{\textit{sở thích/ưu tiên}, sự ưa chuộng hơn giữa các lựa chọn.}
            \EX{I have a strong preference for staying in on weeknights.}
            \EX{Please indicate your course preferences on the form.}
            \CO{have a preference for; show a preference; personal preference}
            \end{ExplainCard}

            \begin{ExplainCard}{clear my mind}{phrase}
            \EN{to reduce worry or mental clutter so you can think calmly.}
            \SY{unwind; reset; declutter one’s thoughts}
            \VI{\textit{thư giãn đầu óc}, gỡ bỏ suy nghĩ rối bời để tỉnh táo hơn.}
            \EX{A short walk helps me clear my mind after work.}
            \EX{I meditate each morning to clear my mind before studying.}
            \CO{clear one’s mind; clear your head; mental reset}
            \end{ExplainCard}

            \begin{ExplainCard}{gravitate towards}{phr.v}
            \EN{(1) to be naturally attracted or drawn to something; (2) to move physically toward something/someone.}
            \SY{(1) be drawn to; lean toward \quad (2) drift toward; move toward}
            \VI{(1) \textit{thiên về/bị hút về} điều gì; (2) \textit{di chuyển} về phía.}
            \EX{(1) Young audiences often gravitate towards upbeat pop.}
            \EX{(2) People gravitated towards the stage as the show began.}
            \CO{gravitate towards {cities/careers/trends}}
            \end{ExplainCard}

            \begin{ExplainCard}{in the company of}{phrase}
            \EN{together with; while being with a particular person/people.}
            \SY{together with; alongside; in the presence of}
            \VI{\textit{trong bầu bạn với}, đi cùng ai đó.}
            \EX{I prefer museums in the company of my family.}
            \EX{He feels more confident in the company of close friends.}
            \CO{in the company of friends/family/experts}
            \end{ExplainCard}

            \begin{ExplainCard}{a penchant for}{phrase}
            \EN{a strong or habitual liking for something; a tendency to do something.}
            \SY{fondness for; preference for; proclivity for}
            \VI{\textit{niềm yêu thích/thói quen} mạnh đối với điều gì.}
            \EX{She has a penchant for late-night movies.}
            \EX{His penchant for snacks is legendary at the office.}
            \CO{a penchant for {jazz/chocolate/minimalism}}
            \end{ExplainCard}

            \begin{ExplainCard}{from time to time}{phrase}
            \EN{occasionally; now and then.}
            \SY{occasionally; at times; every so often}
            \VI{\textit{thỉnh thoảng}, đôi khi.}
            \EX{From time to time I catch a film on a weekday.}
            \EX{We meet from time to time to keep in touch.}
            \CO{from time to time; every now and then}
            \end{ExplainCard}

            \begin{ExplainCard}{outnumber}{v}
            \EN{to be greater in number than someone or something.}
            \SY{exceed in number; surpass; outmatch (in number)}
            \VI{\textit{đông hơn}, vượt trội về số lượng.}
            \EX{Weekend visitors outnumber weekday visitors by far.}
            \EX{At the festival, locals outnumbered tourists two to one.}
            \CO{outnumber by {two to one}; significantly outnumber}
            \end{ExplainCard}

            \begin{ExplainCard}{gambling}{n}
            \EN{(1) playing games of chance for money; (2) taking risky action with uncertain results (figurative).}
            \SY{(1) betting; wagering \quad (2) risk-taking; speculation}
            \VI{(1) \textit{cờ bạc}; (2) \textit{đánh cược mạo hiểm} (nghĩa bóng).}
            \EX{(1) Gambling can lead to serious debt.}
            \EX{(2) Quitting his job to tour with a band was a bit of gambling.}
            \CO{online gambling; problem gambling; gambling addiction}
            \end{ExplainCard}

            \begin{ExplainCard}{kill time}{idiom}
            \EN{to do something to make a period of waiting pass more quickly.}
            \SY{pass the time; while away the time; occupy oneself}
            \VI{\textit{giết thời gian}, làm gì đó chờ đợi cho nhanh qua.}
            \EX{We killed time at the mall before the movie.}
            \EX{I read short stories to kill time on the bus.}
            \CO{kill time by {reading/window-shopping}}
            \end{ExplainCard}

            \begin{ExplainCard}{Lady Luck}{idiom}
            \EN{personification of luck or fortune, especially in gambling or uncertain situations.}
            \SY{good fortune; fate; luck}
            \VI{\textit{Nữ thần may mắn}, vận may.}
            \EX{Lady Luck smiled on him and he won the raffle.}
            \EX{Without strategy, you’re relying on Lady Luck alone.}
            \CO{Lady Luck smiles/frowns; depend on Lady Luck}
            \end{ExplainCard}

            \begin{ExplainCard}{in all likelihood}{phrase}
            \EN{very probably; with a high chance of happening.}
            \SY{most likely; in all probability; very likely}
            \VI{\textit{rất có khả năng}, nhiều khả năng.}
            \EX{In all likelihood, the cinema will be packed on Friday.}
            \EX{She’ll, in all likelihood, choose a quiet night in.}
            \CO{in all likelihood + clause}
            \end{ExplainCard}

            \begin{ExplainCard}{go bankrupt}{phrase}
            \EN{to become insolvent; to be legally declared unable to pay debts.}
            \SY{become insolvent; go under; fail financially}
            \VI{\textit{phá sản}, mất khả năng thanh toán.}
            \EX{Many casinos went bankrupt during the downturn.}
            \EX{If spending isn’t controlled, small venues could go bankrupt.}
            \CO{declare bankruptcy; file for bankruptcy; near-bankrupt}
            \end{ExplainCard}
        \end{VocabExplain}

    \noindent
    \textbf{Part 2.}

    \begin{qa}{Describe one of your friends. You should say:}{
        \begin{itemize}
        \item How you met
        \item How long you have known each other
        \item How you spend time together
        \item and explain why you like this person.
        \end{itemize}

        \begin{qa}{Describe one of your friends.}{I guess I could start off by answering who the person is and the one I'd pick is Nhat Thuc. Speaking of him, he is now a designer, and he is the same age as me, but we do not study together. Let me tell you how I met him. We met each other thanks to a trip to Ninh Binh. 2 years ago, I was \textbf{stressed out} because I just switched my job. In order to \textbf{let it all hang out}, I posted a status in a group on Facebook to find a company to travel with me. At that time, he was dumped by his girlfriend, so he also wanted to pay a visit to somewhere to refresh his life. What struck me most is that we had a lot in common. We both had a passion for shopping and travelling, and we \textbf{got on like a house on fire}. Now we have a habit of meeting up on a weekly basis to shop for some items or just \textbf{tighten the bond} by \textbf{catching upon a gossip} in a café shop. The reason why I like him is that he always \textbf{gives me a listening ear}. Whenever I face some obstacles in my life, he \textbf{is always there for me}. He \textbf{sticks up for me through thick and thin}, so I believe that he is really my true friend. Another reason is that he has a sense of humor. I couldn't help \textbf{laughing my head off} when he tells jokes.}
    \end{qa}

        \begin{VocabExplain}[Part 2]
            \begin{ExplainCard}{stressed out}{adj}
            \EN{feeling very anxious or under a lot of pressure.}
            \SY{overwhelmed; frazzled; tense}
            \VI{\textit{căng thẳng tột độ}, quá tải áp lực.}
            \EX{I was totally stressed out after switching jobs.}
            \EX{She looks stressed out before every deadline.}
            \CO{feel/get stressed out; be stressed out about}
            \end{ExplainCard}

            \begin{ExplainCard}{let it all hang out}{idiom}
            \EN{(1) to relax and be yourself without worrying about what others think; (2) to express feelings openly/vent.}
            \SY{(1) unwind; be yourself \quad (2) open up; let off steam}
            \VI{(1) \textit{thả lỏng, sống đúng là mình}; (2) \textit{bộc lộ cảm xúc}, trút bầu tâm sự.}
            \EX{On holiday we just let it all hang out by the beach.}
            \EX{Over coffee she let it all hang out about her breakup.}
            \CO{just let it all hang out; time to let it all hang out}
            \end{ExplainCard}

            \begin{ExplainCard}{get on like a house on fire}{idiom}
            \EN{to become friends very quickly and get along extremely well.}
            \SY{hit it off; get along famously; click}
            \VI{\textit{hợp nhau ngay}, thân nhau rất nhanh.}
            \EX{We got on like a house on fire from the first trip.}
            \EX{The new teammates got on like a house on fire.}
            \CO{get on like a house on fire with \{sb\}}
            \end{ExplainCard}

            \begin{ExplainCard}{tighten the bond}{phrase}
            \EN{to make a relationship closer and stronger.}
            \SY{strengthen/cement the bond; deepen ties}
            \VI{\textit{thắt chặt mối quan hệ}.}
            \EX{Weekly meetups help us tighten the bond.}
            \EX{Volunteering together tightened the bond between classmates.}
            \CO{tighten the bond with \{friends/family\}; strengthen bonds}
            \end{ExplainCard}

            \begin{ExplainCard}{catch up on gossip}{phrase}
            \EN{to chat and hear the latest news/rumors among friends \textit{(correct collocation: catch up \textbf{on} (the) gossip)}.}
            \SY{chat; dish; swap stories}
            \VI{\textit{tám chuyện}, cập nhật chuyện mọi người.}
            \EX{We met at a café to catch up on gossip.}
            \EX{They love catching up on the office gossip at lunch.}
            \CO{catch up on the gossip; have a gossip; gossip session}
            \end{ExplainCard}

            \begin{ExplainCard}{give (someone) a listening ear}{phrase}
            \EN{to listen attentively and sympathetically to someone.}
            \SY{lend an ear; offer a sympathetic ear; be a good listener}
            \VI{\textit{lắng nghe cảm thông} ai đó.}
            \EX{He always gives me a listening ear when I’m low.}
            \EX{Teachers should give students a listening ear after exams.}
            \CO{give/offer a listening ear; a sympathetic ear}
            \end{ExplainCard}

            \begin{ExplainCard}{be there for (someone)}{phrase}
            \EN{to support or help someone when they need you.}
            \SY{support; stand by; have (sb’s) back}
            \VI{\textit{ở bên cạnh/ủng hộ} ai khi họ cần.}
            \EX{She was there for me during the hardest months.}
            \EX{Good friends are there for each other, no questions asked.}
            \CO{always be there for; be there for \{sb\}}
            \end{ExplainCard}

            \begin{ExplainCard}{stick up for (someone) — through thick and thin}{idiom}
            \EN{(1) \textbf{stick up for}: to defend or support someone; (2) \textbf{through thick and thin}: despite difficulties, at all times.}
            \SY{(1) defend; back up \quad (2) steadfastly; no matter what}
            \VI{(1) \textit{bênh vực/đứng về phía} ai; (2) \textit{dù thăng trầm}, luôn luôn.}
            \EX{He always sticks up for me through thick and thin.}
            \EX{True friends stick up for each other and stay loyal through thick and thin.}
            \CO{stick up for \{a friend\}; loyal through thick and thin}
            \end{ExplainCard}

            \begin{ExplainCard}{laugh one’s head off}{idiom}
            \EN{to laugh very hard and loudly.}
            \SY{crack up; howl with laughter; die laughing}
            \VI{\textit{cười muốn rụng rốn}, cười nghiêng ngả.}
            \EX{His jokes made us laugh our heads off.}
            \EX{We laughed our heads off at the meme he sent.}
            \CO{laugh \{my/your\} head off; make \{sb\} laugh their head off}
            \end{ExplainCard}
        \end{VocabExplain}

    \noindent
    \textbf{Part 3.}

        \begin{VocabExplain}[Part 3]
        \end{VocabExplain}

    \begin{VocabHighlights}
    \end{VocabHighlights}
    \end{test}

    \begin{test}{TEST 4}
    \noindent
    \textbf{Part 1.}

        \begin{VocabExplain}[Part 1]
        \end{VocabExplain}

    \noindent
    \textbf{Part 2.}

        \begin{VocabExplain}[Part 2]
        \end{VocabExplain}

    \noindent
    \textbf{Part 3.}

        \begin{VocabExplain}[Part 3]
        \end{VocabExplain}

    \begin{VocabHighlights}
    \end{VocabHighlights}
    \end{test}
\end{glossarymc}