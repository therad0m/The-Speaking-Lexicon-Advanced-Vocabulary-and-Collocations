\begin{glossarymc}[Cambridge 6]
    \begin{test}{TEST 1}
    \noindent
    \textbf{Part 1. Dancing}
    \begin{qa}{Do you enjoy dancing? (Why/Why not?)}
    The answer, I’m afraid, is an \textbf{emphatic} “No”. I find it hard to catch up with the music rhythms, \textbf{not to mention} rock my body to the beat. In fact, I am a hardcore metal fan, so I’m more of a \textbf{headbanger} than a dancer.
    \end{qa}

    \begin{qa}{Has anyone ever taught you to dance? (Why/Why not?)}
    Yes, of course. When I was in grade 6 or 7, a \textbf{choreographer} was assigned to teach my classmates and me the choreography for a musical. At that moment I realized I was not \textbf{cut out to} dance, as I kept disrupting the group’s natural rhythm. I think I actually \textbf{have two left feet}, so I haven’t tried dancing anymore.
    \end{qa}

    \begin{qa}{Tell me about any traditional dancing in your country.}
    Speaking of traditional dancing, the first thing that \textbf{springs to my mind} is bamboo dancing. This kind of dancing \textbf{makes a name for itself} because it encourages both locals and foreigners to join in. It originated in the Northwest highlands of Vietnam. In this dance, four to six people \textbf{grab hold of} two bamboo poles each and clap in a 4/4 rhythm while another eight dancers step in and out of the poles.
    \end{qa}

    \begin{qa}{Do you think that traditional dancing will be popular in the future? (Why/Why not?)}
    I’m not sure. As long as traditional dancing is easy and fun to play, it can still \textbf{grow in popularity}. In the case of bamboo dancing, everyone can easily \textbf{partake in} it, so I believe in its development in the future.
    \end{qa}

    
        \begin{VocabExplain}[Part 1]
            \begin{ExplainCard}{emphatic}{adj}
            \EN{expressed with force and clarity because the speaker feels strongly; leaving no doubt.}
            \SY{forceful; vehement; unequivocal}
            \VI{\textit{mang tính nhấn mạnh}, dứt khoát.}
            \EX{She gave an emphatic “No” when asked to perform.}
            \EX{The coach was emphatic that everyone arrive on time.}
            \CO{emphatic denial/refusal; emphatic victory; be emphatic that + clause}
            \end{ExplainCard}

            \begin{ExplainCard}{not to mention}{phrase}
            \EN{used to add an extra point that strengthens what has just been said.}
            \SY{let alone; besides; on top of that}
            \VI{\textit{chưa kể đến}, thêm vào đó.}
            \EX{The class is expensive, not to mention far from my house.}
            \EX{He’s talented—\textit{not to mention} incredibly hard-working.}
            \CO{not to mention + noun/V\textsubscript{ing}}
            \end{ExplainCard}

            \begin{ExplainCard}{headbanger}{n}
            \EN{a fan or performer of heavy-metal music, especially one who nods or shakes the head to the beat.}
            \SY{metalhead; rocker}
            \VI{\textit{fan nhạc heavy metal}.}
            \EX{I’m a headbanger rather than a ballroom dancer.}
            \EX{Headbangers packed the arena for the festival.}
            \CO{heavy-metal headbanger; die-hard headbanger}
            \end{ExplainCard}

            \begin{ExplainCard}{choreographer}{n}
            \EN{a person who creates and arranges dance movements and sequences for a performance.}
            \SY{dance maker; dance director}
            \VI{\textit{biên đạo múa/nhảy}.}
            \EX{A famous choreographer staged the opening routine.}
            \EX{She works as a choreographer for music videos.}
            \CO{celebrity choreographer; choreograph a routine/number}
            \end{ExplainCard}

            \begin{ExplainCard}{be cut out for / be cut out to be}{phr.\ v}
            \EN{to have (or not have) the right qualities for a particular task or role.}
            \SY{be suited for; be fit for; be made for}
            \VI{\textit{hợp/không hợp} với công việc/vai trò nào đó.}
            \EX{I’m not cut out for competitive dancing.}
            \EX{She’s cut out to be a teacher—patient and clear.}
            \CO{be (not) cut out for + noun/V\textsubscript{ing}; be cut out to be + noun}
            \end{ExplainCard}

            \begin{ExplainCard}{have two left feet}{idiom}
            \EN{to be very awkward or uncoordinated when dancing.}
            \SY{be a clumsy dancer; lack rhythm}
            \VI{\textit{nhảy vụng về}, không có khiếu nhảy.}
            \EX{I avoid the dance floor—I’ve got two left feet.}
            \EX{He jokes that he has two left feet at weddings.}
            \CO{(seem/feel) to have two left feet}
            \end{ExplainCard}

            \begin{ExplainCard}{spring to (one’s) mind}{phrase}
            \EN{to come immediately or suddenly into someone’s thoughts.}
            \SY{come to mind; occur to; pop into mind}
            \VI{\textit{bất chợt nảy ra trong đầu}.}
            \EX{When you say “traditional dance,” bamboo dancing springs to mind.}
            \EX{Her name sprang to my mind during the audition.}
            \CO{the first thing that springs to mind}
            \end{ExplainCard}

            \begin{ExplainCard}{make a name for oneself}{idiom}
            \EN{to become well known for one’s achievements.}
            \SY{gain recognition; build a reputation; rise to prominence}
            \VI{\textit{tạo dựng tên tuổi}.}
            \EX{She made a name for herself in contemporary dance.}
            \EX{The troupe is making a name for itself on TikTok.}
            \CO{make a name for oneself in + field}
            \end{ExplainCard}

            \begin{ExplainCard}{grab hold of}{phrase}
            \EN{to quickly seize and keep a firm grip on something.}
            \SY{seize; clutch; take hold of}
            \VI{\textit{nắm chặt}, chộp lấy.}
            \EX{Dancers grab hold of the bamboo poles in sync.}
            \EX{He grabbed hold of my arm to keep me from slipping.}
            \CO{grab/ take hold of + object; grab hold tight}
            \end{ExplainCard}

            \begin{ExplainCard}{grow in popularity}{phrase}
            \EN{to become increasingly popular over time.}
            \SY{gain traction; catch on; rise in popularity}
            \VI{\textit{ngày càng phổ biến}.}
            \EX{Folk-fusion routines are growing in popularity online.}
            \EX{The festival has grown in popularity since 2019.}
            \CO{steadily/rapidly grow in popularity}
            \end{ExplainCard}

            \begin{ExplainCard}{partake in}{v}
            \EN{to take part in an activity with others; to share in.}
            \SY{participate in; join in; engage in}
            \VI{\textit{tham gia vào}.}
            \EX{Visitors are welcome to partake in the circle dance.}
            \EX{Only trained members may partake in the finale.}
            \CO{partake in festivities/rituals/activities}
            \end{ExplainCard}
        \end{VocabExplain}

    \noindent
    \textbf{Part 2.}
    \begin{qa}{Describe someone in your family who you like. You should say:}
    How this person is related to you\\
    What this person looks like\\
    What kind of person he/she is\\
    and explain why you like this person.

    I am lucky to have two brothers, and the person I'm closest to in my family is definitely my older brother, Mr. Tam. He was \textbf{named after a legendary figure} in Vietnam's music industry because my parents wanted him to be famous when he grew up. Talking about his appearance, he is \textbf{well-built} with broad shoulders. The reason for his being gripped is that he \textbf{works out} in the gym regularly, so he \textbf{takes great pride in} his good looks. He is 4 years older than me, but he has a youthful appearance. Whenever we go out, many people mistake me for him because we \textbf{are like two peas in a pod}. He and I grew up together, so we always \textbf{get on well} and we hardly ever \textbf{fall out}. He is so creative. He always thinks of new games we could play and make up these stories to \textbf{crack me up}. I love going out with him because he's always \textbf{the life and soul of the party}. A night out with him is never boring! I am constantly amazed by his optimism as well. Well, he is actually a born optimist, so when I need to confide in, he is always there for me and encourage me to \textbf{look on the bright side}. The reason why I like him is that we \textbf{are made for each other}. We are both \textbf{fervent} supporters of music, so we started a band for music lovers 2 years ago. Another reason is that he usually supports me when I am \textbf{in a pickle}. For example, he can give me money to buy shoes \textbf{and the likes} if I am broke.
    \end{qa}

        \begin{VocabExplain}[Part 2]
            \begin{ExplainCard}{be named after (sb/sth)}{phr.v}
            \EN{to receive your name from another person or thing, usually to honor them.}
            \SY{be named for; take one’s name from}
            \VI{được \textit{đặt tên theo} người/vật khác, thường để vinh danh.}
            \EX{Their son was named after his grandfather.}
            \EX{The new stadium is named after a national hero.}
            \CO{be named after/for + person; take one’s name from}
            \end{ExplainCard}
            

            \begin{ExplainCard}{well-built}{adj}
            \EN{(1) having a strong, solid, and proportionate body.\\(2) constructed strongly and durably (of things).}
            \SY{(1) sturdy; muscular; robust \quad (2) solidly built; sturdy}
            \VI{(1) \textit{vạm vỡ}, săn chắc (người).\\(2) được xây dựng chắc chắn (vật).}
            \EX{(1) The goalkeeper is well-built with broad shoulders.}
            \EX{(2) The bridge is well-built and withstands heavy traffic.}
            \CO{well-built physique/frame; well-built house/structure}
            \end{ExplainCard}

            \begin{ExplainCard}{work out}{phr.v}
            \EN{(1) to exercise, especially at a gym.\\(2) to develop in a successful way or to solve a problem by calculation or reasoning.}
            \SY{(1) train; exercise \quad (2) turn out; resolve}
            \VI{(1) \textit{tập luyện}, tập gym.\\(2) diễn tiến tốt/giải quyết được.}
            \EX{(1) She works out three times a week.}
            \EX{(2) We finally worked out a fair plan for everyone.}
            \CO{work out at the gym; work out a solution/plan; it works out}
            \end{ExplainCard}

            \begin{ExplainCard}{take pride in (sth)}{phrase}
            \EN{(1) to feel deep satisfaction about something you or someone close to you has done.\\(2) to care about doing something to a high standard.}
            \SY{be proud of; take satisfaction in; pride oneself on}
            \VI{(1) \textit{tự hào} về điều gì.\\(2) \textit{đề cao chất lượng}, làm việc một cách chỉn chu.}
            \EX{(1) He takes pride in his good looks.}
            \EX{(2) She prides herself on delivering work on time.}
            \CO{take great/immense pride in; pride oneself on + V\textsubscript{ing}/noun}
            \end{ExplainCard}

            \begin{ExplainCard}{be like two peas in a pod}{idiom}
            \EN{to be extremely similar in appearance or character.}
            \SY{be identical; be very much alike}
            \VI{\textit{giống nhau như hai giọt nước}.}
            \EX{People say the twins are like two peas in a pod.}
            \EX{In tastes and habits, the brothers are like two peas in a pod.}
            \CO{be/look like two peas in a pod}
            \end{ExplainCard}

            \begin{ExplainCard}{get on (well) (with sb)}{phr.v}
            \EN{to have a good, friendly relationship with someone.}
            \SY{get along; hit it off}
            \VI{\textit{hoà hợp}, có quan hệ tốt với ai.}
            \EX{We grew up together, so we get on well.}
            \EX{She gets on really well with her new colleagues.}
            \CO{get on well/badly with; get along with; really/remarkably well}
            \end{ExplainCard}

            \begin{ExplainCard}{fall out (with sb)}{phr.v}
            \EN{(1) to argue and stop being friendly with someone.\\(2) (lit.) to drop from a place or container.}
            \SY{(1) quarrel; have a falling-out \quad (2) drop out}
            \VI{(1) \textit{xích mích}, cãi nhau rồi \textit{ngừng chơi}.\\(2) (nghĩa đen) rơi ra.}
            \EX{(1) They hardly ever fall out.}
            \EX{(2) Coins fell out of his pocket.}
            \CO{fall out over sth; have a falling-out; fall out with sb}
            \end{ExplainCard}

            \begin{ExplainCard}{crack (someone) up}{phr.v}
            \EN{(1) to make someone laugh a lot.\\(2) (crack up) to become unable to cope; have a breakdown.}
            \SY{(1) make sb roar; have sb in stitches \quad (2) break down}
            \VI{(1) \textit{làm ai cười lăn}.\\(2) suy sụp tinh thần.}
            \EX{(1) His stories always crack me up.}
            \EX{(2) Under pressure, he started to crack up.}
            \CO{crack sb up; have sb in stitches; completely crack up}
            \end{ExplainCard}

            \begin{ExplainCard}{the life and soul of the party}{idiom}
            \EN{a very lively, entertaining person who makes social events enjoyable.}
            \SY{live wire; crowd-pleaser; social spark}
            \VI{\textit{linh hồn của bữa tiệc}, người khuấy động không khí.}
            \EX{She's always the life and soul of the party.}
            \EX{With him around, no gathering feels dull.}
            \CO{be/act as the life and soul of the party}
            \end{ExplainCard}

            \begin{ExplainCard}{look on the bright side}{idiom}
            \EN{to focus on the positive aspects of a difficult situation.}
            \SY{stay positive; see the silver lining; keep optimistic}
            \VI{\textit{nhìn vào mặt tích cực} của vấn đề.}
            \EX{He encouraged me to look on the bright side after the setback.}
            \EX{Try to look on the bright side—you learned a lot.}
            \CO{always/try to look on the bright side; see the silver lining}
            \end{ExplainCard}

            \begin{ExplainCard}{be made for each other}{idiom}
            \EN{(of two people) to be perfectly suited in character or interests.}
            \SY{be perfectly matched; be a perfect fit}
            \VI{\textit{sinh ra là dành cho nhau}.}
            \EX{Everyone says they are made for each other.}
            \EX{Their shared values show they’re made for each other.}
            \CO{be truly/clearly made for each other}
            \end{ExplainCard}

            \begin{ExplainCard}{fervent}{adj}
            \EN{having or displaying strong and sincere feelings; intensely devoted.}
            \SY{ardent; zealous; impassioned}
            \VI{\textit{nhiệt thành}, mãnh liệt.}
            \EX{They are fervent supporters of live music.}
            \EX{Her fervent belief in education drives her work.}
            \CO{fervent supporter/believer; fervent hope/desire}
            \end{ExplainCard}

            \begin{ExplainCard}{in a pickle}{idiom}
            \EN{in a difficult or troublesome situation.}
            \SY{in a bind; in a jam; in trouble}
            \VI{\textit{gặp rắc rối}, ở thế khó.}
            \EX{I was in a pickle after losing my wallet.}
            \EX{Without a backup plan, the team could be in a pickle.}
            \CO{be/find yourself in a pickle; get out of a pickle}
            \end{ExplainCard}

            \begin{ExplainCard}{and the like(s)}{phrase}
            \EN{and other things/people of the same kind.}
            \SY{and so on; and similar things; and the rest}
            \VI{\textit{vân vân; những thứ/loại tương tự}.}
            \EX{He spends money on shoes, gadgets, and the like.}
            \EX{The lab tests for pesticides, heavy metals, and the like.}
            \CO{X, Y, and the like(s); similar items and the like}
            \end{ExplainCard}
        \end{VocabExplain}


    \noindent
    \textbf{Part 3.}
    \begin{qa}{In what ways can people in a family be similar to each other?}
    This \textbf{open-ended question} is really interesting. From my own perspective, family members can \textbf{bear similarity} in both \textbf{physical appearance} and personal characteristics. This \textbf{resemblance} can be explained by genetics running in a \textbf{lineage}. For example, James Brolin, starring as Thanos in Avengers, the highest-grossing film of all time, bears a strong resemblance to his dad, Josh Brolin, a legendary actor and producer at Hollywood.
    \end{qa}

    \begin{qa}{Do you think that daughters are always more similar to mothers than to male relatives? What about sons and fathers?}
    Well, \textbf{my mind is a blank} when you ask me about this, simply because I am not a genetic researcher. Based on my little knowledge of genetics, it is not \textbf{scientifically proven} that daughters always closely resemble their mothers than their fathers and vice versa. I think this is a natural \textbf{phenomenon} and requires more investigations to figure out the \textbf{enigma}. Speaking differently, it depends. In my case, my daughter is said to \textbf{be a spitting image} of me, her dad.
    \end{qa}

    \begin{qa}{In terms of personality, are people more influenced by their family or by their friends? In what ways?}
    I guess the family will exert greater influences on personal \textbf{traits}. A proper \textbf{upbringing} in a child's formative years will lead to a good personality. By contrast, \textbf{a troubled childhood} is a cause to \textbf{undesirable} characteristics such as \textbf{self-centeredness}, a short fuse, etc. From a friendship perspective, \textbf{peers} will be likely to have some impacts on individuals regarding fashion or eating habits.
    \end{qa}

    \begin{qa}{Where can people in your country get information about genetic research?}
    For the most part, I do not think people in my country \textbf{concern themselves with} genetic research as they might see the reseach too \textbf{theoretical} and \textbf{mundane}. Even if we \textbf{read up on} papers about this topic, I believe my fellow countrymen are still \textbf{none the wiser} about it. Despite that, the significance of genetic research is \textbf{apparent} and most of \textbf{in-depth} investigations are transparent to all inhabitants through media and \textbf{official documents}.
    \end{qa}

    \begin{qa}{How do people in your country feel about genetic research?}
    As I have mentioned, some people, especially healthy ones, find the research has no \textbf{relevance} to their life, so they just \textbf{flick through} these news for fun. However, due to the development of \textbf{genetic-related diseases}, an increasing number of people are now more conscious of genetics and things like genetic \textbf{hypothesis} to find therapies for \textbf{fatal} diseases such as cancer.
    \end{qa}

    \begin{qa}{Should this research be funded by governments or private companies? Why?}
    My answer is both. The authorities cannot fully \textbf{assume} this responsibility due to financial \textbf{constraints}. The state budget cannot only be \textbf{allocated} to genetic research but also to other areas such as education or transportation. Therefore, \textbf{private-funded research} is another solution to \textbf{tackle} the problem because it also strikes a blow for the development of genetic research for the sake of human kind.
    \end{qa}

        \begin{VocabExplain}[Part 3]
            \begin{ExplainCard}{open-ended question}{n}
            \EN{a question with no single correct answer that invites broad, exploratory responses.}
            \SY{exploratory; non-restrictive; unstructured}
            \VI{\textit{câu hỏi mở}, không có một đáp án cố định, khuyến khích trả lời tự do.}
            \EX{It’s an open-ended question, so students can justify different views.}
            \EX{Open-ended questions often reveal deeper attitudes.}
            \CO{pose/ask an open-ended question; open-ended survey items}
            \end{ExplainCard}

            \begin{ExplainCard}{bear similarity (to)}{phrase}
            \EN{to show or have a likeness to someone or something.}
            \SY{resemble; mirror; have affinity with}
            \VI{\textit{mang/cho thấy sự tương đồng (với)} ai/cái gì.}
            \EX{The two proposals bear striking similarity.}
            \EX{She bears similarity to her mother in mannerisms.}
            \CO{bear a/close/striking similarity to}
        \end{ExplainCard}

        \begin{ExplainCard}{physical appearance}{n}
            \EN{the way a person looks, including features, build, and general look.}
            \SY{outward appearance; looks; physique}
            \VI{\textit{ngoại hình}; vẻ bề ngoài.}
            \EX{Employers should not judge candidates by physical appearance alone.}
            \EX{His physical appearance changed after he lost weight.}
            \CO{improve/change physical appearance; outward appearance}
        \end{ExplainCard}

        \begin{ExplainCard}{resemblance}{n}
            \EN{the state of being similar to someone or something, especially in looks.}
            \SY{likeness; similarity; affinity}
            \VI{\textit{sự giống nhau}, nhất là về ngoại hình.}
            \EX{There’s a close resemblance between the sisters.}
            \EX{The sketch bears little resemblance to the original.}
            \CO{bear/show a resemblance to; close/striking/family resemblance}
        \end{ExplainCard}

        \begin{ExplainCard}{lineage}{n}
            \EN{line of descent from an ancestor; family line.}
            \SY{ancestry; bloodline; pedigree}
            \VI{\textit{dòng dõi}, huyết thống.}
            \EX{He traced his lineage back to the 18th century.}
            \EX{The dynasty prided itself on royal lineage.}
            \CO{maternal/paternal lineage; trace one’s lineage}
        \end{ExplainCard}

        \begin{ExplainCard}{my mind is a blank}{idiom}
            \EN{to be unable to think of or remember anything at a particular moment.}
            \SY{go blank; have a mental block}
            \VI{\textit{trống rỗng đầu óc}, không nhớ ra điều gì.}
            \EX{When asked about dates, my mind is a blank.}
            \EX{In the exam his mind went blank for a minute.}
            \CO{my mind went/completely is a blank}
        \end{ExplainCard}

        \begin{ExplainCard}{scientifically proven}{adj}
            \EN{demonstrated to be true or effective through rigorous scientific evidence.}
            \SY{evidence-based; empirically validated}
            \VI{\textit{được chứng minh khoa học}.}
            \EX{The benefits are scientifically proven in clinical trials.}
            \EX{That claim is not scientifically proven.}
            \CO{scientifically proven method/effect/link}
        \end{ExplainCard}

        \begin{ExplainCard}{phenomenon}{n}
            \EN{(1) an observable fact or event, especially unusual or remarkable.\\(2) (infml) an exceptional person or thing.}
            \SY{(1) occurrence; event \quad (2) marvel; standout}
            \VI{(1) \textit{hiện tượng}.\\(2) người/vật \textit{phi thường}.}
            \EX{(1) Climate change is a complex global phenomenon.}
            \EX{(2) The young pianist is a phenomenon in the classical scene.}
            \CO{natural/social phenomenon; explain/observe a phenomenon}
        \end{ExplainCard}

        \begin{ExplainCard}{enigma}{n}
            \EN{a person or thing that is mysterious and difficult to understand.}
            \SY{mystery; puzzle; riddle}
            \VI{\textit{điều bí ẩn}; câu đố.}
            \EX{The cause of the crash remains an enigma.}
            \EX{He’s an enigma even to his closest friends.}
            \CO{remain/pose/solve an enigma}
        \end{ExplainCard}

        \begin{ExplainCard}{be a spitting image (of)}{idiom}
            \EN{to look extremely similar to someone else.}
            \SY{be the image of; be a dead ringer for}
            \VI{\textit{giống y đúc} (ai).}
            \EX{She is the spitting image of her mother.}
            \EX{The boy is a spitting image of his dad at that age.}
            \CO{be/look the spitting image of sb}
        \end{ExplainCard}

        \begin{ExplainCard}{trait}{n}
            \EN{a particular quality that distinguishes a person or thing.}
            \SY{characteristic; attribute; hallmark}
            \VI{\textit{đặc điểm}; nét tính cách.}
            \EX{Honesty is a valuable personality trait.}
            \EX{Some traits are strongly influenced by genetics.}
            \CO{personality/character traits; inherited traits}
        \end{ExplainCard}

        \begin{ExplainCard}{upbringing}{n}
            \EN{the way a child is raised and educated at home.}
            \SY{child-rearing; nurture; socialization}
            \VI{\textit{sự giáo dục} trong gia đình; cách nuôi dạy.}
            \EX{A supportive upbringing helps build resilience.}
            \EX{He had a strict religious upbringing.}
            \CO{good/strict/proper upbringing; have/receive an upbringing}
        \end{ExplainCard}

        \begin{ExplainCard}{a troubled childhood}{phrase}
            \EN{a period of growing up marked by serious difficulties or instability.}
            \SY{difficult; rough; turbulent childhood}
            \VI{\textit{tuổi thơ bất ổn/khó khăn}.}
            \EX{He overcame a troubled childhood to finish college.}
            \EX{A troubled childhood can affect adult relationships.}
            \CO{overcome/suffer from a troubled childhood}
        \end{ExplainCard}

        \begin{ExplainCard}{undesirable}{adj}
            \EN{not wanted or likely to cause problems or harm.}
            \SY{unwanted; adverse; detrimental}
            \VI{\textit{không mong muốn}; bất lợi.}
            \EX{Smoking has many undesirable effects.}
            \EX{The policy led to undesirable outcomes.}
            \CO{undesirable effect/consequence/behavior}
        \end{ExplainCard}

        \begin{ExplainCard}{self-centeredness}{n}
            \EN{excessive focus on oneself and one’s own needs or interests.}
            \SY{egocentrism; selfishness; self-absorption}
            \VI{\textit{tính ích kỷ; chỉ nghĩ đến bản thân}.}
            \EX{Self-centeredness can damage friendships.}
            \EX{Therapy helped reduce his self-centeredness.}
            \CO{display/show self-centeredness; chronic self-centeredness}
        \end{ExplainCard}

        \begin{ExplainCard}{peer}{n}
            \EN{a person of the same age, status, or ability as another.}
            \SY{contemporary; equal; counterpart}
            \VI{\textit{bạn đồng trang lứa}; người cùng vị thế.}
            \EX{Teenagers are influenced by their peers.}
            \EX{She is well respected among her academic peers.}
            \CO{peer pressure/group; among/with peers}
        \end{ExplainCard}

        \begin{ExplainCard}{concern oneself with}{phr.v}
            \EN{to involve or occupy oneself with something; to care or worry about it.}
            \SY{preoccupy oneself with; busy oneself with; bother about}
            \VI{\textit{bận tâm/quan tâm} đến; để ý đến.}
            \EX{Don’t concern yourself with gossip.}
            \EX{Many citizens rarely concern themselves with policy details.}
            \CO{concern oneself with/about; not concern yourself with}
        \end{ExplainCard}

        \begin{ExplainCard}{theoretical}{adj}
            \EN{concerned with ideas and principles rather than practical application.}
            \SY{abstract; conceptual; academic}
            \VI{\textit{mang tính lý thuyết}.}
            \EX{Students need both theoretical and practical knowledge.}
            \EX{His model is elegant but purely theoretical.}
            \CO{theoretical framework/basis/knowledge}
        \end{ExplainCard}

        \begin{ExplainCard}{mundane}{adj}
            \EN{ordinary and not interesting; lacking excitement.}
            \SY{prosaic; humdrum; pedestrian}
            \VI{\textit{bình thường, tẻ nhạt}.}
            \EX{She finds office paperwork mundane.}
            \EX{Science can transform the mundane into the fascinating.}
            \CO{mundane task/detail/routine}
        \end{ExplainCard}

        \begin{ExplainCard}{read up on}{phr.v}
            \EN{to study or research a subject thoroughly, especially by reading.}
            \SY{research; study; familiarize oneself with}
            \VI{\textit{tìm đọc/tra cứu kỹ} về (một chủ đề).}
            \EX{I’m reading up on genetics for my presentation.}
            \EX{Before traveling, read up on local laws.}
            \CO{read up on/about; read up extensively}
        \end{ExplainCard}

        \begin{ExplainCard}{none the wiser}{idiom}
            \EN{still not knowing or understanding something after being told or after an event.}
            \SY{no better informed; still clueless}
            \VI{\textit{vẫn chẳng hiểu/vẫn không biết} gì hơn.}
            \EX{He explained it twice, and I was none the wiser.}
            \EX{After the meeting, we were none the wiser about the plan.}
            \CO{leave sb none the wiser; be none the wiser}
        \end{ExplainCard}

        \begin{ExplainCard}{apparent}{adj}
            \EN{(1) obvious and easy to notice.\\(2) seeming to be true, though possibly not.}
            \SY{(1) evident; manifest \quad (2) ostensible; seeming}
            \VI{(1) \textit{rõ ràng, hiển nhiên}.\\(2) \textit{có vẻ như} vậy.}
            \EX{(1) The error was apparent to everyone.}
            \EX{(2) His apparent calm hid deep anxiety.}
            \CO{make sth apparent; apparent reason/benefit; apparent contradiction}
        \end{ExplainCard}

        \begin{ExplainCard}{in-depth}{adj/adv}
            \EN{thorough and detailed; at a detailed level.}
            \SY{thorough; comprehensive; exhaustive}
            \VI{\textit{kỹ lưỡng, chuyên sâu}.}
            \EX{They conducted an in-depth investigation.}
            \EX{The report examines the issue in depth.}
            \CO{in-depth analysis/study/interview}
        \end{ExplainCard}

        \begin{ExplainCard}{official documents}{n}
            \EN{documents issued or recognized by an authority, especially a government.}
            \SY{formal records; official paperwork; documentation}
            \VI{\textit{tài liệu/chứng từ chính thức}.}
            \EX{You must submit official documents to register.}
            \EX{The policy is stated in official documents.}
            \CO{issue/submit/verify official documents; official documentation}
        \end{ExplainCard}

        \begin{ExplainCard}{relevance (to)}{n}
            \EN{the quality of being directly connected with or important to the matter at hand.}
            \SY{pertinence; applicability; bearing}
            \VI{\textit{tính liên quan/thiết thực (đối với)}.}
            \EX{Some questioned the relevance of the study to practice.}
            \EX{Her examples have clear relevance to the topic.}
            \CO{have/lack relevance to/for; of little/no relevance}
        \end{ExplainCard}

        \begin{ExplainCard}{flick through}{phr.v}
            \EN{to look quickly through the pages of a book, magazine, etc.}
            \SY{skim; leaf through; browse}
            \VI{\textit{lật xem qua}, đọc lướt.}
            \EX{He flicked through the report on the train.}
            \EX{I just flicked through the magazine for the pictures.}
            \CO{flick/leaf/skim through a book/magazine}
        \end{ExplainCard}

        \begin{ExplainCard}{genetic-related diseases}{phrase}
            \EN{diseases linked to abnormalities in genes or heredity.}
            \SY{genetic disorders; hereditary diseases}
            \VI{\textit{bệnh liên quan đến di truyền/genes}.}
            \EX{Screening can detect some genetic-related diseases early.}
            \EX{Family history increases the risk of genetic-related diseases.}
            \CO{risk of/therapy for genetic-related diseases; screen for}
        \end{ExplainCard}

        \begin{ExplainCard}{hypothesis}{n}
            \EN{a tentative explanation that can be tested by study or experiment.}
            \SY{supposition; proposition; conjecture}
            \VI{\textit{giả thuyết}.}
            \EX{They formulated a hypothesis about gene interaction.}
            \EX{Data failed to support the original hypothesis.}
            \CO{formulate/test/support/reject a hypothesis; working/null hypothesis}
        \end{ExplainCard}

        \begin{ExplainCard}{fatal}{adj}
            \EN{(1) causing death.\\(2) leading to failure or disaster.}
            \SY{(1) deadly; lethal \quad (2) ruinous; catastrophic}
            \VI{(1) \textit{chí tử}, gây chết người.\\(2) \textit{tai hại}, đưa đến thất bại.}
            \EX{(1) The crash caused fatal injuries.}
            \EX{(2) A fatal flaw undermined the project.}
            \CO{fatal disease/accident/error; prove fatal}
        \end{ExplainCard}

        \begin{ExplainCard}{assume}{v}
            \EN{(1) to take on (a duty, role, or responsibility).\\(2) to suppose something without proof.}
            \SY{(1) undertake; shoulder \quad (2) presume; take for granted}
            \VI{(1) \textit{đảm nhận/gánh vác}.\\(2) \textit{cho rằng}, giả định.}
            \EX{(1) The agency assumed responsibility for funding.}
            \EX{(2) Don’t assume the results will be the same.}
            \CO{assume responsibility/office/control; assume that + clause}
        \end{ExplainCard}

        \begin{ExplainCard}{constraint}{n}
            \EN{a limitation or restriction that controls what is possible.}
            \SY{limitation; restraint; bottleneck}
            \VI{\textit{sự hạn chế}; ràng buộc.}
            \EX{Budget constraints forced a smaller study.}
            \EX{Time constraints shaped the design.}
            \CO{budget/time/resource constraints; impose/face constraints}
        \end{ExplainCard}

        \begin{ExplainCard}{allocate}{v}
            \EN{to distribute resources or duties for a particular purpose.}
            \SY{assign; apportion; earmark}
            \VI{\textit{phân bổ}; cấp cho (một mục đích).}
            \EX{Funds were allocated to genetic research.}
            \EX{We must allocate time for peer review.}
            \CO{allocate funds/resources/time to/for; allocation of}
        \end{ExplainCard}

        \begin{ExplainCard}{private-funded research}{phrase}
            \EN{research financed by private entities (companies, foundations, individuals) rather than the state.}
            \SY{industry-sponsored; corporate-funded; privately financed research}
            \VI{\textit{nghiên cứu được tài trợ bởi khu vực tư nhân}.}
            \EX{Private-funded research can move faster than public projects.}
            \EX{Disclosure rules reduce bias in private-funded research.}
            \CO{private-funded research project/trial; funding from private companies}
        \end{ExplainCard}

        \begin{ExplainCard}{tackle}{v}
            \EN{(1) to deal with or try to solve a difficult problem.\\(2) (sport) to stop an opposing player by grabbing or knocking them down.}
            \SY{(1) address; confront; grapple with \quad (2) challenge; block}
            \VI{(1) \textit{xử lý/giải quyết}.\\(2) \textit{tackle} (trong thể thao).}
            \EX{(1) The policy aims to tackle fatal diseases.}
            \EX{(2) The defender cleanly tackled the striker.}
            \CO{tackle a problem/issue; tackle head-on; effective tackling}
        \end{ExplainCard}
        \end{VocabExplain}

    \begin{VocabHighlights}
        \VH{emphatic}{(adj) made in a forceful way, because the speaker feels very strongly about what they are saying}{(tính từ) mang tính nhấn mạnh}
        \VH{not to mention}{(phrase) used to introduce an additional fact or point that reinforces the point being made}{(cụm từ) thêm vào đó}
        \VH{headbanger}{(n) a fan or performer of heavy metal music}{(danh từ) 1 người hâm mộ dòng nhạc heavy metal}
        \VH{choreographer}{(n) a person who composes the sequence of steps and moves for a performance of dance}{(danh từ) người biên đạo nhảy}
        \VH{to be cut out to}{(phr.v) have exactly the right qualities for a particular role, task, or job}{(cụm động từ) đủ khả năng làm gì}
        \VH{to have two left feet}{(idiom) to move in a very awkward way when dancing}{(thành ngữ) không thể nhảy được}
        \VH{to spring to one’s mind}{(phrase) to suddenly or immediately appear, materialize, or come to the forefront in one’s mind}{(cụm từ) bất chợt nảy ra trong đầu}
        \VH{to make a name for oneself}{(idiom) become well known}{(thành ngữ) trở nên nổi tiếng}
        \VH{to grab hold of}{(phrase) to quickly take and hold (someone or something) with the hand or arms}{(cụm từ) nhanh tay nắm lấy}
        \VH{to grow in popularity}{(phrase) to become more popular}{(cụm từ) trở nên nổi tiếng hơn}
        \VH{to partake in}{(v) to take part in or experience something along with others}{(động từ) tham gia vào}
        \VH{to be named after}{(phr.v) to be given the same name as someone}{(cụm động từ) được đặt tên theo ai đó}
        \VH{to be well-built}{(adj) be muscular}{(tính từ) vạm vỡ, cơ bắp}
        \VH{to work out}{(phr.v) to do exercise, hit the gym}{(cụm động từ) luyện tập thể dục thể thao}
        \VH{to take great pride in}{(phrase) to be proud of}{(cụm từ) tự hào}
        \VH{to be like two peas in a pod}{(idiom) to be very identical}{(thành ngữ) giống như hai giọt nước}
        \VH{to get on (well)}{(phr.v) to have a good relationship}{(cụm động từ) hòa thuận}
        \VH{to fall out}{(phrase) to quarrel}{(cụm động từ) cãi nhau}
        \VH{to crack somebody up}{(phr.v) to laugh a lot}{(cụm động từ) cười rất nhiều}
        \VH{the life and soul of the party}{(idiom) to be the center of the parties}{(thành ngữ) tâm điểm, linh hồn của tiệc}
        \VH{to be made for each other}{(idiom) to be perfectly matched}{(thành ngữ) hợp nhau mọi thứ}
        \VH{fervent}{(adj) showing strong \& sincere feelings}{(tính từ) nhiệt thành, nhiệt tâm}
        \VH{in a pickle}{(idiom) in difficult times}{(thành ngữ) lúc khó khăn}
        \VH{the likes}{(idiom) other things}{(thành ngữ) đại loại}
        \VH{an opened-ended question}{(phrase) a statement which requires a response}{(cụm từ) câu hỏi mở}
        \VH{to bear}{(v) to have or continue to have something}{(động từ) có đặc điểm}
        \VH{physical appearance}{(phrase) the first thing you see when you look at someone could be their hair, clothes, nose, or figure}{(cụm từ) ngoại hình}
        \VH{resemblance}{(n) the fact of being or looking similar to somebody/something}{(danh từ) sự giống nhau}
        \VH{a lineage}{(n) the series of families that somebody comes from originally}{(danh từ) dòng dõi}
        \VH{my mind is a blank}{(idiom) used for saying that someone becomes unable to remember or think anything}{(thành ngữ) trong đầu trống rỗng}
        \VH{scientifically proven}{(phr.v) evidence relate to science}{(cụm động từ) bằng chứng mang tính khoa học}
        \VH{phenomenon}{(n) a fact or an event in nature or society, especially one that is not fully understood}{(danh từ) hiện tượng}
        \VH{enigma}{(n) a person, thing or situation that is mysterious and difficult to understand}{(danh từ) sự bí ẩn}
        \VH{to be a spitting image of somebody}{(idiom) to look exactly like someone else}{(thành ngữ) trông giống y hệt}
        \VH{traits}{(n) characteristics, behaviours}{(danh từ) tính cách, cách cư xử}
        \VH{upbringing}{(n) the way in which a child is cared for and taught how to behave while it is growing up}{(danh từ) sự nuôi dạy}
        \VH{a troubled childhood}{(phrase) an unhappy childhood}{(cụm từ) tuổi thơ không vui vẻ, êm đềm}
        \VH{undesirable}{(adj) not wanted or approved of; likely to cause trouble or problems}{(tính từ) không mong muốn}
        \VH{self-centeredness}{(n) a self centered person is someone who only thinks about himself, his own needs and his own interests, or is actions or behaviors}{(danh từ) sự ích kỉ, chỉ biết tới bản thân}
        \VH{peers}{(n) a person who is the same age or who has the same social status as you}{(danh từ) người ngang hàng (về độ tuổi hoặc địa vị)}
        \VH{to concern somebody with}{(phrase) to care about something}{(cụm từ) quan tâm tới ai, cái gì}
        \VH{theoretical}{(adj) concerned with the ideas and principles on which a particular subject is based, rather than with practice and experiment}{(tính từ) thuộc giả định, lý thuyết}
        \VH{mundane}{(adj) not interesting or exciting}{(tính từ) vô vị, tầm thường}
        \VH{to read up on}{(phr.v) spend time reading in order to find out information about something}{(cụm động từ) đọc để tìm hiểu về}
        \VH{none the wiser}{(idiom) know no more than before}{(thành ngữ) không biết gì hơn}
        \VH{apparent}{(adj) easy to see or understand}{(tính từ) rõ ràng}
        \VH{in-depth}{(adj) very thorough and detailed}{(tính từ) kỹ lưỡng, tỉ mỉ}
        \VH{relevance}{(n) a close connection with the subject you are discussing or the situation you are thinking about}{(danh từ) sự liên quan}
        \VH{to flick through}{(phr.v) to turn the pages of a book, etc. quickly and look at them without reading everything}{(cụm động từ) nhìn qua; xem qua}
        \VH{genetic-related diseases}{(phrase) a genetic problem caused by one or more abnormalities formed in the genome}{(cụm từ) bệnh liên quan tới gen}
        \VH{hypothesis}{(n) an idea or explanation of something that is based on a few known facts but that has not yet been proved to be true or correct}{(danh từ) giả thuyết}
        \VH{to assume}{(v) to take or begin to have power or responsibility}{(động từ) đảm đương}
        \VH{constraint}{(n) a thing that limits or restricts somebody, something}{(danh từ) ràng buộc, giới hạn}
        \VH{to be allocated to}{(p2) to give something officially to somebody/something for a particular purpose}{(phân từ 2) phân bổ cho}
        \VH{private-funded research}{(phrase) research carried out by private fund}{(cụm từ) đầu tư nghiên cứu tư nhân}
        \VH{to tackle}{(v) to try to deal with something or someone}{(động từ) giải quyết}
    \end{VocabHighlights}

    \end{test}

    \begin{test}{TEST 2}
    \noindent
    \textbf{Part 1. Musical Instruments}
    \begin{qa}{Which instrument do you like listening to most? (Why?)}
    \textbf{All things considered}, guitar is my favorite instrument. You know, I am a hardcore metal fan so listening to the sounds of \textbf{distorted} guitars, aggressive rhythms, and dense bass, is \textbf{music to my ears}. I'm \textbf{enthused} by the idea of immersing myself in the guitar drift at the \textbf{transition} from the verse to the chorus or bridge of a metal song and rocking my body to the aggressive rhythms of rock in general.
    \end{qa}

    \begin{qa}{Have you ever learned to play a musical instrument? (Which one?)}
    I have. It was before I came to the U.K for my master degree that I \textbf{took up} a musical instrument, an acoustic guitar in particular. At first I thought that I could learn the instrument to \textbf{kill time} in England. However, I was \textbf{pressed for} work and did not have much time to practice so I quit it. Maybe playing an instrument is not my \textbf{forte}. Since then, I have not touched the guitar again.
    \end{qa}

    \begin{qa}{Do you think children should learn to play a musical instrument at school? (Why/ Why not?)}
    Yes, definitely. Learning how to play a musical instrument surely can \textbf{do wonders} for children at school. In particular, by mastering the guitar techniques, a child may perform at various events. Once he or she is given a \textbf{standing ovation}, this will be an unforgettable memory for them. Playing a musical instrument may also relax one's mind as well. Moreover, after hours of intensive study at school, a music lesson might relieve students' stress and help them recover energy as well.
    \end{qa}

    \begin{qa}{How easy would it be to learn to play an instrument without a teacher? (Why?)}
    It must be \textbf{an uphill task}. In this 4.0 era, online tutorials for musical instruments are \textbf{prevalent} on social networking sites like Facebook and video clip hosting ones like Youtube. However, for a \textbf{novice}, without the help of a teacher, it will be a struggle for one to understand how to play an instrument and he or she will soon \textbf{go off it}.
    \end{qa}

        \begin{VocabExplain}[Part 1]
            \begin{ExplainCard}{all things considered}{phrase}
            \EN{taking everything into account; overall, after weighing all factors.}
            \SY{on balance; overall; in the grand scheme}
            \VI{\textit{xét mọi yếu tố}; \textit{tổng thể mà nói}.}
            \EX{All things considered, guitar is still my favorite.}
            \EX{All things considered, the project went well.}
            \CO{All things considered, + clause}
            \end{ExplainCard}

        \begin{ExplainCard}{distorted}{adj}
            \EN{(1) (of sound/image) altered so that it is unclear or harsh.\\(2) twisted or misrepresented from the original truth or shape.}
            \SY{(1) warped; gritty; degraded \quad (2) skewed; mangled; misrepresented}
            \VI{(1) \textit{bị méo/biến dạng} (âm thanh/hình ảnh).\\(2) \textit{bị bóp méo}, sai lệch.}
            \EX{(1) I love the distorted tone of rock guitars.}
            \EX{(2) The report gave a distorted picture of events.}
            \CO{distorted guitar/tone/signal; a distorted view/image}
        \end{ExplainCard}

        \begin{ExplainCard}{music to my ears}{idiom}
            \EN{something very pleasant to hear; welcome news or sound.}
            \SY{welcome news; delightful to hear}
            \VI{\textit{nghe thật thích tai}; tin vui.}
            \EX{Your approval is music to my ears.}
            \EX{That smooth riff is music to my ears.}
            \CO{be/ sound like music to sb’s ears}
        \end{ExplainCard}

        \begin{ExplainCard}{enthused (about/over)}{adj/v}
            \EN{feeling or showing strong excitement and interest about something.}
            \SY{excited; enthusiastic; fired up}
            \VI{\textit{hào hứng}, \textit{phấn khích}.}
            \EX{He’s really enthused about the new album.}
            \EX{Fans enthused over the band’s comeback.}
            \CO{be/get enthused about; sound enthused}
        \end{ExplainCard}

        \begin{ExplainCard}{transition (from A to B)}{n}
            \EN{(1) a change from one state or stage to another.\\(2) (music) movement between sections (e.g., verse to chorus).}
            \SY{(1) shift; changeover \quad (2) modulation; passage}
            \VI{(1) \textit{sự chuyển đổi}.\\(2) (nhạc) \textit{đoạn chuyển}, chuyển phần.}
            \EX{(1) The transition from school to work can be tough.}
            \EX{(2) I love the transition from verse into the chorus.}
            \CO{smooth/gradual transition; transition from A to B}
        \end{ExplainCard}

        \begin{ExplainCard}{take up}{phr.v}
            \EN{(1) to start a hobby, activity, or job.\\(2) to occupy time, space, or attention.}
            \SY{(1) pick up; embark on \quad (2) consume; occupy}
            \VI{(1) \textit{bắt đầu} (môn/hoạt động).\\(2) \textit{chiếm} (thời gian/không gian).}
            \EX{(1) She took up the acoustic guitar last year.}
            \EX{(2) Commuting takes up too much of my day.}
            \CO{take up a hobby/sport/instrument; take up space/time}
        \end{ExplainCard}

        \begin{ExplainCard}{kill time}{idiom}
            \EN{to do something while waiting so that time seems to pass more quickly.}
            \SY{pass the time; while away the time}
            \VI{\textit{giết thời gian}.}
            \EX{I practiced chords to kill time at the airport.}
            \EX{We played cards just to kill time.}
            \CO{kill time by + V\textsubscript{ing}; just to kill time}
        \end{ExplainCard}

        \begin{ExplainCard}{pressed for (time/money/space)}{adj}
            \EN{needing more of something because there is not enough of it.}
            \SY{short of; strapped for; under pressure}
            \VI{\textit{thiếu/không đủ} (thời gian, tiền, chỗ).}
            \EX{I was pressed for time and stopped practicing.}
            \EX{Many families are pressed for money this year.}
            \CO{be/get pressed for time/cash/space}
        \end{ExplainCard}

        \begin{ExplainCard}{forte}{n}
            \EN{(1) something that a person does particularly well; strong point.\\(2) (music) a dynamic marking meaning “loud.”}
            \SY{(1) strength; specialty; strong suit \quad (2) loud}
            \VI{(1) \textit{thế mạnh}, sở trường.\\(2) (nhạc) \textit{to}, lớn.}
            \EX{(1) Sight-reading isn’t my forte.}
            \EX{(2) The score marks this passage \textit{forte}.}
            \CO{not my forte; make X your forte; play forte}
        \end{ExplainCard}

        \begin{ExplainCard}{do wonders (for)}{idiom}
            \EN{to have a very good effect on someone or something.}
            \SY{work miracles; benefit greatly; transform}
            \VI{\textit{mang lại hiệu quả tuyệt vời (cho)}.}
            \EX{Music lessons can do wonders for children.}
            \EX{A short break did wonders for my focus.}
            \CO{do wonders for/with; work wonders}
        \end{ExplainCard}

        \begin{ExplainCard}{standing ovation}{n}
            \EN{prolonged applause in which the audience stands to show great approval.}
            \SY{rousing applause; thunderous ovation}
            \VI{\textit{mọi người đứng dậy vỗ tay tán thưởng}.}
            \EX{The young guitarist received a standing ovation.}
            \EX{Her solo drew a standing ovation from the crowd.}
            \CO{get/receive a standing ovation}
        \end{ExplainCard}

        \begin{ExplainCard}{an uphill task}{idiom}
            \EN{a job or goal that is very difficult and requires great effort.}
            \SY{arduous task; steep challenge; formidable effort}
            \VI{\textit{nhiệm vụ gian nan}, đòi hỏi nhiều nỗ lực.}
            \EX{Learning alone can be an uphill task.}
            \EX{Rebuilding trust is always an uphill task.}
            \CO{prove/become an uphill task; face an uphill task}
        \end{ExplainCard}

        \begin{ExplainCard}{prevalent}{adj}
            \EN{widespread or commonly occurring at a particular time or place.}
            \SY{widespread; commonplace; rife}
            \VI{\textit{phổ biến}, lan rộng.}
            \EX{Online tutorials are prevalent nowadays.}
            \EX{This habit is prevalent among teenagers.}
            \CO{become/remain prevalent; prevalent trend/belief}
        \end{ExplainCard}

        \begin{ExplainCard}{novice}{n}
            \EN{a person who is new to and inexperienced in a job or activity.}
            \SY{beginner; newcomer; apprentice}
            \VI{\textit{người mới}, còn thiếu kinh nghiệm.}
            \EX{As a novice, she needs guidance.}
            \EX{Novices often benefit from a structured course.}
            \CO{novice musician/driver; a complete/total novice}
        \end{ExplainCard}

        \begin{ExplainCard}{go off (sth)}{phr.v}
            \EN{(1) to stop liking or lose interest in something.\\(2) (of food) to spoil; (of a device) to explode or make a noise.}
            \SY{(1) tire of; lose interest in \quad (2) spoil; detonate/sound}
            \VI{(1) \textit{chán/không còn thích}.\\(2) (thức ăn) \textit{ôi thiu}; (thiết bị) \textit{nổ/ reo}.}
            \EX{(1) Without a teacher, many learners soon go off it.}
            \EX{(2) The milk has gone off; the alarm went off at 6.}
            \CO{go off sth/doing; quickly go off; milk/food goes off; alarm/bomb goes off}
        \end{ExplainCard}
        \end{VocabExplain}

    \noindent
    \textbf{Part 2.}
    \begin{qa}{Describe something healthy you enjoy doing. You should say:}
    \begin{itemize}
    \item What you do
    \item Where you do it
    \item Who you do it with
    \item and explain why you think doing this is healthy.
    \end{itemize}

    When it comes to something good for health that I'm into, the thing that \textbf{springs to mind} is drinking orange juice on a daily basis. I formed and developed this habit two years ago after I had read an article about health on the Internet \textbf{by chance}. The article presented findings of some studies on extending \textbf{life expectancy}. The survey was \textbf{carried out} by some leading scientists. According to the article, 80\% of people aged 85 and more keep the habit of drinking one or two cups of orange juice every day, which \textbf{came as a surprise} to me. There are some impressive benefits of drinking orange juice. First, orange juice is \textbf{rich in vitamin C}, so we can boost our vitamin C intake. Vitamin C is a crucial vitamin that cannot be produced by the human body, and it helps us to \textbf{combat} free radicals. I was constantly amazed by the fact that orange juice can prevent cancer. Since vitamin C serves as an \textbf{antioxidant}, it also functions \textbf{as a shield} against many types of cancers. Last but not least, orange juice helps to boost immunity. Oranges are known for their powerful healing properties. This is why orange juice is given to patients or individuals who are recovering from an injury. Since a single serving of the juice offers a \textbf{handsome} amount of the vitamin C requirement of the body, it is safe to assume that orange juice is an effective way to \textbf{fortify} our immune system. Thanks to that article, my life has totally changed for the better because I can lead a healthy life.
    \end{qa}

        \begin{VocabExplain}[Part 2]
            \begin{ExplainCard}{spring to mind}{idiom}
            \EN{to come quickly into your thoughts; occur to you without effort.}
            \SY{come to mind; occur to; pop into one’s head}
            \VI{\textit{nảy ra trong đầu} ngay lập tức.}
            \EX{When I think of healthy habits, orange juice \textbf{springs to mind}.}
            \EX{Several solutions sprang to mind during the meeting.}
            \CO{what/first/instantly springs to mind; spring to mind that + clause}
            \end{ExplainCard}

        \begin{ExplainCard}{by chance}{phrase}
            \EN{happening accidentally and not planned; coincidentally.}
            \SY{accidentally; coincidentally; unintentionally}
            \VI{\textit{tình cờ}, \textit{ngẫu nhiên}.}
            \EX{I found the article by chance while browsing online.}
            \EX{We met by pure chance on the train.}
            \CO{meet/find/hear of + by chance; purely/by sheer chance}
        \end{ExplainCard}

        \begin{ExplainCard}{life expectancy}{n}
            \EN{the statistical average number of years a person is expected to live.}
            \SY{average lifespan; longevity (avg.)}
            \VI{\textit{tuổi thọ trung bình}.}
            \EX{Better healthcare has raised life expectancy worldwide.}
            \EX{Women generally have a higher life expectancy than men.}
            \CO{average/rising/low life expectancy; life expectancy at birth}
        \end{ExplainCard}

        \begin{ExplainCard}{carry out}{phr.v}
            \EN{to perform or conduct (a task, study, experiment, plan).}
            \SY{conduct; execute; implement}
            \VI{\textit{tiến hành}, thực hiện.}
            \EX{The survey was \textbf{carried out} by leading scientists.}
            \EX{The team carried out the plan on schedule.}
            \CO{carry out a study/survey/experiment; carry out reforms}
        \end{ExplainCard}

        \begin{ExplainCard}{come as a surprise (to sb)}{phrase}
            \EN{to be unexpected or astonishing for someone.}
            \SY{be unexpected; come out of the blue; take sb by surprise}
            \VI{\textit{khiến ai bất ngờ}; \textit{không ngờ tới}.}
            \EX{The high percentage of elderly drinkers came as a surprise to me.}
            \EX{His resignation came as a complete surprise.}
            \CO{come as a (complete/pleasant) surprise to sb}
        \end{ExplainCard}

        \begin{ExplainCard}{rich in (sth)}{adj phrase}
            \EN{containing a large amount of a particular nutrient or substance.}
            \SY{high in; abundant in; packed with}
            \VI{\textit{giàu, dồi dào} (chất nào đó).}
            \EX{Citrus fruits are \textbf{rich in vitamin C}.}
            \EX{Whole grains are rich in fiber.}
            \CO{rich in vitamins/minerals/fiber/protein}
        \end{ExplainCard}

        \begin{ExplainCard}{combat}{v}
            \EN{(1) to take action to reduce or stop something harmful.\\(2) to fight in a battle (lit.).}
            \SY{(1) fight; counter; curb \quad (2) battle; engage}
            \VI{(1) \textit{chống lại}, \textit{ngăn chặn}.\\(2) \textit{giao chiến}.}
            \EX{(1) Vitamin C helps \textbf{combat} free radicals.}
            \EX{(2) Troops were deployed to combat the invaders.}
            \CO{combat disease/poverty/crime; measures to combat X}
        \end{ExplainCard}

        \begin{ExplainCard}{antioxidant}{n/adj}
            \EN{(1) \textbf{n.} a substance that inhibits oxidation and neutralizes free radicals.\\(2) \textbf{adj.} preventing oxidation.}
            \SY{free-radical scavenger; protective compound}
            \VI{(1) \textbf{danh từ:} \textit{chất chống oxy hoá}.\\(2) \textbf{tính từ:} \textit{chống oxy hoá}.}
            \EX{(1) Vitamin C is a powerful \textbf{antioxidant}.}
            \EX{(2) The tea’s antioxidant properties are well documented.}
            \CO{powerful/natural antioxidants; antioxidant properties/effects; rich in antioxidants}
        \end{ExplainCard}

        \begin{ExplainCard}{serve/act as a shield (against)}{phrase}
            \EN{to provide protection from harm or danger.}
            \SY{protect; safeguard; act as a barrier}
            \VI{\textit{làm lá chắn} (chống/đối phó với…).}
            \EX{Vitamin C functions \textbf{as a shield} against cell damage.}
            \EX{Insurance acts as a shield against financial loss.}
            \CO{serve/act/function as a shield; a shield against X}
        \end{ExplainCard}

        \begin{ExplainCard}{handsome}{adj}
            \EN{(1) (of an amount) large and generous.\\(2) (of a person) good-looking (esp. male).}
            \SY{(1) sizable; substantial; considerable \quad (2) good-looking; attractive}
            \VI{(1) \textit{khá lớn, hậu hĩnh}.\\(2) \textit{đẹp trai}.}
            \EX{(1) One serving provides a \textbf{handsome} amount of vitamin C.}
            \EX{(2) He’s a handsome actor with classic features.}
            \CO{a handsome amount/sum/profit; a handsome salary}
        \end{ExplainCard}

        \begin{ExplainCard}{fortify}{v}
            \EN{(1) to make stronger or more resilient.\\(2) to strengthen against attack (military) or to enrich food with nutrients.}
            \SY{(1) strengthen; reinforce; bolster \quad (2) secure; enrich}
            \VI{(1) \textit{củng cố}, tăng cường.\\(2) \textit{gia cố}/\textit{bổ sung vi chất}.}
            \EX{(1) Orange juice can \textbf{fortify} the immune system.}
            \EX{(2) Milk is often fortified with vitamin D.}
            \CO{fortify the immune system/defences; fortify food with + nutrient}
        \end{ExplainCard}
        \end{VocabExplain}

    \noindent
    \textbf{Part 3.}
    \begin{qa}{What do most people do to keep fit in your country?}
    There are \textbf{an abundance of} methods for people to \textbf{get lean}. \textbf{Hitting the gym to work out} is the most popular one among young people. Many fitness centers offer \textbf{a wide range} of facilities which are suitable for each exercise. However, the elderly tend to prefer to take \textbf{leisurely strolls} around a nearby park to \textbf{take delight in} the fresh air.
    \end{qa}

    \begin{qa}{How important is it for people to do some regular physical exercise?}
    I strongly believe \textbf{workout} really \textbf{does wonders} for our health. Firstly, doing exercise frequently is a fantastic way to \textbf{strengthen} \textbf{muscles} and improve the functionality of organs like \textbf{digestive} system or \textbf{cardiovascular} system. What’s more, regular exercise can \textbf{relieve} stress and depression. As a result, it is necessary for people to keep this routine on a daily basis.
    \end{qa}

    \begin{qa}{Why do some people think that modern lifestyles are not healthy?}
    Well, if we define all modern lifestyles are unhealthy, that is too \textbf{oversimplified}. However, we may \textbf{lapse into a sedentary lifestyle}, which is prevailing in the modern world because life, especially in cities, is getting more hectic and physical inactivity has somehow been taken for granted. Personally, I do \textbf{contend} that this lifestyle should be gotten rid of.
    \end{qa}

    \begin{qa}{Why do some people choose to lead unhealthy lives?}
    I suppose many may opt for an unhealthy life \textbf{as a last resort} due to some following reasons. Firstly, \textbf{the fast pace of life} \textbf{deprives} people of their free time to get proper relaxation, which is one of the main causes to health problems such as heart attack or even \textbf{stroke}. Moreover, a lack of awareness of health protection is another \textbf{culprit} of \textbf{deteriorating} health.
    \end{qa}

    \begin{qa}{Should individuals or governments be responsible for making people's lifestyle healthy?}
    In my opinion, this is not only one of government obligations but also personal responsibility. \textbf{For the sake} of social well-being, the government should launch some campaigns to \textbf{promote} the benefits of healthy lifestyles to the citizens to help them avoid \textbf{destructive} habits. From an individual perspective, health is an asset to every single person, and any medical conditions can \textbf{impede} opportunities in his life. All in all, an individual needs to raise his awareness to protect himself for his own good.
    \end{qa}

    \begin{qa}{What could be done to encourage people to live in a healthy way?}
    Of course, there are \textbf{a whole bunch} of measures which can be implemented to \textbf{combat} health issues. Political leaders could launch campaigns for health promotion which \textbf{heighten} the public awareness. On the other hand, individuals are encouraged to \textbf{equip} themselves with \textbf{comprehensive} knowledge about health protection and prevention.
    \end{qa}

        \begin{VocabExplain}[Part 3]
            \begin{ExplainCard}{an abundance of}{phrase}
                \EN{a very large quantity of something; more than enough.}
                \SY{a wealth of; plenty of; profusion of}
                \VI{\textit{rất nhiều}, \textit{dồi dào}.}
                \EX{There is \textbf{an abundance of} methods to keep fit.}
                \EX{The region enjoys an abundance of natural resources.}
                \CO{have/with an abundance of; an abundance of + plural noun}
            \end{ExplainCard}

            \begin{ExplainCard}{get lean}{phr.v}
                \EN{(1) to reduce body fat and become slim or toned.\\(2) (business) to operate with minimal waste or excess.}
                \SY{(1) slim down; trim down; cut\\(2) streamline; downsize}
                \VI{(1) \textit{giảm mỡ, săn chắc}.\\(2) (kinh doanh) \textit{tinh gọn}.}
                \EX{(1) Many people diet and lift weights to \textbf{get lean}.}
                \EX{(2) The firm got lean to survive the recession.}
                \CO{get/stay lean; lean physique; lean operations}
            \end{ExplainCard}

            \begin{ExplainCard}{hit the gym}{phrase}
                \EN{to go to a gym to exercise.}
                \SY{go work out; train at the gym}
                \VI{\textit{đi tập gym}.}
                \EX{After work I usually \textbf{hit the gym}.}
                \EX{She hits the gym three times a week.}
                \CO{regularly/often/suddenly hit the gym}
            \end{ExplainCard}

            \begin{ExplainCard}{work out}{phr.v}
                \EN{(1) to exercise, especially at a gym.\\(2) to develop successfully or find a solution.}
                \SY{(1) train; exercise\\(2) pan out; resolve}
                \VI{(1) \textit{tập luyện}.\\(2) \textit{diễn tiến tốt}/\textit{giải được}.}
                \EX{(1) He \textbf{works out} every morning.}
                \EX{(2) We finally worked out a fair schedule.}
                \CO{work out at the gym; work out a plan/solution; it works out}
            \end{ExplainCard}

            \begin{ExplainCard}{a wide range (of)}{phrase}
                \EN{a large and varied selection or spectrum.}
                \SY{a broad array of; a diverse set of}
                \VI{\textit{nhiều loại}, \textit{đa dạng}.}
                \EX{The center offers \textbf{a wide range of} classes.}
                \EX{Our tool supports a wide range of formats.}
                \CO{a wide range of options/services/products}
            \end{ExplainCard}

            \begin{ExplainCard}{leisurely stroll}{n}
                \EN{a slow, relaxed walk for enjoyment rather than exercise.}
                \SY{amble; saunter; unhurried walk}
                \VI{\textit{đi dạo thong thả}.}
                \EX{The elderly enjoy \textbf{leisurely strolls} in the park.}
                \EX{We took a leisurely stroll along the river.}
                \CO{take/enjoy/go for a leisurely stroll}
            \end{ExplainCard}

            \begin{ExplainCard}{take delight in}{phrase}
                \EN{to enjoy something very much, sometimes with a sense of savoring it.}
                \SY{relish; take pleasure in; enjoy}
                \VI{\textit{thích thú (với)}.}
                \EX{She \textbf{takes delight in} cooking for friends.}
                \EX{Children take delight in simple games.}
                \CO{take delight in + V\textsubscript{ing}/noun}
            \end{ExplainCard}

            \begin{ExplainCard}{workout}{n}
                \EN{a session of physical exercise.}
                \SY{training session; exercise routine}
                \VI{\textit{buổi tập luyện}.}
                \EX{A 30-minute \textbf{workout} can lift your mood.}
                \EX{He keeps a log of his daily workouts.}
                \CO{do/have a workout; intense/quick/full-body workout}
            \end{ExplainCard}

            \begin{ExplainCard}{do wonders (for)}{idiom}
                \EN{to have a very good effect on someone or something.}
                \SY{work miracles; benefit greatly; transform}
                \VI{\textit{mang lại hiệu quả tuyệt vời (cho)}.}
                \EX{Regular exercise \textbf{does wonders} for mental health.}
                \EX{Fresh paint did wonders for the room.}
                \CO{do wonders for/with + noun}
            \end{ExplainCard}

            \begin{ExplainCard}{strengthen}{v}
                \EN{to make or become stronger, more effective, or more resilient.}
                \SY{fortify; reinforce; bolster}
                \VI{\textit{tăng cường}, \textit{củng cố}.}
                \EX{Aerobic exercise \textbf{strengthens} the heart.}
                \EX{They strengthened policies to protect data.}
                \CO{strengthen muscles/immune system/policy/relationship}
            \end{ExplainCard}

            \begin{ExplainCard}{muscle}{n}
                \EN{(1) body tissue that contracts to produce movement.\\(2) physical power or strength (figurative).}
                \SY{(1) — \quad (2) might; strength}
                \VI{(1) \textit{cơ bắp}.\\(2) \textit{sức mạnh}.}
                \EX{(1) Resistance training builds \textbf{muscle}.}
                \EX{(2) The firm used its financial muscle to expand.}
                \CO{build/lose/gain muscle; muscle mass; muscle tone}
            \end{ExplainCard}

            \begin{ExplainCard}{digestive}{adj}
                \EN{relating to the process of breaking down food in the body.}
                \SY{gastrointestinal; alimentary}
                \VI{\textit{(thuộc) tiêu hoá}.}
                \EX{A healthy \textbf{digestive} system needs fiber.}
                \EX{Digestive enzymes aid nutrient absorption.}
                \CO{digestive system/tract/enzymes/problems}
            \end{ExplainCard}

            \begin{ExplainCard}{cardiovascular}{adj}
                \EN{relating to the heart and blood vessels.}
                \SY{cardiac; circulatory}
                \VI{\textit{(thuộc) tim mạch}.}
                \EX{\textbf{Cardiovascular} exercise strengthens the heart.}
                \EX{Smoking increases cardiovascular risk.}
                \CO{cardiovascular system/disease/health/fitness}
            \end{ExplainCard}

            \begin{ExplainCard}{relieve}{v}
                \EN{to reduce pain, stress, or difficulty; to make a problem less serious.}
                \SY{alleviate; ease; mitigate}
                \VI{\textit{làm dịu/bớt}, \textit{giảm nhẹ}.}
                \EX{Stretching \textbf{relieves} muscle tension.}
                \EX{The policy relieved pressure on hospitals.}
                \CO{relieve stress/pain/symptoms/pressure}
            \end{ExplainCard}

            \begin{ExplainCard}{oversimplified}{adj}
                \EN{described in a way that ignores important details or complexities.}
                \SY{simplistic; reductive}
                \VI{\textit{đơn giản hoá quá mức}.}
                \EX{Calling all modern lifestyles unhealthy is \textbf{oversimplified}.}
                \EX{An oversimplified model can mislead decision-makers.}
                \CO{oversimplified view/assumption/explanation}
            \end{ExplainCard}

            \begin{ExplainCard}{lapse into a sedentary lifestyle}{phrase}
                \EN{to gradually fall back into a pattern of little or no physical activity.}
                \SY{slip into inactivity; become desk-bound}
                \VI{\textit{sa vào lối sống ít vận động}.}
                \EX{Office workers may \textbf{lapse into a sedentary lifestyle}.}
                \EX{During exams, many students lapse into sedentary habits.}
                \CO{lapse into + habit/state; sedentary lifestyle/habits}
            \end{ExplainCard}

            \begin{ExplainCard}{contend}{v}
                \EN{(1) to argue or maintain that something is true.\\(2) to compete or struggle against difficulties or rivals.}
                \SY{(1) assert; maintain; claim\\(2) compete; vie; struggle}
                \VI{(1) \textit{cho rằng, lập luận}.\\(2) \textit{tranh đấu}, \textit{đối phó}.}
                \EX{(1) Experts \textbf{contend} that inactivity harms health.}
                \EX{(2) Several teams are contending for the title.}
                \CO{contend that + clause; contend with problems; contend for a prize}
            \end{ExplainCard}

            \begin{ExplainCard}{as a last resort}{phrase}
                \EN{only when all other methods have failed.}
                \SY{as a fallback; in extremis; as a final option}
                \VI{\textit{như biện pháp cuối cùng}.}
                \EX{People may choose fast food \textbf{as a last resort}.}
                \EX{Use antibiotics as a last resort to prevent resistance.}
                \CO{use/turn to X as a last resort}
            \end{ExplainCard}

            \begin{ExplainCard}{the fast pace of life}{phrase}
                \EN{the quick, demanding tempo of modern living, especially in cities.}
                \SY{hectic pace; rapid tempo of life}
                \VI{\textit{nhịp sống nhanh/hối hả}.}
                \EX{\textbf{The fast pace of life} leaves little time for rest.}
                \EX{Many move to the countryside to escape the fast pace of life.}
                \CO{the fast/hectic/busy pace of (city) life}
            \end{ExplainCard}

            \begin{ExplainCard}{deprive (sb) of (sth)}{v}
                \EN{to prevent someone from having something they need or want.}
                \SY{deny; strip of; rob of}
                \VI{\textit{tước đoạt} (của ai) (điều gì).}
                \EX{Long hours \textbf{deprive} workers of sleep.}
                \EX{Poverty can deprive children of opportunities.}
                \CO{deprive sb of rights/sleep/opportunities}
            \end{ExplainCard}

            \begin{ExplainCard}{stroke}{n}
                \EN{a medical condition in which blood supply to the brain is interrupted, causing damage.}
                \SY{cerebrovascular accident; CVA}
                \VI{\textit{đột quỵ}.}
                \EX{High blood pressure raises the risk of \textbf{stroke}.}
                \EX{He recovered well after suffering a minor stroke.}
                \CO{have/suffer/prevent stroke; stroke risk/factors}
            \end{ExplainCard}

            \begin{ExplainCard}{culprit}{n}
                \EN{(1) the person guilty of a crime or fault.\\(2) the cause of a problem.}
                \SY{(1) offender; wrongdoer\\(2) cause; source}
                \VI{(1) \textit{thủ phạm}.\\(2) \textit{nguyên nhân} gây ra vấn đề.}
                \EX{(1) Police caught the \textbf{culprit}.}
                \EX{(2) A sedentary lifestyle is a key culprit behind obesity.}
                \CO{the real/main culprit; identify the culprit}
            \end{ExplainCard}

            \begin{ExplainCard}{deteriorating}{adj}
                \EN{becoming worse in quality or condition.}
                \SY{worsening; declining; degenerating}
                \VI{\textit{(đang) suy giảm, xuống cấp}.}
                \EX{\textbf{Deteriorating} health restricts daily activities.}
                \EX{They acted to stop the deteriorating situation.}
                \CO{deteriorating health/conditions/relations}
            \end{ExplainCard}

            \begin{ExplainCard}{for the sake of}{phrase}
                \EN{for the purpose, benefit, or interest of someone or something.}
                \SY{for the benefit of; in the interest of; for}
                \VI{\textit{vì lợi ích/mục đích của}.}
                \EX{Policies are changed \textbf{for the sake of} public health.}
                \EX{He stayed for the sake of his children.}
                \CO{for the sake of + noun/V\textsubscript{ing}; for X’s sake}
            \end{ExplainCard}

            \begin{ExplainCard}{promote}{v}
                \EN{(1) to encourage or support the growth or progress of something.\\(2) to advertise a product or event.\\(3) to raise someone to a higher rank.}
                \SY{(1) foster; advance\\(2) publicize; market\\(3) elevate; upgrade}
                \VI{(1) \textit{thúc đẩy}.\\(2) \textit{quảng bá}.\\(3) \textit{thăng chức}.}
                \EX{(1) Campaigns \textbf{promote} healthy lifestyles.}
                \EX{(2) The band is promoting its new album.}
                \CO{promote health/innovation; promote a product/event; be promoted to}
            \end{ExplainCard}

            \begin{ExplainCard}{destructive}{adj}
                \EN{causing great harm or damage; undermining.}
                \SY{ruinous; harmful; detrimental}
                \VI{\textit{phá huỷ}, \textit{có hại}.}
                \EX{\textbf{Destructive} habits undermine wellbeing.}
                \EX{The storm left a destructive trail.}
                \CO{destructive habits/behavior/criticism; highly destructive}
            \end{ExplainCard}

            \begin{ExplainCard}{impede}{v}
                \EN{to slow or prevent the progress or movement of something.}
                \SY{hinder; hamper; obstruct}
                \VI{\textit{cản trở}, \textit{gây trở ngại}.}
                \EX{Illness can \textbf{impede} career opportunities.}
                \EX{Heavy traffic impeded emergency services.}
                \CO{impede progress/growth/recovery; factors that impede}
            \end{ExplainCard}

            \begin{ExplainCard}{a whole bunch (of)}{phrase}
                \EN{a large number or quantity of; informal.}
                \SY{a lot of; loads of; a slew of}
                \VI{\textit{rất nhiều} (khẩu ngữ).}
                \EX{There are \textbf{a whole bunch of} measures to try.}
                \EX{We received a whole bunch of emails today.}
                \CO{a whole bunch of + plural noun}
            \end{ExplainCard}

            \begin{ExplainCard}{combat}{v}
                \EN{to take action to reduce, stop, or fight something harmful.}
                \SY{fight; counter; tackle}
                \VI{\textit{chống lại}, \textit{đấu tranh với}.}
                \EX{Policies aim to \textbf{combat} lifestyle diseases.}
                \EX{We must combat misinformation online.}
                \CO{combat disease/poverty/crime; measures to combat X}
            \end{ExplainCard}

            \begin{ExplainCard}{heighten}{v}
                \EN{to increase the degree or intensity of something.}
                \SY{intensify; amplify; raise}
                \VI{\textit{nâng cao}, \textit{gia tăng}.}
                \EX{Campaigns can \textbf{heighten} public awareness.}
                \EX{The news heightened investors’ anxiety.}
                \CO{heighten awareness/tension/security/risk}
            \end{ExplainCard}

            \begin{ExplainCard}{equip (sb) with (sth)}{v}
                \EN{to supply someone with the tools, skills, or knowledge needed.}
                \SY{furnish; provide; arm; prepare}
                \VI{\textit{trang bị} (cho ai) (cái gì).}
                \EX{Courses \textbf{equip} students with practical skills.}
                \EX{He was equipped with the latest safety gear.}
                \CO{equip sb with skills/tools/knowledge; be well-equipped}
            \end{ExplainCard}

            \begin{ExplainCard}{comprehensive}{adj}
                \EN{complete and covering all or nearly all elements or aspects.}
                \SY{exhaustive; thorough; all-inclusive}
                \VI{\textit{toàn diện}.}
                \EX{We need \textbf{comprehensive} health education.}
                \EX{The report offers a comprehensive analysis.}
                \CO{comprehensive plan/coverage/overview/approach}
            \end{ExplainCard}
        \end{VocabExplain}

    \begin{VocabHighlights}
        \VH{All things considered}{(phrase) having weighed all aspects of a situation}{(cụm từ) sau khi cân nhắc mọi khía cạnh}
        \VH{to distort}{(v) change the shape, appearance or sound of something so that it is strange or not clear}{(động từ) bóp méo}
        \VH{music to one’s ears}{(idiom) something that is pleasing to hear}{(thành ngữ) thứ dễ nghe}
        \VH{to be enthused}{(p2) to be excited}{(phân từ 2) phấn khích}
        \VH{transition}{(n) the process or a period of changing from one state or condition to another}{(danh từ) sự chuyển trạng thái}
        \VH{to take up}{(phr. v) to pursue, start a hobby}{(cụm động từ) bắt đầu theo đuổi 1 sở thích}
        \VH{to kill time}{(phrase) to pass time aimlessly}{(cụm từ) giết thời gian}
        \VH{to be pressed for}{(phrase) not having enough of something, especially time or money}{(cụm từ) không đủ (thời gian, tiền bạc...)}
        \VH{forte}{(n) a thing that somebody does particularly well}{(danh từ) thế mạnh}
        \VH{to do wonders for}{(idiom) to have a very good effect on somebody/something}{(thành ngữ) có tác dụng tốt lên}
        \VH{to be given a standing ovation}{(idiom) the audience stands up to clap in order to show its admiration or support}{(thành ngữ) được khán giả đứng dậy vỗ tay hoan hô nhiệt liệt}
        \VH{to be an uphill task}{(idiom) something that is difficult and takes a lot of effort over a long period of time}{(thành ngữ) 1 việc rất khó khăn}
        \VH{novice}{(n) a person who is new and has little experience in a skill, job or situation}{(danh từ) người mới học việc}
        \VH{to go off something}{(phr.v) to stop liking or being interested in}{(cụm động từ) hết hứng}
        \VH{to spring to one’s mind}{(idiom) to happen in one’s mind}{(thành ngữ) nảy ra trong đầu}
        \VH{by chance}{(idiom) unexpectedly}{(thành ngữ) bất chợt, tình cờ}
        \VH{life expectancy}{(phrase) life span}{(cụm từ) tuổi thọ}
        \VH{to carry out}{(phr.v) do, perform}{(cụm động từ) thực hiện}
        \VH{to come as a surprise}{(phrase) something surprising}{(cụm từ) điều bất ngờ}
        \VH{to be rich in}{(phrase) be full of, be abundant in}{(cụm từ) giàu}
        \VH{to combat}{(v) prevent}{(động từ) ngăn ngừa}
        \VH{antioxidant}{(n) a substance that slows down the rate at which something decays because of oxidization}{(danh từ) chất chống oxy hóa}
        \VH{handsome}{(adj) much}{(tính từ) nhiều}
        \VH{to fortify}{(v) strengthen, make something stronger}{(động từ) củng cố}
        \VH{an abundance of}{(phrase) a large quantity that is more than enough}{(cụm từ) dồi dào}
        \VH{to get lean}{(v) to become healthy and fit}{(động từ) khỏe mạnh}
        \VH{to hit the gym}{(phrase) go to the gym}{(cụm từ) đến phòng tập}
        \VH{to work out}{(v) to train the body by physical exercise}{(động từ) luyện tập thể thao, tập thể dục}
        \VH{leisurely strolls}{(phrase) to walk slowly}{(cụm từ) dạo bộ thong thả}
        \VH{muscles}{(n) a piece of body tissue that you contract and relax in order to move a particular part of the body}{(danh từ) cơ bắp}
        \VH{digestive}{(adj) connected with the digestion of food}{(tính từ) thuộc tiêu hóa}
        \VH{cardiovascular}{(adj) connected with the heart and the blood vessels}{(tính từ) thuộc tim mạch}
        \VH{to relieve}{(v) to remove or reduce an unpleasant feeling or pain}{(động từ) làm nhẹ đi; giảm đi (cảm xúc; cảm giác)}
        \VH{oversimplified}{(adj) to describe a situation... in a way that is too simple and ignores some of the facts}{(tính từ) bị bỏ qua, coi thường}
        \VH{to lapse into}{(phr.v) to change to a less active state}{(cụm động từ) chuyển sang trạng thái trì trệ hơn}
        \VH{a sedentary lifestyle}{(n) a lifestyle that lacks physical activities and you spend a lot of time sitting down}{(danh từ) lối sống lười vận động}
        \VH{to contend}{(v) to say that something is true, especially in an argument}{(động từ) cho rằng; dám chắc rằng}
        \VH{the fast pace of life}{(phrase) a life of full actions and constant activity}{(cụm từ) nhịp sống hối hả}
        \VH{to deprive}{(v) to take something, especially something necessary or pleasant, away from someone}{(động từ) tước đi; lấy đi}
        \VH{stroke}{(n) a sudden serious illness when a blood vessel in the brain bursts or is blocked}{(danh từ) cơn đột quỵ}
        \VH{culprit}{(n) a fact or situation that is the reason for something bad happening}{(danh từ) thủ phạm; nguyên nhân}
        \VH{deteriorating}{(adj) to become worse}{(tính từ) trầm trọng thêm; làm tình trạng xấu đi}
        \VH{obligations}{(n) the state of being forced to do something because it is your duty}{(danh từ) nghĩa vụ}
        \VH{the sake of}{(n) in order to get or keep something}{(danh từ) vì lợi ích của ai}
        \VH{destructive}{(adj) causing destruction or damage}{(tính từ) có hại, có tính phá hủy}
        \VH{to impede}{(v) to delay or stop the progress of something}{(động từ) cản trở}
        \VH{a whole bunch of}{(phrase) a great deal of something}{(cụm từ) rất nhiều}
        \VH{to combat}{(v) to deal with something}{(động từ) giải quyết}
        \VH{to heighten}{(v) if a feeling or an effect heightens... it becomes stronger or increases}{(động từ) tăng cao; nâng cao lên}
        \VH{to equip}{(v) to prepare somebody for an activity or task}{(động từ) trang bị}
        \VH{comprehensive}{(adj) including all, or almost all, the items, details, facts, information, etc.}{(tính từ) bao quát, toàn diện}
    \end{VocabHighlights}
    \end{test}

    \begin{test}{TEST 3}
    \noindent
    \textbf{Part 1. Traffic where you live}
    \begin{qa}{How do most people travel to work where you live?}
    Like other cities in Vietnam, in Hanoi, the primary means of transportation is motorbike. Unlike cars which are rather \textbf{bulky} and easily stuck in \textbf{jam-packed} streets, motorbikes offer their riders the advantages of being mobile enough to \textbf{thread their way} through \textbf{bumper-to-bumper} streets. Moreover, it is faster and requires fewer \textbf{strenuous efforts} compared to riding bicycles. That is \textbf{accountable} for the \textbf{prevalence} of motorbikes in Hanoi's streets.
    \end{qa}

    \begin{qa}{What traffic problems are there in your area? (Why?)}
    Like I said above, traffic jam is the \textbf{thorniest} yet \textbf{unresolved} issue in Hanoi. \textbf{Gridlock} frequently happens during peak hours when people are generally \textbf{in a rush} to go back home after work. Sometimes \textbf{mobbed} streets occur in the case of any collisions or accidents in the streets. The traffic is constantly \textbf{crawling}, which irritates me a lot.
    \end{qa}

    \begin{qa}{How do traffic problems affect you?}
    It has serious effects on me. I am not familiar with using a face mask although I might \textbf{run the risk} of inhaling contaminated air emitted by numerous cars stuck in \textbf{the main thoroughfare}. I \textbf{dread} being contracted with respiratory diseases such as \textbf{bronchitis} or lung cancer later on. Another problem that may arise during heavy traffic is the \textbf{proneness} to arriving at one's desired destination later than expected. I occasionally arrive at work late, which might be bad for my reputation.
    \end{qa}

    \begin{qa}{How would you reduce the traffic problems in your area?}
    Traffic problems might \textbf{surface} anytime so I can come up with some solutions to avoid it. Firstly, when driving a car, I frequently check the live channel broadcast by VOV, Voice of Vietnam, to know which route is stuck to avoid. Secondly, more modes of public transport such as the tram or the tube should be \textbf{put into effect} to \textbf{lessen} the burden placed on passengers. I believe that the more forms of public transportation is encouraged, the less likely traffic congestion occurs.
    \end{qa}

        \begin{VocabExplain}[Part 1]
            \begin{ExplainCard}{bulky}{adj}
                \EN{large and awkward to carry or move; taking up a lot of space.}
                \SY{cumbersome; unwieldy; hefty}
                \VI{\textit{cồng kềnh}, khó mang vác.}
                \EX{Bulky cars often clog narrow streets.}
                \EX{This package is too bulky to strap on a scooter.}
                \CO{bulky luggage/package/furniture; bulky coat}
            \end{ExplainCard}

            \begin{ExplainCard}{jam-packed (with)}{adj}
                \EN{extremely full or crowded.}
                \SY{crammed; packed; heaving}
                \VI{\textit{đông nghịt}; chật kín.}
                \EX{The streets were jam-packed at rush hour.}
                \EX{The stadium was jam-packed with fans.}
                \CO{jam-packed streets/trains; be jam-packed with}
            \end{ExplainCard}

            \begin{ExplainCard}{thread one’s way (through)}{phr.v}
                \EN{to move carefully through a tight or obstructed space.}
                \SY{weave through; wend one’s way; edge through}
                \VI{\textit{len lỏi}/\textit{lách} qua.}
                \EX{Motorbikes thread their way through traffic.}
                \EX{Tourists threaded their way through the market stalls.}
                \CO{thread one’s way through traffic/crowds}
            \end{ExplainCard}

            \begin{ExplainCard}{bumper-to-bumper}{adj/adv}
                \EN{(of traffic) with vehicles very close together, moving slowly or not at all.}
                \SY{nose-to-tail; at a standstill}
                \VI{\textit{nối đuôi sát nhau} (kẹt xe).}
                \EX{We crawled along bumper-to-bumper for miles.}
                \EX{Bumper-to-bumper traffic is common on the ring road.}
                \CO{bumper-to-bumper traffic/queues; crawl/drive bumper-to-bumper}
            \end{ExplainCard}

            \begin{ExplainCard}{strenuous}{adj}
                \EN{requiring great effort or energy; physically demanding.}
                \SY{arduous; taxing; demanding}
                \VI{\textit{vất vả}, tốn sức.}
                \EX{Cycling uphill is a strenuous workout.}
                \EX{He made strenuous efforts to finish on time.}
                \CO{strenuous effort/exercise/activity}
            \end{ExplainCard}

            \begin{ExplainCard}{accountable (for)}{adj}
                \EN{responsible and expected to explain or justify actions.}
                \SY{responsible; answerable; liable}
                \VI{\textit{chịu trách nhiệm (về)}.}
                \EX{Officials are accountable for road safety.}
                \EX{Hold drivers accountable for violations.}
                \CO{be/hold sb accountable for; accountability}
            \end{ExplainCard}

            \begin{ExplainCard}{prevalence (of)}{n}
                \EN{the fact of being widespread or common; rate at which something occurs.}
                \SY{pervasiveness; ubiquity; frequency}
                \VI{\textit{mức độ phổ biến}; tần suất hiện hành.}
                \EX{The prevalence of motorbikes is high in Hanoi.}
                \EX{Researchers track disease prevalence by region.}
                \CO{high/low prevalence; prevalence of X}
            \end{ExplainCard}

            \begin{ExplainCard}{thorniest}{adj}
                \EN{most difficult and sensitive to deal with (superlative of \textit{thorny}).}
                \SY{trickiest; knotty; vexed}
                \VI{\textit{nan giải nhất}; gai góc nhất.}
                \EX{Congestion is the thorniest urban issue here.}
                \EX{They postponed the thorniest questions to last.}
                \CO{thorny/thorniest problem/issue/question}
            \end{ExplainCard}

            \begin{ExplainCard}{unresolved}{adj}
                \EN{not settled or solved.}
                \SY{pending; outstanding; unanswered}
                \VI{\textit{chưa được giải quyết}.}
                \EX{The parking dispute remains unresolved.}
                \EX{Several unresolved complaints fuel frustration.}
                \CO{remain unresolved; unresolved issue/question}
            \end{ExplainCard}

            \begin{ExplainCard}{gridlock}{n}
                \EN{(1) complete traffic congestion where movement stops.\\(2) a situation where progress is impossible (fig.).}
                \SY{(1) standstill \quad (2) stalemate; deadlock}
                \VI{(1) \textit{kẹt xe toàn diện}.\\(2) \textit{bế tắc}.}
                \EX{(1) A crash caused gridlock downtown.}
                \EX{(2) Policy gridlock delayed reforms.}
                \CO{traffic/political gridlock; be in gridlock}
            \end{ExplainCard}

            \begin{ExplainCard}{in a rush}{idiom}
                \EN{hurried; needing to do something quickly.}
                \SY{in a hurry; pressed for time}
                \VI{\textit{vội vã}.}
                \EX{Commuters in a rush packed the station.}
                \EX{I’m in a rush to meet a deadline.}
                \CO{be/feel in a rush; rush to do sth}
            \end{ExplainCard}

            \begin{ExplainCard}{mobbed}{adj}
                \EN{(1) very crowded.\\(2) surrounded by a crowd (esp. fans).}
                \SY{packed; thronged; swarmed}
                \VI{(1) \textit{đông nghịt}.\\(2) \textit{bị vây quanh}.}
                \EX{(1) The street was mobbed after the match.}
                \EX{(2) The singer was mobbed by admirers.}
                \CO{mobbed streets/shops; be mobbed by fans}
            \end{ExplainCard}

            \begin{ExplainCard}{crawl}{v}
                \EN{(1) to move very slowly (of traffic).\\(2) to move on hands and knees.}
                \SY{(1) inch; creep \quad (2) —}
                \VI{(1) \textit{nhích từng chút}.\\(2) \textit{bò}.}
                \EX{(1) Traffic crawled along the avenue.}
                \EX{(2) The baby crawled across the floor.}
                \CO{traffic crawls; crawl along/through}
            \end{ExplainCard}

            \begin{ExplainCard}{run the risk (of)}{phrase}
                \EN{to face the possibility of something bad happening.}
                \SY{risk; be in danger of; court}
                \VI{\textit{đối mặt/liều với nguy cơ}.}
                \EX{Cyclists run the risk of inhaling fumes.}
                \EX{Investors run the risk of losing capital.}
                \CO{run the risk of + V\textsubscript{ing}/noun}
            \end{ExplainCard}

            \begin{ExplainCard}{thoroughfare}{n}
                \EN{a public road or passage, especially a main route through a city.}
                \SY{main road; artery; boulevard}
                \VI{\textit{tuyến đường chính}; đường huyết mạch.}
                \EX{Cars choked the main thoroughfare at noon.}
                \EX{Our office sits on a busy thoroughfare.}
                \CO{main/busy thoroughfare; city thoroughfares}
            \end{ExplainCard}

            \begin{ExplainCard}{dread}{v/n}
                \EN{(1) \textbf{v.} to feel great fear or anxiety about something in the future.\\(2) \textbf{n.} great fear or apprehension.}
                \SY{fear; fear greatly; apprehend}
                \VI{(1) \textit{sợ hãi, lo sợ}.\\(2) \textit{nỗi sợ}.}
                \EX{(1) I dread getting stuck in rush-hour traffic.}
                \EX{(2) She felt a dread of the upcoming exam.}
                \CO{dread doing sth; live in dread of; dread the prospect}
            \end{ExplainCard}

            \begin{ExplainCard}{bronchitis}{n}
                \EN{inflammation of the bronchial tubes in the lungs.}
                \SY{—}
                \VI{\textit{viêm phế quản}.}
                \EX{Air pollution can trigger bronchitis.}
                \EX{He was treated for acute bronchitis last winter.}
                \CO{acute/chronic bronchitis; treat/develop bronchitis}
            \end{ExplainCard}

            \begin{ExplainCard}{proneness (to)}{n}
                \EN{a natural tendency or susceptibility to something, especially something undesirable.}
                \SY{tendency; susceptibility; predisposition}
                \VI{\textit{khuynh hướng}/\textit{dễ mắc} (điều gì).}
                \EX{Lack of sleep increases proneness to errors.}
                \EX{There is proneness to delay during heavy traffic.}
                \CO{proneness to injury/error/anxiety}
            \end{ExplainCard}

            \begin{ExplainCard}{surface}{v}
                \EN{(1) to appear or become known after being hidden.\\(2) to rise to the surface of water.}
                \SY{(1) emerge; arise; crop up \quad (2) surface}
                \VI{(1) \textit{trỗi lên}, xuất hiện.\\(2) \textit{nổi lên mặt nước}.}
                \EX{(1) Problems may surface during peak hours.}
                \EX{(2) The diver surfaced near the boat.}
                \CO{issues/problems surface; newly surfaced evidence}
            \end{ExplainCard}

            \begin{ExplainCard}{put into effect}{phrase}
                \EN{to implement and make operative (a plan, law, or policy).}
                \SY{implement; carry out; enforce}
                \VI{\textit{đưa vào thực thi}.}
                \EX{New bus lanes will be put into effect next month.}
                \EX{Strict emission rules were put into effect citywide.}
                \CO{put a policy/plan/regulation into effect}
            \end{ExplainCard}

            \begin{ExplainCard}{lessen}{v}
                \EN{to make or become smaller in amount, degree, or intensity.}
                \SY{reduce; diminish; mitigate}
                \VI{\textit{giảm bớt}; làm nhẹ.}
                \EX{More public transport can lessen congestion.}
                \EX{Planting trees helps lessen air pollution.}
                \CO{lessen the impact/burden/risk}
            \end{ExplainCard}
        \end{VocabExplain}

    \noindent
    \textbf{Part 2.}
    \begin{qa}{Describe a game or sport you enjoy playing. You should say:}

    \begin{itemize}
    \item What kind of game or sport it is
    \item Who you play it with
    \item Where you play it
    \item and explain why you enjoy playing it.
    \end{itemize}

    It is my fortune to be born into a family whose members are sports lovers, and to be specific, swimming \textbf{runs in the family}. My mother taught me how to swim when I was 5 years old. She used to be a swimming athlete, so teaching me how to swim properly \textbf{was a breeze}. She also perceived my potential and \textbf{ignited my passion} for swimming. At that time, swimming pools were very scarce \textbf{in my neck of the woods}, so she took me to her training center all day long in the summer. I was always eager about playing there because I could play with some friends of my mother. She taught me how to master some basic skills such as: floating, water comfort, and \textbf{strokes}. One of the greatest challenges for \textbf{novice} swimmers was breath control, which made me \textbf{in a panic} sometimes. There are so many advantages of swimming. First, it makes my body more flexible. Swimming helps my body parts to \textbf{coordinate with} each other in great harmony. Another remarkable benefit is that swimming \textbf{tones muscles} and builds strength. Thanks to swimming, I can get into shape, so I feel more confident.
    \end{qa}

        \begin{VocabExplain}[Part 2]
            \begin{ExplainCard}{run in the family}{idiom}
                \EN{(of a trait/ability/condition) to be common among several members of the same family, often due to heredity or shared environment.}
                \SY{be hereditary; be a family trait; be common in the family}
                \VI{\textit{(tính chất/khả năng/bệnh) có tính gia truyền}, phổ biến trong gia đình.}
                \EX{Musical talent \textbf{runs in our family}.}
                \EX{Studies show that hypertension often \textbf{runs in families} with similar lifestyles.}
                \CO{talent/disease/condition runs in the family; a family trait}
            \end{ExplainCard}

            \begin{ExplainCard}{be a breeze}{idiom}
                \EN{to be very easy or require little effort.}
                \SY{effortless; a cinch; child’s play}
                \VI{\textit{dễ ợt}, rất dễ.}
                \EX{With a good coach, learning to float \textbf{is a breeze}.}
                \EX{Once the pipeline is set up, deploying updates \textbf{is a breeze} compared with initial configuration.}
                \CO{be a breeze to do; make sth a breeze}
            \end{ExplainCard}

            \begin{ExplainCard}{ignite (one’s) passion (for)}{phrase}
                \EN{to arouse or stimulate strong enthusiasm or interest in something.}
                \SY{spark; kindle; fuel}
                \VI{\textit{khơi dậy/nhóm lên đam mê} (đối với điều gì).}
                \EX{Her first coach \textbf{ignited my passion for} swimming.}
                \EX{Fieldwork \textbf{ignited students’ passion for} environmental advocacy.}
                \CO{ignite/spark/kindle passion for; rekindle one’s passion}
            \end{ExplainCard}

            \begin{ExplainCard}{in my neck of the woods}{idiom}
                \EN{in my local area or region; nearby.}
                \SY{in my area; in this vicinity; around here}
                \VI{\textit{ở khu mình sống}, vùng lân cận.}
                \EX{There aren’t many swimming pools \textbf{in my neck of the woods}.}
                \EX{Community health resources are scarce \textbf{in this neck of the woods}.}
                \CO{in sb’s/this neck of the woods; around these parts}
            \end{ExplainCard}

            \begin{ExplainCard}{stroke (swimming)}{n}
                \EN{(1) a single movement of the arms/legs while swimming.\\(2) a particular style of swimming (e.g., freestyle, breaststroke).}
                \SY{(2) swimming style; technique}
                \VI{(1) \textit{nhịp/sải} khi bơi.\\(2) \textit{kiểu bơi} (tự do, ếch, bướm, ngửa).}
                \EX{I’m learning different \textbf{strokes} like freestyle and butterfly.}
                \EX{Correct \textbf{stroke} mechanics reduce shoulder injury risk in competitive swimmers.}
                \CO{freestyle/breaststroke/backstroke/butterfly; stroke rate/technique}
            \end{ExplainCard}

            \begin{ExplainCard}{novice}{n}
                \EN{a person who is new to and inexperienced in an activity.}
                \SY{beginner; newcomer; neophyte; rookie}
                \VI{\textit{người mới}, chưa có kinh nghiệm.}
                \EX{I joined the class as a \textbf{novice} and could barely float.}
                \EX{\textbf{Novice} learners benefit from scaffolded instruction and clear feedback.}
                \CO{novice swimmer/driver/teacher; complete/total novice}
            \end{ExplainCard}

            \begin{ExplainCard}{in a panic}{phrase}
                \EN{experiencing sudden, overwhelming fear or anxiety.}
                \SY{panic-stricken; frantic; alarmed}
                \VI{\textit{hoảng loạn}, luống cuống.}
                \EX{I went \textbf{in a panic} when I couldn’t catch my breath.}
                \EX{Panic responses can impair decision-making during high-stress tasks.}
                \CO{be/go in a panic; go into a panic; panic attack}
            \end{ExplainCard}

            \begin{ExplainCard}{coordinate with}{v}
                \EN{to organize activities or parts so they work together smoothly; to align actions with someone/something.}
                \SY{synchronize; align; harmonize}
                \VI{\textit{phối hợp/đồng bộ với} ai/cái gì.}
                \EX{Let’s \textbf{coordinate with} the team about pool times.}
                \EX{Neural signals \textbf{coordinate with} muscular contractions to produce efficient movement.}
                \CO{coordinate with sb/on sth; coordinate efforts/activities; motor coordination}
            \end{ExplainCard}

            \begin{ExplainCard}{tone (one’s) muscles}{v}
                \EN{to make muscles firmer and stronger, often without increasing bulk.}
                \SY{firm up; condition; strengthen}
                \VI{\textit{săn chắc cơ}, làm cơ khỏe hơn.}
                \EX{Swimming \textbf{tones muscles} all over the body.}
                \EX{Regular resistance training \textbf{tones skeletal muscles} and improves functional capacity.}
                \CO{tone muscles/abs/legs; tone up the body}
            \end{ExplainCard}
        \end{VocabExplain}

    \noindent
    \textbf{Part 3.}
    \begin{qa}{How have games changed from the time when you were a child?}
    Oh, \textbf{substantial} changes have been witnessed in game genres in recent years. The most striking development is that video games, mobile and computer inclusive, have \textbf{infiltrated into every nook and cranny} of people's lives and gradually \textbf{substitute} for traditional games. Today, \textbf{video game consoles} are gradually becoming \textbf{inseparable} from children's activities while all I had in my childhood was folk games.
    \end{qa}

    \begin{qa}{Do you think this has been a positive change? Why?}
    Personally, this tendency has both positives and negatives. On the positive note, a \textbf{thriving} gaming industry means children will have more choices based on their enjoyment. Besides, there are many educational games which can boost the \textbf{cognitive development} of children such as Angry Birds, Tetris, etc. However, it should be noted that excessive gaming or \textbf{gaming addiction} can \textbf{trigger} damage to children's health as well as their studies. Gamers may become a slave to immersive video games and this will cause \textbf{great anguish} later on.
    \end{qa}

    \begin{qa}{Why do you think children like playing games?}
    Understandably, playing games is an effective way for \textbf{youngsters} to \textbf{chill out} after school. Games have evolved from simple point-and-shoot ones to \textbf{something} that completely engages the users because they can cater to every gamer's needs. Not only do children play games for fun but they can also win monetary prizes in such e-sport games as League of Legends, Dota, FIFA, PES and so on. Basically, a work-life balance should be targeted for people to \textbf{encounter} stress and yield better results, and children are no exception.
    \end{qa}

    \begin{qa}{Do you think competitive games are good or bad for children? In what ways?}
    On the one hand, competitive games can \textbf{simulate obstacles} and difficulties that children may face up to later in life. Through these games, \textbf{critical thinking} and problem-solving skills can be developed, \textbf{contributing to} children's overall development. However, children in \textbf{formative years} are easily \textbf{prone} to violence in competitive games, so there should be \textbf{parental supervision} in the way children playing the games. Moreover, excessive gaming may create gaming disorder, which is likely to ruin personal relationships in reality.
    \end{qa}

    \begin{qa}{How can games sometimes help to unite people?}
    When playing games, especially online ones, players can have a wide circle of friends and acquaintances. People from \textbf{different walks of life} can make friends with those who they have a lot in common. These relationships \textbf{consolidate interpersonal} support among players in both games or sometimes in reality. They can cooperate as brothers-in-arms to fight a virtual battle and a victory will cement their relationships, I guess.
    \end{qa}

    \begin{qa}{Why is competition often seen as important in today's society?}
    This \textbf{notion} might be \textbf{rooted} in the fact that competition is one of \textbf{prerequisites} for development. Today, more and more breakthroughs borne out by competitions have been made to improve human experiences and the living standards in many regions. For example, Apple and Samsung have been \textbf{fighting tooth and nail} for a decade and, the real beneficiaries are end users who can have \textbf{a wide variety} of smartphones to choose from. Another ground for this is that if people lack competition, we will \textbf{fall behind} with other nations and live in \textbf{harsh} conditions.
    \end{qa}

        \begin{VocabExplain}[Part 3]
            \begin{ExplainCard}{substantial}{adj}
                \EN{large in amount or degree; significant rather than minor.}
                \SY{considerable; sizable; meaningful}
                \VI{\textit{đáng kể}, lớn.}
                \EX{There’s been a substantial rise in mobile gaming.}
                \EX{The policy produced substantial improvements in learning outcomes.}
                \CO{substantial change/increase/benefit; a substantial body of}
            \end{ExplainCard}

            \begin{ExplainCard}{infiltrate into every nook and cranny}{idiom}
                \EN{to spread into all places or aspects of something.}
                \SY{pervade; penetrate; permeate}
                \VI{\textit{xâm nhập/lan toả khắp mọi ngóc ngách}.}
                \EX{Smartphones have infiltrated into every nook and cranny of daily life.}
                \EX{Social media infrastructures now infiltrate every nook and cranny of civic discourse.}
                \CO{infiltrate into/through; every nook and cranny of + domain}
            \end{ExplainCard}

            \begin{ExplainCard}{substitute (for)}{v}
                \EN{to take the place of something; to replace.}
                \SY{replace; supplant; stand in for}
                \VI{\textit{thay thế}.}
                \EX{Many kids substitute online games for outdoor play.}
                \EX{Digital platforms increasingly substitute for traditional delivery channels.}
                \CO{substitute A for B; act/serve as a substitute for}
            \end{ExplainCard}

            \begin{ExplainCard}{video game console}{n}
                \EN{a dedicated electronic device designed for playing video games on a TV or monitor.}
                \SY{gaming console; console system}
                \VI{\textit{máy chơi game} (console).}
                \EX{He saved up to buy a new video game console.}
                \EX{Console lifecycles shape developers’ hardware optimization strategies.}
                \CO{home/handheld console; next-gen console; console exclusive}
            \end{ExplainCard}

            \begin{ExplainCard}{inseparable (from)}{adj}
                \EN{so closely connected that it is hard to separate.}
                \SY{closely linked; inextricable; indivisible}
                \VI{\textit{không thể tách rời}.}
                \EX{For many teens, phones are inseparable from free time.}
                \EX{Creativity is often inseparable from iterative experimentation.}
                \CO{become/remain inseparable from; inextricably inseparable link}
            \end{ExplainCard}

            \begin{ExplainCard}{thriving}{adj}
                \EN{growing vigorously; very successful or healthy.}
                \SY{flourishing; booming; prosperous}
                \VI{\textit{phát triển mạnh mẽ}, thịnh vượng.}
                \EX{Mobile gaming is a thriving market.}
                \EX{A thriving innovation ecosystem accelerates regional growth.}
                \CO{thriving industry/business/community; thrive on}
            \end{ExplainCard}

            \begin{ExplainCard}{cognitive development}{n}
                \EN{the growth of mental processes such as memory, reasoning, and problem-solving.}
                \SY{intellectual growth; mental development}
                \VI{\textit{sự phát triển nhận thức}.}
                \EX{Puzzle games can support children’s cognitive development.}
                \EX{Early language exposure predicts trajectories of cognitive development.}
                \CO{support/boost/track cognitive development; stages of}
            \end{ExplainCard}

            \begin{ExplainCard}{gaming addiction}{n}
                \EN{compulsive and excessive gaming that impairs health, study, or relationships.}
                \SY{problem gaming; gaming disorder; dependency}
                \VI{\textit{nghiện game}.}
                \EX{He sought help for gaming addiction.}
                \EX{Clinical criteria now recognize gaming addiction as a behavioral disorder.}
                \CO{develop/treat/overcome gaming addiction; addictive patterns}
            \end{ExplainCard}

            \begin{ExplainCard}{trigger}{v}
                \EN{(1) to cause something to start happening.\\(2) to set off a device or mechanism.}
                \SY{(1) spark; provoke; induce \quad (2) activate; initiate}
                \VI{(1) \textit{kích hoạt/gây ra}.\\(2) \textit{kích nổ/kích hoạt} (thiết bị).}
                \EX{(1) Late-night gaming can trigger sleep problems.}
                \EX{(2) The sensor triggers an alert when motion is detected.}
                \CO{trigger a reaction/response/symptoms; trigger mechanism}
            \end{ExplainCard}

            \begin{ExplainCard}{great anguish}{n}
                \EN{severe mental or physical pain; deep distress.}
                \SY{agony; torment; anguish}
                \VI{\textit{đau khổ tột độ}; dằn vặt.}
                \EX{Failing the exam caused him great anguish.}
                \EX{Prolonged social isolation is associated with great psychological anguish.}
                \CO{cause/feel/suffer great anguish; anguish over}
            \end{ExplainCard}

            \begin{ExplainCard}{youngsters}{n}
                \EN{children or young people.}
                \SY{youths; adolescents; kids}
                \VI{\textit{thiếu niên}, người trẻ.}
                \EX{Many youngsters unwind with casual games.}
                \EX{Programs targeting youngsters can reduce risky behaviors.}
                \CO{urban/rural youngsters; support services for youngsters}
            \end{ExplainCard}

            \begin{ExplainCard}{chill out}{phr.v}
                \EN{to relax and stop being anxious or excited.}
                \SY{unwind; relax; decompress}
                \VI{\textit{thư giãn}, bình tĩnh lại.}
                \EX{I chill out with a puzzle game after class.}
                \EX{Mindfulness exercises help students chill out before exams.}
                \CO{chill out at home/with friends; time to chill out}
            \end{ExplainCard}

            \begin{ExplainCard}{something (that …)}{pron}
                \EN{an unspecified thing, used when the exact item/idea is not named.}
                \SY{thing; item; matter}
                \VI{\textit{điều/việc nào đó} (không xác định).}
                \EX{I need something to play on the train.}
                \EX{The intervention offers something that traditional lectures lack—interactivity.}
                \CO{something to do/eat/say; something that + clause}
            \end{ExplainCard}

            \begin{ExplainCard}{encounter (stress/problems)}{v}
                \EN{to experience or face (especially unexpectedly).}
                \SY{face; experience; come up against}
                \VI{\textit{gặp phải}/\textit{trải qua} (căng thẳng/vấn đề).}
                \EX{Students often encounter stress near finals.}
                \EX{New ventures routinely encounter regulatory hurdles.}
                \CO{encounter difficulties/barriers/conflict; encounter X during}
            \end{ExplainCard}

            \begin{ExplainCard}{simulate obstacles}{phrase}
                \EN{to imitate challenges so that users can practice dealing with them.}
                \SY{model; replicate; emulate}
                \VI{\textit{mô phỏng các chướng ngại/khó khăn}.}
                \EX{Training games simulate obstacles you’ll face at work.}
                \EX{VR platforms simulate obstacles to assess decision-making under pressure.}
                \CO{simulate scenarios/obstacles/conditions; simulation of}
            \end{ExplainCard}

            \begin{ExplainCard}{critical thinking}{n}
                \EN{the ability to analyze information objectively and make reasoned judgments.}
                \SY{analytical reasoning; evaluative thinking}
                \VI{\textit{tư duy phản biện}.}
                \EX{Debate clubs sharpen students’ critical thinking.}
                \EX{Critical thinking predicts performance on complex, ill-structured tasks.}
                \CO{develop/foster/apply critical thinking; critical-thinking skills}
            \end{ExplainCard}

            \begin{ExplainCard}{contributing to}{phrase}
                \EN{helping to cause or bring about a result; playing a part in.}
                \SY{leading to; fostering; facilitating}
                \VI{\textit{góp phần vào}, dẫn tới.}
                \EX{Screen time is contributing to poor sleep.}
                \EX{Urban design is contributing to measurable gains in walkability.}
                \CO{contributing to growth/decline/outcomes; be a contributing factor}
            \end{ExplainCard}

            \begin{ExplainCard}{formative years}{n}
                \EN{the early period of a person’s life when character and abilities are shaped.}
                \SY{early years; developmental years}
                \VI{\textit{những năm hình thành} (tuổi thơ/thiếu niên).}
                \EX{Reading widely in her formative years helped a lot.}
                \EX{Nutrition during the formative years has long-term cognitive effects.}
                \CO{in one’s formative years; during the formative period}
            \end{ExplainCard}

            \begin{ExplainCard}{prone (to)}{adj}
                \EN{likely to suffer from or do something, usually regrettable.}
                \SY{susceptible; liable; inclined}
                \VI{\textit{dễ mắc/phải}, có xu hướng.}
                \EX{Kids are prone to staying up late for games.}
                \EX{Individuals with poor sleep are prone to attentional lapses.}
                \CO{be prone to errors/injury/violence; error-prone}
            \end{ExplainCard}

            \begin{ExplainCard}{parental supervision}{n}
                \EN{parents’ monitoring and guidance of a child’s activities.}
                \SY{parental oversight; guardians’ monitoring}
                \VI{\textit{sự giám sát của phụ huynh}.}
                \EX{Set screen-time limits under parental supervision.}
                \EX{Parental supervision correlates with lower incidences of online risk.}
                \CO{under/with parental supervision; lack of parental supervision}
            \end{ExplainCard}

            \begin{ExplainCard}{different walks of life}{idiom}
                \EN{people from many social backgrounds or occupations.}
                \SY{all backgrounds; every stratum; all sectors}
                \VI{\textit{mọi tầng lớp xã hội}.}
                \EX{Gamers from different walks of life team up online.}
                \EX{The survey sampled respondents from different walks of life across regions.}
                \CO{people from different walks of life; across all walks of life}
            \end{ExplainCard}

            \begin{ExplainCard}{consolidate (interpersonal) support/relationships}{v}
                \EN{to strengthen and make relationships or support more stable.}
                \SY{fortify; cement; reinforce}
                \VI{\textit{củng cố} sự hỗ trợ/quan hệ (giữa người với người).}
                \EX{Shared victories consolidated their friendship.}
                \EX{Regular collaboration consolidates interpersonal networks within teams.}
                \CO{consolidate relationships/networks/gains; consolidation of}
            \end{ExplainCard}

            \begin{ExplainCard}{notion}{n}
                \EN{an idea, belief, or concept.}
                \SY{idea; concept; belief}
                \VI{\textit{khái niệm}, quan niệm.}
                \EX{I disagree with the notion that games are a waste of time.}
                \EX{The notion of fairness underpins many economic models.}
                \CO{challenge/support the notion that…; abstract notions}
            \end{ExplainCard}

            \begin{ExplainCard}{be rooted in}{phr}
                \EN{to have as an origin or fundamental basis.}
                \SY{originate in; be grounded in; stem from}
                \VI{\textit{bắt nguồn từ}, dựa trên.}
                \EX{Her love of puzzles is rooted in childhood.}
                \EX{The policy is rooted in evidence from longitudinal studies.}
                \CO{deeply/rooted in tradition/history/theory}
            \end{ExplainCard}

            \begin{ExplainCard}{prerequisite}{n}
                \EN{something that must exist or be done before something else is possible.}
                \SY{requirement; precondition; foundation}
                \VI{\textit{điều kiện tiên quyết}.}
                \EX{Basic coding is a prerequisite for the course.}
                \EX{Robust data governance is a prerequisite for AI deployment.}
                \CO{a prerequisite for/of/to; meet/satisfy prerequisites}
            \end{ExplainCard}

            \begin{ExplainCard}{fight tooth and nail}{idiom}
                \EN{to fight or compete with all one’s strength and determination.}
                \SY{battle fiercely; compete aggressively; go all out}
                \VI{\textit{đấu tranh quyết liệt}, hết mình.}
                \EX{Brands fight tooth and nail for gamers’ attention.}
                \EX{Firms fought tooth and nail to capture emergent market share.}
                \CO{fight tooth and nail for/against; be fighting tooth and nail}
            \end{ExplainCard}

            \begin{ExplainCard}{a wide variety (of)}{phrase}
                \EN{a large and diverse range.}
                \SY{a broad array; a wide range; an extensive selection}
                \VI{\textit{nhiều loại đa dạng}.}
                \EX{The store sells a wide variety of headsets.}
                \EX{Users can access a wide variety of open-source datasets.}
                \CO{a wide variety/range/selection of}
            \end{ExplainCard}

            \begin{ExplainCard}{fall behind}{phr.v}
                \EN{to fail to keep up with others; to become less successful or advanced.}
                \SY{lag; trail; slip behind}
                \VI{\textit{tụt lại phía sau}.}
                \EX{Skip practice and you’ll fall behind.}
                \EX{Without investment, regions may fall behind in digital capacity.}
                \CO{fall behind on/in/with; lag far behind}
            \end{ExplainCard}

            \begin{ExplainCard}{harsh}{adj}
                \EN{severe or unpleasant; causing discomfort or difficulty.}
                \SY{severe; grim; austere}
                \VI{\textit{khắc nghiệt}; gay gắt.}
                \EX{The game’s penalties felt harsh.}
                \EX{Harsh socioeconomic conditions exacerbate educational inequality.}
                \CO{harsh conditions/criticism/penalties; harsh reality}
            \end{ExplainCard}
        \end{VocabExplain}

    \begin{VocabHighlights}
        \VH{bulky}{(adj) large and difficult to move or carry}{(tính từ) cồng kềnh}
        \VH{jam-packed}{(adj) very full or crowded}{(tính từ) đông đúc}
        \VH{to thread one's way}{(phrase) to move or make something move through a narrow space, avoiding things that are in the way}{(cụm từ) luồn lách qua}
        \VH{bumper-to-bumper}{(adj) with almost no space between one car and the next in a line of cars}{(tính từ) chật như nêm}
        \VH{strenuous}{(adj) needing great effort and energy}{(tính từ) tốn nhiều công sức}
        \VH{be accountable for}{(adj) responsible for your decisions or actions}{(tính từ) chịu trách nhiệm về}
        \VH{prevalence}{(n) the fact of existing or being very common at a particular time or in a particular place}{(danh từ) sự phổ cập}
        \VH{thorny}{(adj) causing difficulty or disagreement}{(tính từ) gai góc, khó nhằn}
        \VH{unresolved}{(adj) not yet solved or answered; not having been resolved}{(tính từ) chưa giải quyết xong}
        \VH{gridlock}{(n) a situation in which there are so many cars in the streets of a town that the traffic cannot move at all}{(danh từ) sự tắc nghẽn giao thông}
        \VH{in a rush}{(idiom) very quickly, especially more than normal}{(thành ngữ) vội vã}
        \VH{mobbed}{(adj) very crowded}{(tính từ) rất đông đúc}
        \VH{to crawl}{(v) to move forward very slowly}{(động từ) tiến lên rất chậm}
        \VH{to run the risk of}{(phrase) expose oneself to the possibility of something unpleasant occurring}{(cụm từ) để có nguy cơ}
        \VH{the main thoroughfare}{(phrase) major road where there is most traffic}{(cụm từ) đường chính nơi thường có mật độ giao thông lớn}
        \VH{to dread v-ing}{(v) to be very afraid of something; to fear that something bad is going to happen}{(động từ) sợ làm gì}
        \VH{bronchitis}{(n) an illness that affects the bronchial tubes leading to the lungs}{(danh từ) bệnh viêm phế quản}
        \VH{proneness}{(n) the quality of being likely to suffer from something or to do something bad}{(danh từ) sự dễ mắc phải}
        \VH{to surface}{(v) to suddenly appear or become obvious after having been hidden for a while}{(động từ) nảy sinh}
        \VH{to put into effect}{(idiom) to cause something to come into use}{(thành ngữ) hiện thực hóa}
        \VH{to lessen}{(v) to become or make something become smaller, weaker, less important, etc}{(động từ) làm giảm nhẹ, suy yếu đi}
        \VH{to run in the family}{(idiom) a particular movement that is usually repeated in a method of swimming}{(thành ngữ) ăn trong máu, có gen di truyền}
        \VH{to be a breeze}{(idiom) to be extremely easy}{(thành ngữ) rất dễ}
        \VH{to ignite one's passion}{(phrase) to arouse one's passion}{(cụm từ) kích thích sự khát khao}
        \VH{in somebody's neck of the woods}{(idiom) the area someone comes from, or the area where you are}{(thành ngữ) nơi mình sinh sống}
        \VH{stroke}{(n) a particular movement that is usually repeated in a method of swimming}{(danh từ) sải bơi}
        \VH{novice}{(n) a person who is inexperienced}{(danh từ) kẻ mới vào nghề}
        \VH{in a panic}{(phrase) a sudden strong feeling of fear that prevents reasonable thought and action}{(thành ngữ) hoảng loạn}
        \VH{to coordinate with}{(phrase) to make many different things work effectively as a whole}{(động từ) phối hợp}
        \VH{to tone muscles}{(phrase) make something firmer and stronger, usually by doing physical exercise}{(cụm từ) tăng cơ}
        \VH{substantial}{(adj) large in amount, value or importance}{(tính từ) nhiều; đáng kể}
        \VH{to infiltrate}{(v) to pass slowly into something}{(động từ) thâm nhập}
        \VH{substitute}{(n) a person or thing that you use or have instead of the one you normally use or have}{(danh từ) người/vật thay thế}
        \VH{every nook and cranny}{(phrase) every part or aspect of something}{(cụm từ) mọi ngóc ngách}
        \VH{video game consoles}{(phrase) a specialized desktop computer used to play video games}{(cụm từ) máy chơi game}
        \VH{inseparable}{(adj) not able to divide}{(tính từ) không thể tách rời}
        \VH{thriving}{(adj) continuing to be successful, strong, healthy, etc}{(tính từ) giàu có; thịnh vượng, rất phát triển}
        \VH{acquaintance}{(n) a person you know but who is not a close friend}{(danh từ) người quen}
        \VH{from different walks of life}{(phrase) used to refer to people who have many different jobs or positions in society}{(cụm từ) từ các tầng lớp khác nhau}
        \VH{to consolidate}{(v) to make a position of power or success stronger so that it is more likely to continue}{(động từ) củng cố}
        \VH{interpersonal}{(adj) connected with relationships between people}{(tính từ) giữa cá nhân với nhau}
        \VH{notion}{(n) an idea, a belief or an understanding of something}{(danh từ) khái niệm}
        \VH{be rooted in}{(adj) fixed in one place; not moving or changing}{(tính từ) ăn sâu; bén rễ, bắt nguồn từ}
        \VH{prerequisite}{(n) something that must exist or happen before something else can happen or be done}{(danh từ) điều kiện tiên quyết}
        \VH{to fight tooth and nail}{(idiom) engage in vigorous combat or make a strenuous effort, using all one's resources}{(thành ngữ) cạnh tranh khốc liệt}
        \VH{to fall behind}{(phrase) to fail to do something fast enough or on time}{(cụm từ) tụt lại phía sau}
        \VH{harsh}{(adj) cruel, severe and unkind}{(tính từ) khắc nghiệt}
    \end{VocabHighlights}

    \end{test}

    \begin{test}{TEST 4}
    \noindent
    \textbf{Part 1. Your Friends}
    \begin{qa}{Do you prefer to have one particular friend or a group of friends? Why?}
    It does not matter whether I have one or a group of friends. It lies in whether he, she or they can \textbf{stand up for} me in despair or not. \`\`\textbf{A friend in need is a friend indeed}''. I do not need \textbf{fair-weather friends} who can \textbf{turn their back on} me \textbf{for the sake of} their benefits or \textbf{leave me in the lurch}.
    \end{qa}

    \begin{qa}{What do you like doing most with your friend/s?}
    \textbf{Having a little chit-chat} is my favorite activity. Almost everyone needs to \textbf{go through} pressurized moments at the workplace so having a friend who is willing to \textbf{lend a sympathetic ear} is important.
    \end{qa}

    \begin{qa}{Do you think it's important to keep in contact with friends you knew as a child? (Why/Why not?)}
    Yes, definitely. Childhood friends can \textbf{jog one's memory} when when he or she was a child. It is from these memories that we can \textbf{derive} moral lessons, no matter how excellent or terrible they are. They surely \textbf{aid} us in the process of becoming a \textbf{full-fledged} adult later on.
    \end{qa}

    \begin{qa}{What makes a friend into a good friend?}
    It depends on his or her trait. He or she should be considerate towards others and be honest about everything to \textbf{gain others' trust}. \textbf{Sticking by} someone no matter what happens is also another thing worth considering.
    \end{qa}

        \begin{VocabExplain}[Part 1]
            \begin{ExplainCard}{stand up for (sb/sth)}{phr.v}
                \EN{to defend or support someone or something, especially when it is criticized or attacked.}
                \SY{defend; uphold; champion}
                \VI{\textit{đứng ra bảo vệ}/\textit{bênh vực} ai/điều gì.}
                \EX{Good friends \textbf{stand up for} you when others are unfair.}
                \EX{Employees must feel safe to \textbf{stand up for} ethical standards in the workplace.}
                \CO{stand up for yourself/sb’s rights; courage to stand up for}
            \end{ExplainCard}

            \begin{ExplainCard}{A friend in need is a friend indeed}{proverb}
                \EN{someone who helps you when you are in trouble is a true friend.}
                \SY{true friendship shows in hardship}
                \VI{\textit{hoạn nạn mới biết bạn hiền}.}
                \EX{He drove me to the hospital—\textbf{a friend in need is a friend indeed}.}
                \EX{The proverb holds in crisis-response teams, where reliability under pressure defines trust.}
                \CO{prove/show that a friend in need is a friend indeed}
            \end{ExplainCard}

            \begin{ExplainCard}{fair-weather friend}{n}
                \EN{a person who is only a friend when things are going well.}
                \SY{inconstant friend; fickle ally}
                \VI{\textit{bạn thời vụ}, chỉ ở bên khi thuận lợi.}
                \EX{When I lost my job, the \textbf{fair-weather friends} vanished.}
                \EX{Social networks can amplify \textbf{fair-weather} relationships with low commitment.}
                \CO{be/have/spot a fair-weather friend; fair-weather loyalty}
            \end{ExplainCard}

            \begin{ExplainCard}{turn one’s back on (sb/sth)}{idiom}
                \EN{to reject or abandon someone or something in time of need.}
                \SY{abandon; desert; forsake}
                \VI{\textit{quay lưng lại}, bỏ rơi.}
                \EX{Real friends don’t \textbf{turn their back on} you.}
                \EX{Some institutions \textbf{turned their backs on} long-term partners during restructuring.}
                \CO{turn one’s back on obligations/friends; never turn your back on}
            \end{ExplainCard}

            \begin{ExplainCard}{for the sake of}{phrase}
                \EN{for the benefit, interest, or purpose of someone or something.}
                \SY{for the benefit of; in the interest of; for}
                \VI{\textit{vì lợi ích/mục đích của}.}
                \EX{They stayed calm \textbf{for the sake of} the children.}
                \EX{Policies were revised \textbf{for the sake of} transparency and public trust.}
                \CO{for the sake of + noun/V\textsubscript{ing}; for X’s sake}
            \end{ExplainCard}

            \begin{ExplainCard}{leave (sb) in the lurch}{idiom}
                \EN{to abandon someone who needs help or support.}
                \SY{desert; strand; forsake}
                \VI{\textit{bỏ mặc ai lúc khó khăn}.}
                \EX{He \textbf{left me in the lurch} when the project got tough.}
                \EX{Last-minute supplier failures can \textbf{leave} small firms \textbf{in the lurch}.}
                \CO{leave sb in the lurch; be left in the lurch}
            \end{ExplainCard}

            \begin{ExplainCard}{(have) a little chit-chat}{phrase}
                \EN{a brief, informal conversation about light topics.}
                \SY{small talk; chat; banter}
                \VI{\textit{tán gẫu}, chuyện phiếm ngắn.}
                \EX{We had \textbf{a little chit-chat} over coffee.}
                \EX{Pre-meeting \textbf{chit-chat} can strengthen team cohesion.}
                \CO{have/enjoy a chit-chat; casual chit-chat with}
            \end{ExplainCard}

            \begin{ExplainCard}{go through}{phr.v}
                \EN{(1) to experience or endure something.\\(2) to examine something carefully.}
                \SY{(1) undergo; endure \quad (2) review; scrutinize}
                \VI{(1) \textit{trải qua/chịu đựng}.\\(2) \textit{xem xét kỹ}.}
                \EX{Everyone \textbf{goes through} stressful times.}
                \EX{Auditors \textbf{went through} the records line by line.}
                \CO{go through hardship/a phase; go through files/documents}
            \end{ExplainCard}

            \begin{ExplainCard}{lend a sympathetic ear (to sb)}{idiom}
                \EN{to listen to someone with empathy and understanding.}
                \SY{listen compassionately; be a good listener}
                \VI{\textit{lắng nghe cảm thông}.}
                \EX{Thanks for \textbf{lending me a sympathetic ear}.}
                \EX{Mentorship programs \textbf{lend a sympathetic ear} to early-career staff.}
                \CO{lend/offer a sympathetic ear to; need a sympathetic ear}
            \end{ExplainCard}

            \begin{ExplainCard}{jog one’s memory}{idiom}
                \EN{to cause someone to remember something.}
                \SY{prompt; refresh; cue}
                \VI{\textit{kích gợi trí nhớ}.}
                \EX{Old photos \textbf{jogged my memory} of school days.}
                \EX{Contextual cues can \textbf{jog memory} in retrieval experiments.}
                \CO{jog sb’s memory about; visual cues jog memory}
            \end{ExplainCard}

            \begin{ExplainCard}{derive (sth) from (sth)}{v}
                \EN{(1) to obtain something from a source.\\(2) to infer logically from given facts.}
                \SY{(1) obtain; draw; extract \quad (2) deduce; infer}
                \VI{(1) \textit{rút ra}/\textit{nhận được} từ.\\(2) \textit{suy ra}.}
                \EX{She \textbf{derived} comfort from friends’ support.}
                \EX{We \textbf{derive} principles from empirical observations.}
                \CO{derive benefits/insights from; derive A from B}
            \end{ExplainCard}

            \begin{ExplainCard}{aid}{v}
                \EN{to help or assist someone or something to achieve a goal.}
                \SY{assist; facilitate; support}
                \VI{\textit{hỗ trợ}, giúp đỡ.}
                \EX{Notes can \textbf{aid} your memory.}
                \EX{Targeted feedback \textbf{aids} skill acquisition in learners.}
                \CO{aid recovery/communication/learning; aid and abet (law)}
            \end{ExplainCard}

            \begin{ExplainCard}{full-fledged}{adj}
                \EN{completely developed or qualified; having achieved full status.}
                \SY{fully developed; fully fledged; mature}
                \VI{\textit{trưởng thành/đúng nghĩa}; \textit{đủ tư cách}.}
                \EX{He became a \textbf{full-fledged} member of the team.}
                \EX{The startup evolved into a \textbf{full-fledged} enterprise within two years.}
                \CO{full-fledged adult/member/professional; become/turn into}
            \end{ExplainCard}

            \begin{ExplainCard}{gain (someone’s) trust}{phrase}
                \EN{to earn another person’s confidence through reliable and honest behavior.}
                \SY{earn confidence; win trust; build credibility}
                \VI{\textit{giành được niềm tin} của ai.}
                \EX{Be consistent if you want to \textbf{gain her trust}.}
                \EX{Transparent communication helps organizations \textbf{gain public trust}.}
                \CO{gain/win/build/maintain trust; trust-building measures}
            \end{ExplainCard}

            \begin{ExplainCard}{stick by (sb)}{phr.v}
                \EN{to remain loyal to someone, especially in difficulty.}
                \SY{stand by; back; stay loyal to}
                \VI{\textit{luôn ở bên}/\textit{trung thành với} ai.}
                \EX{True friends \textbf{stick by} you through hard times.}
                \EX{Leaders who \textbf{stick by} their teams foster resilience during crises.}
                \CO{stick by a friend/decision/promise; continue to stick by}
            \end{ExplainCard}
        \end{VocabExplain}

    \noindent
    \textbf{Part 2.}

    \begin{qa}{Describe an important choice you had to make in your life. You should say:}

    \begin{itemize}
    \item When you had to make this choice
    \item What you had to choose between
    \item Whether you made a good choice
    \item and explain how you felt when you were making this choice.
    \end{itemize}

    To be honest, I'm not good at making decisions. I am scared of bearing \textbf{undesirable} consequences. Some decisions are easy to make, and some aren't. One of the most important decisions in my life is where I should work after \textbf{coming fresh out of university}. Two years ago, having obtained my Bachelor's degree from Foreign Trade University, I had to look for a job in the labor market. At that time, winning a job was \textbf{a tall order} because there was \textbf{an economic recession}. The plight of the economy led a great number of companies to \textbf{go belly-up}. Luckily, I received two offer letters from two companies. In particular, a multinational company offered me the position as a financial consultant, while the other was a start-up that required me to \textbf{undertake more responsibilities}. \textbf{The crux of the matter} was that if I worked for a multinational company, I would have to \textbf{relocate} to Ho Chi Minh for a year, but the salary was rewarding enough for me to \textbf{sit pretty}. It was not the case for the start-up. So, I was \textbf{in two minds about} my destination. I thought a lot about the \textbf{pros and cons} and I also tried \textbf{picking some friends and relatives' brains}. To be honest, it was \textbf{touch and go} which my final decision would be like because this decision would be a \textbf{turning point} in my life. \textbf{On second thought}, I chose to work for a multinational company because of \textbf{promotion opportunities} and \textbf{further study}. Although my family \textbf{overwhelmingly supports me}, I \textbf{had cold feet} about it a couple of times. If I look back to my life, this is one of the events that I will never forget. I will never \textbf{kick myself} for that.
    \end{qa}


        \begin{VocabExplain}[Part 2]
            \begin{ExplainCard}{undesirable}{adj}
                \EN{not wanted because it is harmful, unpleasant, or likely to cause problems.}
                \SY{unwelcome; adverse; detrimental}
                \VI{\textit{không mong muốn}; gây hại/phiền toái.}
                \EX{Working late every night has some \textbf{undesirable} effects.}
                \EX{The policy produced \textbf{undesirable} externalities in labor markets.}
                \CO{undesirable effect/outcome/side-effect; minimize/avoid sth undesirable}
            \end{ExplainCard}

            \begin{ExplainCard}{(come) fresh out of university}{phrase}
                \EN{to have just graduated from university with little or no work experience.}
                \SY{newly graduated; straight out of college}
                \VI{\textit{vừa tốt nghiệp đại học}, hầu như chưa có kinh nghiệm.}
                \EX{I was \textbf{fresh out of university} and nervous about interviews.}
                \EX{Firms often design onboarding for hires \textbf{fresh out of university}.}
                \CO{be/come fresh out of college/university; fresh graduate}
            \end{ExplainCard}

            \begin{ExplainCard}{a tall order}{n}
                \EN{a task or request that is very difficult to fulfil.}
                \SY{big ask; uphill task; formidable challenge}
                \VI{\textit{yêu cầu/việc quá khó}.}
                \EX{Landing a job in a week is \textbf{a tall order}.}
                \EX{Achieving net-zero by 2030 remains \textbf{a tall order} for most cities.}
                \CO{prove/be a tall order; quite/pretty a tall order}
            \end{ExplainCard}

            \begin{ExplainCard}{economic recession}{n}
                \EN{a period of significant decline in economic activity across the economy.}
                \SY{downturn; contraction; slump}
                \VI{\textit{suy thoái kinh tế}.}
                \EX{During an \textbf{economic recession}, hiring slows.}
                \EX{The \textbf{recession} reduced aggregate demand and investment.}
                \CO{enter/avoid/recover from a recession; deep/prolonged recession}
            \end{ExplainCard}

            \begin{ExplainCard}{go belly-up}{idiom}
                \EN{to fail completely, especially of a business.}
                \SY{go bust; go under; collapse}
                \VI{\textit{phá sản; sụp đổ}.}
                \EX{Several cafés \textbf{went belly-up} after the rent hike.}
                \EX{Under-capitalized startups tend to \textbf{go belly-up} during shocks.}
                \CO{company/venture goes belly-up; risk of going belly-up}
            \end{ExplainCard}

            \begin{ExplainCard}{undertake more responsibilities}{v}
                \EN{to accept and begin to do additional duties or tasks.}
                \SY{assume; shoulder; take on}
                \VI{\textit{đảm nhận thêm trách nhiệm}.}
                \EX{At the startup I had to \textbf{undertake more responsibilities}.}
                \EX{Managers \textbf{undertake responsibilities} for compliance and risk.}
                \CO{undertake/take on/assume responsibilities; increased responsibilities}
            \end{ExplainCard}

            \begin{ExplainCard}{the crux of the matter}{n}
                \EN{the most important or difficult point of a problem.}
                \SY{heart; core; nub}
                \VI{\textit{cốt lõi vấn đề}.}
                \EX{The \textbf{crux of the matter} was the relocation.}
                \EX{Funding mechanisms are \textbf{the crux of the matter} in policy design.}
                \CO{get to/reach the crux; the crux lies in …}
            \end{ExplainCard}

            \begin{ExplainCard}{relocate}{v}
                \EN{to move to a new place to live or work.}
                \SY{move; transfer; resettle}
                \VI{\textit{chuyển nơi ở/công tác}.}
                \EX{They asked me to \textbf{relocate} to Ho Chi Minh City.}
                \EX{Firms \textbf{relocate} operations to optimize supply chains.}
                \CO{relocate to/from; relocation package/allowance}
            \end{ExplainCard}

            \begin{ExplainCard}{sit pretty}{idiom}
                \EN{to be in a very comfortable or advantageous position, especially financially.}
                \SY{be well off; be in clover; be comfortable}
                \VI{\textit{ở vị thế thuận lợi, thoải mái} (thường về tài chính).}
                \EX{With that salary, I’d \textbf{sit pretty} for a while.}
                \EX{Cash-rich incumbents \textbf{sit pretty} during downturns.}
                \CO{sit pretty with/on + money/assets; be sitting pretty}
            \end{ExplainCard}

            \begin{ExplainCard}{be in two minds (about)}{phrase}
                \EN{to be unable to decide between two options.}
                \SY{be torn; be undecided; waver}
                \VI{\textit{phân vân}, chưa quyết.}
                \EX{I was \textbf{in two minds about} the offers.}
                \EX{Stakeholders remain \textbf{in two minds} about privatization.}
                \CO{be in two minds about/over; waver between A and B}
            \end{ExplainCard}

            \begin{ExplainCard}{pros and cons}{n(pl)}
                \EN{the advantages and disadvantages of something.}
                \SY{benefits and drawbacks; upsides and downsides}
                \VI{\textit{mặt lợi và hại}.}
                \EX{List the \textbf{pros and cons} before you decide.}
                \EX{The report weighs the \textbf{pros and cons} of remote work.}
                \CO{weigh/balance/consider the pros and cons of}
            \end{ExplainCard}

            \begin{ExplainCard}{pick sb’s brain(s)}{idiom}
                \EN{to ask someone knowledgeable for ideas or advice.}
                \SY{consult; tap into; sound out}
                \VI{\textit{hỏi ý kiến/nhờ tư vấn} người có kinh nghiệm.}
                \EX{I \textbf{picked} my mentor’s \textbf{brain} over coffee.}
                \EX{Leaders \textbf{pick the brains} of domain experts before committing.}
                \CO{pick sb’s brain(s) about/on; have your brain picked}
            \end{ExplainCard}

            \begin{ExplainCard}{touch and go}{idiom/adj}
                \EN{uncertain and risky; the outcome could go either way.}
                \SY{precarious; dicey; uncertain}
                \VI{\textit{bấp bênh}, chưa chắc chắn.}
                \EX{It was \textbf{touch and go} until I signed the offer.}
                \EX{Cash-flow projections made survival \textbf{touch-and-go}.}
                \CO{be/look touch and go; a touch-and-go situation}
            \end{ExplainCard}

            \begin{ExplainCard}{turning point}{n}
                \EN{a decisive time when an important change happens.}
                \SY{watershed; milestone; pivot}
                \VI{\textit{bước ngoặt}.}
                \EX{That job was a real \textbf{turning point} for me.}
                \EX{The study marks a \textbf{turning point} in cancer therapy.}
                \CO{a major/critical turning point in/for}
            \end{ExplainCard}

            \begin{ExplainCard}{on second thought(s)}{phrase}
                \EN{after reconsideration; having changed one’s mind.}
                \SY{upon reflection; after rethinking}
                \VI{\textit{nghĩ lại thì}/suy đi tính lại.}
                \EX{\textbf{On second thought}, I took the safer offer.}
                \EX{\textbf{Upon further thought}, the committee altered its recommendation.}
                \CO{on second thoughts I/you…; upon reflection}
            \end{ExplainCard}

            \begin{ExplainCard}{promotion opportunities}{n}
                \EN{chances to advance to a higher rank or position at work.}
                \SY{career progression; advancement prospects}
                \VI{\textit{cơ hội thăng tiến}.}
                \EX{I chose the firm with better \textbf{promotion opportunities}.}
                \EX{Clear \textbf{promotion opportunities} enhance retention and performance.}
                \CO{offer/seek/limit promotion opportunities; clear career ladder}
            \end{ExplainCard}

            \begin{ExplainCard}{further study}{n}
                \EN{continuing one’s education beyond the current level, often while working.}
                \SY{postgraduate study; continuing education}
                \VI{\textit{học tiếp}, học nâng cao.}
                \EX{The company supports employees’ \textbf{further study}.}
                \EX{Scholarships encourage \textbf{further study} in STEM disciplines.}
                \CO{pursue/finance/sponsor further study; opportunities for further study}
            \end{ExplainCard}

            \begin{ExplainCard}{overwhelmingly (support)}{adv/phrase}
                \EN{by a very large majority or degree; to support almost unanimously.}
                \SY{decisively; resoundingly; by and large}
                \VI{\textit{một cách áp đảo}; \textit{ủng hộ gần như tuyệt đối}.}
                \EX{My family \textbf{overwhelmingly support} my decision.}
                \EX{Survey respondents \textbf{overwhelmingly supported} remote-work flexibility.}
                \CO{overwhelmingly approve/support/back; overwhelming support}
            \end{ExplainCard}

            \begin{ExplainCard}{have cold feet}{idiom}
                \EN{to suddenly become nervous and hesitate about a major decision.}
                \SY{waver; get second thoughts; lose nerve}
                \VI{\textit{chùn bước}, sợ hãi vào phút chót.}
                \EX{I \textbf{had cold feet} before signing the contract.}
                \EX{Acquirers sometimes \textbf{get cold feet} as due diligence reveals risks.}
                \CO{get/have cold feet about; cold-feet moment}
            \end{ExplainCard}

            \begin{ExplainCard}{kick oneself (for)}{idiom}
                \EN{to feel annoyed with yourself for doing or not doing something.}
                \SY{regret; beat oneself up; rue}
                \VI{\textit{tự trách} mình; tiếc nuối.}
                \EX{If I’d refused, I’d \textbf{kick myself} later.}
                \EX{Investors \textbf{kicked themselves} for ignoring early signals.}
                \CO{kick yourself for + V-ing; really/only have yourself to blame}
            \end{ExplainCard}
        \end{VocabExplain}

    \noindent
    \textbf{Part 3.}
    \begin{qa}{What are the typical choices people make at different stages of their lives?}
    In \textbf{carefree childhood}, it is \textbf{as easy as falling off a log} for children to decide simple things like favorite food or clothes. But when people get on with age, they are often obsessed with their studies and work like how to get good mark or \textbf{promotion}. And, to many married couples, decisions on how to bring up their babies will be their most concern, I guess.
    \end{qa}

    \begin{qa}{Should important choices be made by parents rather than by young adults?}
    Well, it kind of depends. I mean the definition of different choices seem \textbf{vague and dissimilar} to each other. In my opinion, young adults should \textbf{have their voice} in the decision-making process as they are integral parts of their family. In the past, parents with more experience could have greater powers in drawing conclusions. In this day and age, they should grant their children the rights to be decision-makers. This will help their offspring become responsible and dependable adults \textbf{down the road}.
    \end{qa}

    \begin{qa}{Why do some people like to discuss choices with other people?}
    I am of the opinion that \textbf{a problem shared is a problem halved}, so it is \textbf{comprehensible} to many people to share their stories with others. When facing a \textbf{daunting task}, for instance, people are apt to \textbf{enlist the help} of others to look for \textbf{sound advice}, or simply empathy for their situation. Getting someone to \textbf{lend a sympathetic ear} to their problems is also a good way to reduce stress or pressure in their life.
    \end{qa}

    \begin{qa}{What kind of choices do people have to make in their everyday life?}
    There are countless choices for people to choose day after day. At home, there are choices associated with \textbf{daily necessities} such as how to prepare food or dress up. But these decisions are pretty simple in comparison with those people have to make in the workplace like how to \textbf{meet the deadlines} or how to satisfy the customers' requirements.
    \end{qa}

    \begin{qa}{Why do some people choose to do the same things every day?}
    If people choose to follow the same routine in their daily life, I guess the reason is for the convenience. These days, working adults \textbf{have so much on their plate}, from household chores to workload, so they often feel \textbf{overwhelmed} when it comes to making decisions too often. This is why many choose to lead \textbf{a minimalist lifestyle} which can \textbf{free up} some time for themselves.
    \end{qa}

    \begin{qa}{Are there any disadvantages in this?}
    It \textbf{goes without saying} that performing the same tasks every day can be \textbf{bored to death}. This lifestyle is full of \textbf{sheer} boredom and may cultivate \textbf{a sense of demotivation} for people. If this happens for a long time, people will suffer from decreased \textbf{morale} and never want to break the routine. The problem is getting more worse when their working productivity will be adversely affected as a result. Such routines somehow \textbf{dissuade} people from \textbf{acting beyond their comfort zones} to experience new things in life.
    \end{qa}

    \begin{qa}{Do you think that people today have more choices to make today than in the past?}
    Obviously, people nowadays are presented with \textbf{a myriad of choices}. For example, technological advancements in air transport have popularized international tourism to many people. Nowadays, they can choose from many alternative \textbf{holiday destinations} instead of travelling domestically. Besides, the advent of the Internet and telecommunication has \textbf{paved the way} for trade transactions, for example. Furthermore, globalization is a \textbf{precursor} of international exchange and more foreign products are now sold in domestic markets.
    \end{qa}

        \begin{VocabExplain}[Part 3]
            \begin{ExplainCard}{carefree childhood}{n phrase}
                \EN{a period of early life with few worries or responsibilities.}
                \SY{untroubled youth; easygoing early years}
                \VI{\textit{tuổi thơ vô tư, ít lo nghĩ}.}
                \EX{He spent a \textbf{carefree childhood} in the countryside.}
                \EX{Public play spaces can help restore elements of a \textbf{carefree childhood} in dense cities.}
                \CO{enjoy/recall a carefree childhood; carefree childhood years}
            \end{ExplainCard}

            \begin{ExplainCard}{as easy as falling off a log}{idiom}
                \EN{extremely easy to do.}
                \SY{a breeze; effortless; child’s play}
                \VI{\textit{dễ ợt}, vô cùng dễ.}
                \EX{For her, basic algebra is \textbf{as easy as falling off a log}.}
                \EX{With templates, deploying a small site becomes \textbf{as easy as falling off a log}.}
                \CO{be/feel as easy as falling off a log}
            \end{ExplainCard}

            \begin{ExplainCard}{promotion}{n}
                \EN{advancement to a higher rank or position at work.}
                \SY{advancement; elevation; career progression}
                \VI{\textit{thăng chức}.}
                \EX{She’s hoping for a \textbf{promotion} next quarter.}
                \EX{Transparent criteria improve access to \textbf{promotion} across departments.}
                \CO{win/seek/deny promotion; promotion to + role}
            \end{ExplainCard}

            \begin{ExplainCard}{vague}{adj}
                \EN{not clearly expressed or defined.}
                \SY{unclear; imprecise; ambiguous}
                \VI{\textit{mơ hồ}, không rõ ràng.}
                \EX{The instructions were \textbf{vague} and confusing.}
                \EX{\textbf{Vague} goals correlate with lower task completion rates.}
                \CO{vague idea/terms/answer; remain vague about}
            \end{ExplainCard}

            \begin{ExplainCard}{dissimilar}{adj}
                \EN{not alike; different in nature or form.}
                \SY{unlike; distinct; divergent}
                \VI{\textit{khác biệt}, không giống.}
                \EX{Their tastes are \textbf{dissimilar} in every way.}
                \EX{\textbf{Dissimilar} datasets require tailored preprocessing pipelines.}
                \CO{dissimilar to/from; highly/markedly dissimilar}
            \end{ExplainCard}

            \begin{ExplainCard}{have (one’s) voice (in)}{phrase}
                \EN{to be allowed to express opinions and influence decisions.}
                \SY{have a say; be heard; participate}
                \VI{\textit{có tiếng nói (trong)} việc ra quyết định.}
                \EX{Teenagers should \textbf{have their voice in} family plans.}
                \EX{Stakeholders who \textbf{have a voice in} policy design show higher buy-in.}
                \CO{have a/your voice in; give sb a voice}
            \end{ExplainCard}

            \begin{ExplainCard}{down the road}{idiom}
                \EN{at a later time; in the future.}
                \SY{later on; in the long run; eventually}
                \VI{\textit{về sau}, trong tương lai.}
                \EX{Saving now will help you \textbf{down the road}.}
                \EX{Early literacy predicts academic performance \textbf{down the road}.}
                \CO{problems/benefits down the road}
            \end{ExplainCard}

            \begin{ExplainCard}{a problem shared is a problem halved}{proverb}
                \EN{talking about a problem makes it easier to bear or solve.}
                \SY{sharing eases burdens}
                \VI{\textit{chia sẻ sẽ vơi bớt gánh nặng}.}
                \EX{Call me—\textbf{a problem shared is a problem halved}.}
                \EX{Peer-support programs operate on the idea that \textbf{a problem shared is a problem halved}.}
                \CO{live by/believe that a problem shared is a problem halved}
            \end{ExplainCard}

            \begin{ExplainCard}{comprehensible}{adj}
                \EN{able to be understood; clear enough to grasp.}
                \SY{understandable; intelligible; lucid}
                \VI{\textit{dễ hiểu}, có thể hiểu được.}
                \EX{Please keep the summary \textbf{comprehensible}.}
                \EX{Visualizations make complex data more \textbf{comprehensible} to non-experts.}
                \CO{readily/highly/broadly comprehensible; make sth comprehensible}
            \end{ExplainCard}

            \begin{ExplainCard}{daunting task}{collocation}
                \EN{a job that appears difficult and likely to demand great effort.}
                \SY{formidable challenge; arduous assignment}
                \VI{\textit{nhiệm vụ đáng ngại, khó nhằn}.}
                \EX{Moving house alone is a \textbf{daunting task}.}
                \EX{Coordinating cross-border trials is a \textbf{daunting task} for small labs.}
                \CO{face/tackle a daunting task; a truly/especially daunting task}
            \end{ExplainCard}

            \begin{ExplainCard}{enlist the help (of)}{phrase}
                \EN{to secure someone’s support or assistance.}
                \SY{recruit; obtain; call on}
                \VI{\textit{nhờ/thuê sự giúp đỡ (của ai)}.}
                \EX{We \textbf{enlisted the help of} neighbors for the move.}
                \EX{Researchers \textbf{enlisted the help of} clinicians to validate the tool.}
                \CO{enlist the help/support/services of}
            \end{ExplainCard}

            \begin{ExplainCard}{sound advice}{collocation}
                \EN{well-reasoned, reliable guidance.}
                \SY{solid counsel; prudent guidance}
                \VI{\textit{lời khuyên xác đáng}.}
                \EX{My mentor gave me \textbf{sound advice} about money.}
                \EX{\textbf{Sound advice} from advisors reduces startup failure rates.}
                \CO{seek/follow/offer sound advice}
            \end{ExplainCard}

            \begin{ExplainCard}{lend a sympathetic ear}{idiom}
                \EN{to listen with empathy and understanding.}
                \SY{listen compassionately; be a good listener}
                \VI{\textit{lắng nghe cảm thông}.}
                \EX{Thanks for \textbf{lending me a sympathetic ear}.}
                \EX{Support hotlines exist to \textbf{lend a sympathetic ear} to callers in distress.}
                \CO{lend/offer a sympathetic ear to}
            \end{ExplainCard}

            \begin{ExplainCard}{daily necessities}{collocation}
                \EN{basic items or needs required for everyday life.}
                \SY{essentials; staples; basic needs}
                \VI{\textit{nhu yếu phẩm hằng ngày}.}
                \EX{The store sells \textbf{daily necessities} like rice and soap.}
                \EX{Inflation in \textbf{daily necessities} disproportionately affects low-income households.}
                \CO{buy/afford/provide daily necessities}
            \end{ExplainCard}

            \begin{ExplainCard}{meet the deadlines}{phrase}
                \EN{to finish tasks by the required time.}
                \SY{hit deadlines; deliver on time; keep to schedule}
                \VI{\textit{kịp hạn}, hoàn thành trước thời điểm quy định.}
                \EX{We worked late to \textbf{meet the deadlines}.}
                \EX{Agile sprints help teams consistently \textbf{meet deadlines}.}
                \CO{struggle/fail to meet deadlines; meet tight/strict deadlines}
            \end{ExplainCard}

            \begin{ExplainCard}{have so much on (one’s) plate}{idiom}
                \EN{to be very busy with many duties or problems.}
                \SY{be swamped; be snowed under; be overloaded}
                \VI{\textit{có quá nhiều việc phải lo}.}
                \EX{I \textbf{have so much on my plate} this week.}
                \EX{Nurses often \textbf{have too much on their plate} during outbreaks.}
                \CO{already/always have a lot on your plate}
            \end{ExplainCard}

            \begin{ExplainCard}{overwhelmed}{adj}
                \EN{feeling unable to cope due to too much work or emotion.}
                \SY{overloaded; swamped; inundated}
                \VI{\textit{quá tải}, choáng ngợp.}
                \EX{She felt \textbf{overwhelmed} by emails.}
                \EX{First-year teachers report being \textbf{overwhelmed} by administrative tasks.}
                \CO{feel/get/become overwhelmed by/with}
            \end{ExplainCard}

            \begin{ExplainCard}{minimalist lifestyle}{n phrase}
                \EN{a way of living that intentionally keeps possessions and commitments to a simple, essential set.}
                \SY{simple living; pared-down lifestyle}
                \VI{\textit{lối sống tối giản}.}
                \EX{He adopted a \textbf{minimalist lifestyle} after moving.}
                \EX{A \textbf{minimalist lifestyle} can reduce consumption and environmental impact.}
                \CO{adopt/embrace/live a minimalist lifestyle}
            \end{ExplainCard}

            \begin{ExplainCard}{free up}{v}
                \EN{to make time, money, or resources available for use.}
                \SY{release; liberate; allocate}
                \VI{\textit{giải phóng}/\textit{dành ra}.}
                \EX{Automating reports \textbf{frees up} my evenings.}
                \EX{Cloud migration \textbf{frees up} capital for product development.}
                \CO{free up time/budget/space/resources}
            \end{ExplainCard}

            \begin{ExplainCard}{it goes without saying}{idiom}
                \EN{used to emphasize that something is obvious.}
                \SY{obviously; needless to say; self-evidently}
                \VI{\textit{khỏi phải nói}, quá rõ ràng.}
                \EX{\textbf{It goes without saying} that rest matters.}
                \EX{\textbf{It goes without saying} that data privacy is non-negotiable.}
                \CO{it goes without saying that + clause}
            \end{ExplainCard}

            \begin{ExplainCard}{bored to death}{idiom}
                \EN{extremely bored.}
                \SY{bored stiff; bored out of one’s mind}
                \VI{\textit{chán đến chết}.}
                \EX{I was \textbf{bored to death} during the wait.}
                \EX{Monotonous tasks leave employees \textbf{bored to death}, lowering output.}
                \CO{be/get bored to death with/by}
            \end{ExplainCard}

            \begin{ExplainCard}{sheer}{adj}
                \EN{complete and not mixed with anything else; used for emphasis.}
                \SY{utter; pure; absolute}
                \VI{\textit{thuần tuý, hoàn toàn} (nhấn mạnh).}
                \EX{They won by \textbf{sheer} luck.}
                \EX{\textbf{Sheer} volume of data necessitates automation.}
                \CO{sheer luck/joy/size/number}
            \end{ExplainCard}

            \begin{ExplainCard}{a sense of demotivation}{n phrase}
                \EN{a feeling of reduced drive or willingness to act.}
                \SY{loss of motivation; apathy; disinterest}
                \VI{\textit{cảm giác mất động lực}.}
                \EX{Repetitive tasks create \textbf{a sense of demotivation}.}
                \EX{Poor feedback loops foster \textbf{demotivation} across teams.}
                \CO{feel/experience demotivation; combat/address demotivation}
            \end{ExplainCard}

            \begin{ExplainCard}{morale}{n}
                \EN{the confidence and enthusiasm of a person or group.}
                \SY{spirit; esprit de corps; confidence}
                \VI{\textit{tinh thần}, sĩ khí.}
                \EX{Perks helped lift team \textbf{morale}.}
                \EX{Transparent leadership significantly improves employee \textbf{morale}.}
                \CO{boost/undermine morale; high/low morale}
            \end{ExplainCard}

            \begin{ExplainCard}{dissuade (sb) from (doing)}{v}
                \EN{to persuade someone not to do something.}
                \SY{discourage; deter; talk out of}
                \VI{\textit{ngăn can}, khuyên không làm.}
                \EX{They \textbf{dissuaded} me from quitting suddenly.}
                \EX{Clear warnings can \textbf{dissuade} risky online behaviors.}
                \CO{dissuade sb from action/plan; attempt to dissuade}
            \end{ExplainCard}

            \begin{ExplainCard}{(act/step) beyond one’s comfort zone}{phrase}
                \EN{to do things that feel unfamiliar or challenging.}
                \SY{step out of one’s comfort zone; push boundaries}
                \VI{\textit{bước ra khỏi vùng an toàn}.}
                \EX{Try speaking up—act \textbf{beyond your comfort zone}.}
                \EX{Experiential learning nudges students \textbf{beyond their comfort zones}.}
                \CO{move/venture/act beyond your comfort zone}
            \end{ExplainCard}

            \begin{ExplainCard}{a myriad of choices}{n phrase}
                \EN{an extremely large number of options.}
                \SY{a multitude of; countless; a wealth of}
                \VI{\textit{vô số lựa chọn}.}
                \EX{Streaming offers \textbf{a myriad of choices}.}
                \EX{Globalization presents consumers with \textbf{a myriad of choices} across markets.}
                \CO{offer/face a myriad of choices; a myriad of + plural noun}
            \end{ExplainCard}

            \begin{ExplainCard}{holiday destination}{n}
                \EN{a place people travel to for leisure.}
                \SY{vacation spot; resort; getaway}
                \VI{\textit{điểm du lịch nghỉ dưỡng}.}
                \EX{Da Nang is a popular \textbf{holiday destination}.}
                \EX{Marketing repositioned the city as a premium \textbf{holiday destination}.}
                \CO{popular/favorite holiday destination; choose/pick a destination}
            \end{ExplainCard}

            \begin{ExplainCard}{pave the way (for)}{idiom}
                \EN{to make later progress or development possible.}
                \SY{prepare the ground; enable; open the door}
                \VI{\textit{mở đường}, tạo điều kiện.}
                \EX{Online payments \textbf{paved the way for} e-commerce.}
                \EX{Seminal studies \textbf{paved the way for} modern vaccine design.}
                \CO{pave the way for reforms/innovation/adoption}
            \end{ExplainCard}

            \begin{ExplainCard}{precursor (to/of)}{n}
                \EN{something that comes before and leads to the development of another thing.}
                \SY{forerunner; predecessor; harbinger}
                \VI{\textit{tiền thân}, dấu hiệu báo trước.}
                \EX{Email was a \textbf{precursor to} today’s chat apps.}
                \EX{Early pilot programs served as a \textbf{precursor of} nationwide policy.}
                \CO{a precursor to/of X; serve as/act as a precursor}
            \end{ExplainCard}
        \end{VocabExplain}

    \begin{VocabHighlights}
        \VH{to stand up for}{(phr. v) to support or defend somebody/something}{(cụm động từ) ủng hộ, hỗ trợ ai}
        \VH{a friend in need is a friend indeed}{(proverb) a person who helps at a difficult time is a true person}{(tục ngữ) trong gian nan mới biết ai là bạn}
        \VH{fair-weather friends}{(phrase) a person who stops being a friend in times of difficulty}{(cụm từ) người bạn không tốt (bỏ bạn mình đi lúc khó khăn)}
        \VH{to turn one's back on}{(idiom) ignore (someone) by turning away, reject or abandon}{(thành ngữ) lơ đi}
        \VH{for the sake of}{(phrase) to get or keep something}{(cụm từ) để đạt được}
        \VH{to leave somebody in the lurch}{(idiom) to leave someone at a time when they need you to stay and help}{(thành ngữ) bỏ rơi ai lúc hoạn nạn}
        \VH{to have a little chit-chat}{(phrase) to have an informal conversation about matters that are not important}{(cụm từ) nói chuyện phiếm}
        \VH{to go through}{(phr. v) to experience or suffer something}{(cụm động từ) trải qua}
        \VH{to aid}{(v) to help somebody/something to do something, especially by making it easier}{(động từ) giúp ai làm gì}
        \VH{full-fledged}{(adj) completely developed; with all qualifications necessary}{(tính từ) đủ lông đủ cánh, đủ khả năng}
        \VH{to gain others' trust}{(phrase) to make others trust you}{(cụm từ) khiến người khác tin}
        \VH{to stick by something/somebody}{(phr. v) to continue to support something or someone, especially in a difficult situation}{(cụm động từ) tiếp tục ủng hộ, tin ai}
        \VH{undesirable}{(adj) not wanted, bad}{(tính từ) không được như mong muốn}
        \VH{to come fresh out of university}{(phrase) just graduate}{(cụm từ) vừa mới tốt nghiệp}
        \VH{to be a tall order}{(idiom) to be something that is difficult to do}{(thành ngữ) rất khó}
        \VH{economic recession}{(phrase) a period when the economy of a country is not successful and conditions for business are bad}{(cụm từ) suy thoái kinh tế}
        \VH{to go belly-up}{(idiom) if a company or plan goes belly up, it fails}{(thành ngữ) phá sản}
        \VH{to undertake more responsibilities}{(phrase) do more tasks}{(cụm từ) đảm nhiệm nhiều trách nhiệm}
        \VH{the crux of the matter}{(idiom) the focal, central, or most important element of a topic, problem, or issue}{(thành ngữ) mấu chốt của vấn đề là}
        \VH{rewarding}{(adj) satisfactory}{(tính từ) xứng đáng với công sức bỏ ra}
        \VH{to sit pretty}{(idiom) to be in a good, safe, or comfortable position}{(thành ngữ) sống thoải mái}
        \VH{in two minds about}{(idiom) unable to make a decision}{(thành ngữ) đắn đo}
        \VH{pros and cons}{(idiom) the favorable and the unfavorable factors or reasons; advantages and disadvantages}{(thành ngữ) lợi ích và bất lợi}
        \VH{to pick somebody's brain}{(idiom) to consult somebody}{(thành ngữ) tham vấn}
        \VH{to be touch and go}{(idiom) if a situation is touch-and-go, it is uncertain}{(thành ngữ) không chắc chắn}
        \VH{a turning point}{(phrase) the time at which a situation starts to change in an important way}{(cụm từ) điểm then chốt}
        \VH{on second thought}{(idiom) used when you want to change a decision you have made}{(thành ngữ) sau khi suy nghĩ kĩ}
        \VH{promotion opportunities}{(phrase) chances for holding higher positions}{(cụm từ) cơ hội thăng tiến}
        \VH{overwhelmingly}{(adv) strongly or completely}{(trạng từ) nhiệt tình}
        \VH{to have cold feet}{(idiom) feel frightened}{(thành ngữ) ớn lạnh, sợ hãi}
        \VH{carefree childhood}{(adj) a childhood without any worries or problems}{(tính từ) thời thơ ấu vô lo}
        \VH{as easy as falling off a log}{(phrase) very easy}{(cụm từ) dễ dàng}
        \VH{promotion}{(n) a move to a more important job or rank in a company or an organization}{(danh từ) thăng tiến; tiến bộ}
        \VH{vague}{(adj) not clear in a person's mind}{(tính từ) mập mờ}
        \VH{dissimilar}{(adj) not the same}{(tính từ) không giống; khác}
        \VH{to have somebody's voice in}{(phrase) be able to raise opinions}{(cụm từ) có tiếng nói}
        \VH{down the road}{(idiom) in the future}{(thành ngữ) trong tương lai}
        \VH{a problem shared is a problem halved}{(idiom) talking about a problem with someone else usually makes it seem less daunting or troubling}{(thành ngữ) một rắc rối được chia sẻ là một điều rắc rối đã được giải quyết tới một nửa rồi}
        \VH{comprehensible}{(adj) that can be understood by somebody}{(tính từ) có thể hiểu được; dễ hiểu}
        \VH{daunting}{(adj) making somebody feel nervous and less confident about doing something; likely to make somebody feel this way}{(tính từ) khó khăn; làm nản lòng}
        \VH{to enlist the help of}{(phrase) to persuade somebody to help you or to join you in doing something}{(cụm từ) nhờ vào sự giúp đỡ của ai đó}
        \VH{a sound advice}{(phrase) a good advice}{(cụm từ) một lời khuyên bổ ích}
        \VH{daily necessities}{(phrase) basic needs}{(cụm từ) nhu cầu thiết yếu}
        \VH{to meet the deadlines}{(phrase) to finish something at the time it is meant to be finished}{(cụm từ) hoàn thành đúng hạn}
        \VH{to have so much on somebody's plate}{(idiom) too busy}{(thành ngữ) quá bận rộn với nhiều thứ cùng một lúc}
        \VH{overwhelmed}{(adj) to have such a strong emotional effect on somebody that it is difficult for them to resist or know how to react}{(tính từ) bị quá tải}
        \VH{a minimalist lifestyle}{(n) a lifestyle which is using very simple ideas or a very small number of simple elements}{(danh từ) phong cách sống tối giản}
        \VH{to free up}{(phr. v) to make something available to be used}{(cụm động từ) giải phóng bớt}
        \VH{it goes without saying}{(idiom) it is obvious that}{(thành ngữ) rõ ràng là}
        \VH{bored to death}{(idiom) too boring}{(thành ngữ) quá buồn chán}
        \VH{sheer}{(adj) used to emphasize the size, degree or amount of something}{(tính từ) hoàn toàn (nhấn mạnh)}
        \VH{a sense of demotivation}{(phrase) feel less productive}{(cụm từ) cảm giác mất nghị lực, không có tinh thần làm gì}
        \VH{morale}{(n) the amount of confidence \& enthusiasm a person or a group has at a particular time}{(danh từ) nhuệ khí}
        \VH{to dissuade}{(v) persuade (someone) not to take a particular course of action}{(động từ) thuyết phục ai đó không làm gì}
        \VH{to act beyond somebody's comfort zones}{(phrase) to leave somebody's comfort zones}{(cụm từ) ra khỏi vùng an toàn}
        \VH{holiday destinations}{(phrase) a great place to travel to}{(cụm từ) điểm du lịch hấp dẫn}
        \VH{to pave the way for}{(phrase) to make the other thing possible}{(cụm từ) tạo điều kiện cho}
        \VH{a precursor}{(n) a person or thing that comes before somebody/something similar and that leads to or influences its development}{(danh từ) nguyên nhân; dấu hiệu báo trước}
    \end{VocabHighlights}

    \end{test}
\end{glossarymc}